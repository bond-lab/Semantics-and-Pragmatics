\documentclass[headrule,footrule]{foils}


%%
%%%  Macros
%%%
%%% fonts-sil-charis for IPA in week 5

\newcommand{\logo}{HG2002 (2021)}
\usepackage[hidelinks]{hyperref}

\newcommand{\header}[3]{%
  \title{\vspace*{-2ex} \large 
    HG2002 Semantics and Pragmatics
% \thanks{Creative
%       Commons Attribution License: 
%       you are free to share and adapt as long as you give 
%       appropriate credit and add no additional restrictions: 
%       \protect\url{https://creativecommons.org/licenses/by/4.0/}.
%     }
    \\[2ex] \Large  \emp{#2} \\ \emp{#3}}
  \author{\blu{Francis Bond}   \\ 
    \normalsize  \textbf{Division of Linguistics and Multilingual Studies}\\
    \normalsize  \url{http://www3.ntu.edu.sg/home/fcbond/}\\
    \normalsize  \texttt{bond@ieee.org}}
  \MyLogo{\logo}
  % \MyLogo{奈良女子大学:欧米言語情報理論II}
  \date{#1
    \\  \url{https://bond-lab.github.io/Semantics-and-Pragmatics/}
\\[.5ex] \footnotesize Creative  Commons Attribution License:  you are free to share and adapt 
\\[-.25ex] \footnotesize   as long as you give    appropriate credit and add no
additional restrictions: 
\\ \small  \protect\url{https://creativecommons.org/licenses/by/4.0/}.
}
  % \renewcommand{\logo}{#2}
  % \special{! /pdfmark where
  %   {pop} {userdict /pdfmark /cleartomark load put} ifelse
  %   [ /Author (Francis Bond)
  %   /Title (#1: #2)
  %   /Subject (HG2002: Semantics and Pragmatics)
  %   /Keywords (Semantics, Pragmatics, Meaning)
  %   /DOCINFO pdfmark}
  %   }
  \hypersetup{%
    final       = true,
    colorlinks  = true,
    urlcolor    = blue,
    citecolor   = blue,
    linkcolor   = MidnightBlue,
    unicode     = true,
    pdfauthor   = {Francis Bond},
    pdfkeywords = {Semantics, Pragmatics, Meaning},
    pdftitle    = {#1: #2},
    pdfsubject  = {HG2002 Semantics and Pragmatics; License CC BY 4.0}
  }
}


\usepackage[a4paper,landscape]{geometry}
%\usepackage[dvips]{xcolor}
\usepackage[dvipsnames,x11names]{xcolor}
\usepackage{graphicx}
\newcommand{\blu}[1]{\textcolor{blue}{#1}}
\newcommand{\grn}[1]{\textcolor{green}{#1}}
\newcommand{\hide}[1]{\textcolor{white}{#1}}
\newcommand{\emp}[1]{\textcolor{red}{#1}}
\newcommand{\txx}[1]{\textbf{\textcolor{blue}{#1}}}
\newcommand{\lex}[1]{\textbf{\mtcitestyle{#1}}}


\usepackage{amsmath,latexsym}
\usepackage{pifont}
\renewcommand{\labelitemi}{\textcolor{violet}{\ding{227}}}
\renewcommand{\labelitemii}{\textcolor{purple}{\ding{226}}}

\newcommand{\subhead}[1]{\noindent\textbf{#1}\\[5mm]}

\newcommand{\Bad}{\emp{\raisebox{0.15ex}{\ensuremath{\mathbf{\otimes}}}}}
\newcommand{\bad}{*}

\newcommand{\com}[1]{\hfill \textnormal{(\emp{#1})}}%
\newcommand{\cxm}[1]{\hfill \textnormal{(\txx{#1})}}%
\newcommand{\cmm}[1]{\hfill \textnormal{(#1)}}%

\usepackage{relsize,xspace}
\newcommand{\into}{\ensuremath{\rightarrow}\xspace}
\newcommand{\ent}{\ensuremath{\Rightarrow}\xspace}
\newcommand{\nent}{\ensuremath{\not\Rightarrow}\xspace}
\newcommand{\tot}{\ensuremath{\leftrightarrow}\xspace}
\usepackage{url}
\newcommand{\lurl}[1]{\MyLogo{\url{#1}}}

\usepackage{mygb4e}
\let\eachwordone=\itshape
\newcommand{\lx}[1]{\textbf{\textit{#1}}}
\newcommand{\ix}{\ex\it}

\newcommand{\cen}[2]{\multicolumn{#1}{c}{#2}}
%\usepackage{times}
%\usepackage{nttfoilhead}
\newcommand{\myslide}[1]{\foilhead[-25mm]{\raisebox{12mm}[0mm]{\emp{#1}}}\MyLogo{\logo}}
\newcommand{\myslider}[1]{\rotatefoilhead[-25mm]{\raisebox{12mm}[0mm]{\emp{#1}}}}
%\newcommand{\myslider}[1]{\rotatefoilhead{\raisebox{-8mm}{\emp{#1}}}}

\newcommand{\section}[1]{\myslide{}{\begin{center}\Huge \emp{#1}\end{center}}}



\usepackage[lyons,j,e,k]{mtg2e}
\renewcommand{\mtcitestyle}[1]{\textcolor{teal}{\textsl{#1}}}
%\renewcommand{\mtcitestyle}[1]{\textsl{#1}}
\newcommand{\ja}[1]{\mtcitestyle{\makexeCJKactive #1\makexeCJKinactive}}
\newcommand{\chn}{\mtciteform}
\newcommand{\zsm}{\mtciteform}
%\newcommand{\cmn}[1]{make\cjkactive\mtciteform#1\makecjkinactive}
\newcommand{\iz}[1]{\textup{\texttt{\textcolor{blue}{\textbf{#1}}}}}
\newcommand{\con}[1]{\textsc{#1}}
\newcommand{\gm}{\textsc}
\newcommand{\cmp}[1]{{[\textsc{#1}]}}
\newcommand{\sr}[1]{\ensuremath{\langle}#1\ensuremath{\rangle}}
\usepackage[normalem]{ulem}
\newcommand{\ul}{\uline}
\newcommand{\ull}{\uuline}
\newcommand{\wl}{\uwave}
\newcommand{\vs}{\ensuremath{\Leftrightarrow}~}
%%% theta role
\newcommand{\tr}[1]{\textcolor{Chartreuse4}{\textsc{#1}}}
%%% theta grid
\newcommand{\grid}[1]{\ensuremath{\langle}\tr{#1}{\ensuremath{\rangle}}}

%%%
%%% Bibliography
%%%
\usepackage{natbib}
%\usepackage{url}
\usepackage{bibentry}
%\usepackage{CJKutf8}


\usepackage{fontenc}
\usepackage{polyglossia}
\setmainlanguage{english}
\setotherlanguages{tamil}
\setmainfont[Ligatures=TeX]{TeX Gyre Pagella}
\setsansfont[Ligatures=TeX]{TeX Gyre Heros}
\newfontfamily\ipafont{Charis SIL}
\newcommand\ipa[1]{\mtcitestyle{\ipafont #1}}


\usepackage{xeCJK}
\makexeCJKinactive
\newcommand{\zh}[1]{\mtcitestyle{\makexeCJKactive #1\makexeCJKinactive}}
%\newcommand{\ja}[1]{\makexeCJKactive #1\makexeCJKinactive}
\setCJKmainfont{Noto Sans CJK JP}
\setCJKsansfont{Noto Sans CJK SC}
\setCJKmonofont{Noto Sans CJK SC}

\newfontfamily\tamilfont[Script=Tamil]{Noto Sans Tamil}
\newfontfamily\tamilfontsf[Script=Tamil]{Noto Sans Tamil}
\newcommand{\tam}[1]{\texttamil{#1}}
%%% From Tim
\newcommand{\WMngram}[1][]{$n$-gram#1\xspace}
\newcommand{\infers}{$\rightarrow$\xspace}


\usepackage{rtrees,qtree}
\renewcommand{\lf}[1]{\br{#1}{}}
\usepackage{avm}
%\avmoptions{topleft,center}
\newcommand{\ft}[1]{\textsc{#1}}
\renewcommand{\val}[1]{\textit{#1}}
\newcommand{\typ}[1]{\textit{#1}}
\avmfont{\sc}
\avmvalfont{\sc}
\renewcommand{\avmtreefont}{\sc}
\avmsortfont{\it}


%%% From CSLI book
\newcommand{\mc}{\multicolumn}
\newcommand{\HD}{\textbf{H}\xspace}
\newcommand{\el}{\< \>}
\makeatother
\long\def\smalltree#1{\leavevmode{\def\\{\cr\noalign{\vskip12pt}}%
\def\mc##1##2{\multispan{##1}{\hfil##2\hfil}}%
\tabskip=1em%
\hbox{\vtop{\halign{&\hfil##\hfil\cr
#1\crcr}}}}}
\makeatletter

%\usepackage{tipa}
\usepackage{multicol}


\newcommand{\task}{\marginpar{\large ~~~\textbf{?}}}
\newcommand{\sh}[1]{\href{https://www.arthur-conan-doyle.com/index.php?title=#1}{#1}}

\usepackage{tikz}
\usepackage{tikz-qtree}
\usepackage{forest}



\newcommand{\izn}[2]{\iz{#1:#2}}
\newcommand{\izb}[1]{(\iz{#1})}



\begin{document}

\header{Lecture 9}{Meaning Components}{}\maketitle

%\include{schedule}

\myslide{Overview}
\begin{itemize}\addtolength{\itemsep}{-1ex}
\item Revision: Speech as Action
  \begin{itemize}
  \item Austin's Speech Act Theory
  \item Categorizing Speech Acts
  \item Indirect Speech Acts
  \end{itemize}
  % \item Sentence Types
\item Componential Analysis
\item Katz's Semantic Theory
\item Levin's Verbal Alternations
\item Talmy's Cognitive Structure
\item Jackendoff's (Lexical) Conceptual Structure
\item Pustejovsky's Generative Lexicon
\item Next Lecture: Chapter 10 --- \emp{Formal Semantics}
\end{itemize}


%%%
%%% this changes each year, so keep separate
%%%

\MyLogo{}



\section{Revision: \\ Speech as Action}


\myslide{Speech as Action}

\begin{itemize}
\item Language is often used to \emp{do} things: \txx{speech acts}
  \\ language has both
  \begin{itemize}
  \item \txx{context dependence}
  \item \txx{interactivity}
  \end{itemize}
\item There are four syntactic types that correlate closely to pragmatic uses

  \begin{tabular}{lcl}
\textbf{Syntactic Type} &  & \textbf{Speech Act} \\
    \hline
  \txx{declarative}  &$\leftrightarrow$& \txx{assertion} \\
  \txx{interrogative} &$\leftrightarrow$& \txx{question} \\
  \txx{imperative} &$\leftrightarrow$& \txx{order} or \txx{command} \\
  \txx{optative} &$\leftrightarrow$& \txx{wish}
  \end{tabular}
\item Mismatches are \txx{indirect speech acts}
\end{itemize}

\myslide{Performative Utterances}

\begin{exe}
  \ex \eng{I promise I won't drive home}
  \ex \eng{I bet you 5 bucks they get caught}
  \ex \eng{I declare this lecture over} 
  \ex \eng{I warn you that legal action will ensue}
  \ex \eng{I name this ship \textit{the Lollipop}} 
\end{exe}

\begin{itemize}
\item Uttering these (in an appropriate context) \emp{is} acting
\\  \emp{Utterances themselves can be actions}
\item In English, we can signal this explicitly with \textit{hereby}
\end{itemize}

\myslide{Felicity Conditions}
\begin{itemize}
\item Performatives (vs Constantives) \hfill (Austin)
\\ Given the correct \txx{felicity conditions}
  \begin{description}
  \item[A1] There must exist an accepted conventional procedure that
    includes saying certain words by certain persons in certain
    circumstances,
  \item[A2] The circumstances must be appropriate for the invocation
  \item[B1] All participants must do it both correctly
  \item[B2] \ldots  and completely
  \item[C1] The intention must be to do this the act
  \item[C2] The participants must conduct themselves so subsequently.
  \end{description}
\item If the conditions don't hold, the speech act is \txx{infelicitous}
  \begin{itemize}
  \item Failing \textbf{A} or \textbf{B} is a \txx{misfire}
  \item Failing \textbf{C} is an \txx{abuse}
  \end{itemize}
  \end{itemize}

\myslide{Explicit and Implicit Performatives}
\begin{itemize}
\item \txx{Explicit Performatives}
  \begin{itemize}
  \item Tend to be first person
  \item The main verb  is a performative: 
    \eng{promise, warn, sentence, bet, pronounce, \ldots}
  \item You can use \eng{hereby}
  \end{itemize}
\item \txx{Implicit Performatives}
  \begin{exe}
    \ex \eng{You are hereby charged with treason}
    \ex \eng{Students are requested to be quiet in the halls}
    \ex \eng{10 bucks says they'll be late}
    \ex \eng{Come up and see me some time!}
  \end{exe}
  Can be made explicit by adding a performative verb
\end{itemize}

\myslide{Elements of Speech Acts}
\begin{description}
\item \txx{Locutionary act} the act of saying something that makes
  sense in a language
\item \txx{\underline{Illocutionary act}} the force of the statement  as intended by the speaker (not necessarily the surface interpretation)
\item \txx{Perlocutionary act} the effects of the statement
``such as persuading, convincing, scaring, enlightening, inspiring, or otherwise getting someone to do or realize something whether intended or not'' (Austin 1962)
\end{description}
% Illocutionary force indicating devices(IFID)
% \begin{itemize}
% \item   word order;  stress;    intonation contour;  punctuation; the mood of the verb
%  performative verbs: \eng{I (Vp) you that \ldots} 
% \end{itemize}

\myslide{Searle's speech act classification}
  \begin{description}
  \item \txx{Declarative} changes the world (like performatives)
  \item \txx{Representative} describes the (speaker's view of the) world 
  \item \txx{Expressives}  express how the speaker feels
  \item \txx{Directives} get someone else to do something
  \item \txx{Comissives} commit oneself to a future action
  \end{description}


\myslide{Literal and non-literal uses}

\begin{exe}
  \ex
  \begin{xlist}
    \ex \eng{Could you get that?} 
    \ex \eng{Please get pass the salt.}
  \end{xlist}
  \ex 
  \begin{xlist}
    \ex \eng{I wish you wouldn't do that.}
    \ex \eng{Please don't do that.}
  \end{xlist}
  \ex
  \begin{xlist}
    \ex \eng{ You left the door open.}
    \ex \eng{Please close the door.}
  \end{xlist}
\end{exe}
\begin{itemize}
\item People have access to both the literal and non-literal meanings
\item Non literal meanings can be slower to understand
\item Some non-literal uses are very conventionalized 
  \\ \eng{Can/Could you X?} \into \eng{Please X}
\item In general: questioning the felicity conditions produces an indirect version
\end{itemize}

\myslide{Why be Indirect?}

\begin{itemize}
\item Mainly for politeness
\begin{itemize}
\item \txx{Positive Face} desire to seem worthy and deserving of approval
\item \txx{Negative Face} desire to be autonomous, unimpeded by others
\item Threats to another’s face
  \begin{itemize}
  \item to positive: disapproval, disagreement, interruption
  \item to negative: orders, requests, suggestions
  \end{itemize}
\item Face-saving acts: 
  \begin{itemize}
  \item don't threaten another’s face: \eng{I may be wrong but, \ldots}
  \item allow for negative face: \eng{Could you please, \ldots}
  \end{itemize}

\end{itemize}
\end{itemize}

% \section{Sentence Types}

\section{Componential Analysis}

\myslide{Break word meaning into its components}
\MyLogo{Inspired by work on phonetics in the Prague School}
\begin{itemize}
\item For example:
  \\[2ex] \begin{tabular}{lllll}
    \lex{woman} & \cmp{female} & \cmp{adult} & \cmp{human} & \\
    \lex{spinster} & \cmp{female} & \cmp{adult} & \cmp{human} & \cmp{unmarried} \\
    \lex{bachelor} & \cmp{male} & \cmp{adult} & \cmp{human} & \cmp{unmarried} \\
    \lex{wife} & \cmp{female} & \cmp{adult} & \cmp{human} & \cmp{married} \\
    \lex{girl} & \cmp{female} & \cmp{child} & \cmp{human} & \\
    \lex{boy} & \cmp{male} & \cmp{child} & \cmp{human} & \\
  \end{tabular}
  \\[2ex] \txx{semantic components}/\txx{primitives} shown as \cmp{component}
  \begin{itemize}
  \item components allow a compact description
  \item interact with morphology/syntax
  \item form part of our cognitive architecture
  \end{itemize}
\end{itemize}

\myslide{Defining Relations using Components}

\begin{itemize}
\item \txx{hyponymy}
  \begin{quote}
    A lexical item P is a hyponym of Q if all the components of Q are also in P.
  \end{quote}
  \begin{tabular}{lllll}
    \lex{woman} & \cmp{female} & \cmp{adult} & \cmp{human} & \\
    \lex{spinster} & \cmp{female} & \cmp{adult} & \cmp{human} & \cmp{unmarried} \\
%    \lex{bachelor} & \cmp{male} & \cmp{adult} & \cmp{human} & \cmp{unmarried} \\
    \lex{wife} & \cmp{female} & \cmp{adult} & \cmp{human} & \cmp{married} \\
  \end{tabular}
\\[2ex]  \lex{spinster} $\subset$ \lex{woman}; \lex{wife} $\subset$ \lex{woman}
\item \txx{incompatibility}
  \begin{quote}
    A lexical item P is incompatible with Q if they share some
    components but differ in one or more \txx{contrasting} components
  \end{quote}
%   \begin{tabular}{lllll}
% %    \lex{woman} & \cmp{female} & \cmp{adult} & \cmp{human} & \\
%     \lex{spinster} & \cmp{female} & \cmp{adult} & \cmp{human} & \cmp{unmarried} \\
% %    \lex{bachelor} & \cmp{male} & \cmp{adult} & \cmp{human} & \cmp{unmarried} \\
%     \lex{wife} & \cmp{female} & \cmp{adult} & \cmp{human} & \cmp{married} \\
%   \end{tabular}
  \lex{spinster} $\not\approx$ \lex{wife}

\end{itemize}

\myslide{Binary Features}

\begin{itemize}
\item We can make things more economical (fewer components):
  \\[2ex] \begin{tabular}{lllll}
    \lex{woman} & \cmp{+female} & \cmp{+adult} & \cmp{+human} & \\
    \lex{spinster} & \cmp{+female} & \cmp{+adult} & \cmp{+human} & \cmp{--married} \\
    \lex{bachelor} & \cmp{--female} & \cmp{+adult} & \cmp{+human} & \cmp{--married} \\
    \lex{wife} & \cmp{+female} & \cmp{+adult} & \cmp{+human} & \cmp{+married} \\
    \lex{girl} & \cmp{+female} & \cmp{-adult} & \cmp{+human} & \\
  \end{tabular}
  \begin{itemize}
  \item Which should be $+$? \cmp{+female} or \cmp{--male}
  \item Presumably also \cmp{--electric}, \cmp{--conical}, \ldots
    \\ Only show \txx{relevant} features
  \item \txx{antonyms} differ in only one binary component
  \end{itemize}
\end{itemize}

\myslide{Redundancy Rules}

\begin{itemize}
\item We can add relations between components:
\\[2ex]  \begin{tabular}{llll}
     \cmp{+human} & \into & \cmp{+animate}  \\
     \cmp{+adult} & \into & \cmp{+animate}  \\
     \cmp{+animate} & \into & \cmp{+concrete}  \\
     \cmp{+married} & \into & \cmp{+adult}  \\
     \cmp{+married} & \into & \cmp{+human}   & \ldots
  \end{tabular}
\item Which allows us to write:
  \\[2ex] \begin{tabular}{lllll}
    \lex{woman} & \cmp{+female} & \cmp{+adult} & \cmp{+human} & \\
    \lex{spinster} & \cmp{+female} & \cmp{+adult} & \cmp{+human} & \cmp{--married} \\
    \lex{bachelor} & \cmp{--female} & \cmp{+adult} & \cmp{+human} & \cmp{--married} \\
    \lex{wife} & \cmp{+female} & &   & \cmp{+married} 
  \end{tabular}
  \\[2ex] Can we say  \cmp{--married}  \into\  \cmp{+human}?
\end{itemize} 

\myslide{More Complex Breakdowns}

\begin{itemize}
\item We can add relations between components:
\\[2ex]  \begin{tabular}{llll}
     \cmp{+father} & \into & \cmp{+male} \cmp{+parent}  \\
     \cmp{+father}(\ul{x},y) & \into & \cmp{+male}(x) \cmp{+parent}(x,y) \\
     \cmp{+son}(\ul{x},y) & \into & \cmp{+male}(x) \cmp{+parent}(y,x) \\
     \cmp{+brother}(\ul{x},y) & \into & \cmp{+male}(x) 
     \cmp{+parent}(z,x) \cmp{+parent}(z,y) \\
     \cmp{+grandfather}(\ul{x},y) & \into & \cmp{+male}(x) 
     \cmp{+parent}(x,z)  \cmp{+parent}(z,y) \\
  \end{tabular}
\item Assume \cmp{+parent}(x,y)  means ``x is the parent of y''
\item There are various ways you can formalize such relationships
  \begin{itemize}
  \item Many parts of language can be formalized in such a way
  \item Can you do this for demonstratives?
    \\ \lex{this, that, these, those, what, here, there, where}\task
  \end{itemize}
\end{itemize}


%%% FIXME inheritance

\section{Katz’s Semantic Theory}

\myslide{Katz’s Semantic Theory}
\MyLogo{\txx{Compositional} : the meaning of the whole depends only on the meanings of the parts and the method of combination.}
\begin{itemize}
\item Two Central Ideas:
  \begin{itemize}
  \item Semantic rules must be recursive \\ to deal with infinite meaning
  \item Semantic rules interact with syntactic rules \\ to build up meaning,
    which is \txx{compositional}
  \end{itemize}
\item Two major components:
  \begin{itemize}
  \item A dictionary pairing lexical items with semantic representations
  \item A set of \txx{projection rules} that show how meaning is built up
  \end{itemize}
\end{itemize}

\myslide{The dictionary}
\MyLogo{Similar to \txx{genus} and \txx{differentiae}.}
\begin{itemize}
\item \lex{bachelor} \{N\}
  \begin{enumerate}
  \item \izb{human} \izb{male} [one who has never been married]
  \item \izb{human} \izb{male} [young knight serving under the standard of another knight]
  \item \izb{human} [one who has the lowest academic degree]
  \item \izb{animal} \izb{male} [young fur seal without a mate in the breeding season]
  \end{enumerate}
\item \izb{semantic markers} are the links that bind lexical items
  together in lexical relations
\item {[\txx{distinguishers}]} serve to identify this particular lexical item
    \\ this information is not relevant to syntax
\end{itemize}

\myslide{Projection Rules}
\MyLogo{More about this in Theories of Syntax/HPSG}

\begin{enumerate}
\item Projection rules combine with syntactic rules to produce the
  meaning of a sentence
  \begin{itemize}
  \item Information is passed up the tree and collected at the top.
    \begin{itemize}
    \item Information is only added, never deleted
    \item It must come from words or rules (or constructions)
    \end{itemize}
  \end{itemize}
\item \txx{Selectional restrictions} \sr{} help to reduce ambiguity and
  limit the possible readings
\end{enumerate}

\myslide{Selectional restrictions} 

\MyLogo{Modern theories prefer \txx{selectional preferences}:
  probabilities not categories.}
\begin{enumerate}
\item \lex{colorful} \{adj\}
  \begin{enumerate}
  \item\label{A} \izb{color} [abounding in contrast or variety of bright colors) 
    \sr{\izb{physical object} or \izb{social activity}}
  \item\label{B} \izb{evaluative} [having distinctive character, vividness or picturesqueness) 
    \sr{\izb{aesthetic object} or \izb{social activity}}
  \end{enumerate}

\item \lex{ball} \{N\}
  \begin{enumerate}
  \item\label{D} \izb{social activity} \izb{large} \izb{assembly} [for the purpose of social dancing]
  \item\label{E} \izb{physical object} [having globular shape]
  \item\label{F} \izb{physical object} [solid missile for project by engine of war]
  \end{enumerate}
\end{enumerate}
\begin{itemize}
\item \eng{colorful ball}: The selectional restrictions rule out: \ref{B} $+$ \ref{E},  \ref{B} $+$ \ref{F}
\end{itemize}

  
\section{Grammatical Rules and Semantic Components}
\MyLogo{}

\myslide{Verb Classification}
\MyLogo{\citep{Levin:1993}}

\begin{itemize}
\item We can investigate the meaning of a verb by looking at its
  grammatical behavior
  \begin{exe}
    \ex Consider the following transitive verbs
    \begin{xlist}
      \ex \eng{Margaret \ul{cut} the bread}
      \ex \eng{Janet \ul{broke} the vase}
      \ex \eng{Terry \ul{touched} the cat}
      \ex \eng{Carla \ul{hit} the door}
    \end{xlist}
  \end{exe}
\item These do not all allow the same argument structure alternations

\end{itemize}
\myslide{Diathesis Alternations}

\begin{itemize}
\item \txx{Causative/inchoative} alternation:
  \begin{quote}
    \eng{Kim \ul{broke} the window} $\leftrightarrow$ \eng{The window \ul{broke}}
    \\ also \eng{the window \ul{is broken}} (state)
  \end{quote}
\item \txx{Middle construction} alternation:
  \begin{quote}
    \eng{Kim \ul{cut} the bread} $\leftrightarrow$ \eng{The bread \ul{cut} easily}
  \end{quote}
\item \txx{Conative} alternation:
  \begin{quote}
    \eng{Kim \ul{hit} the door} $\leftrightarrow$ \eng{Kim \ul{hit} at the door}
  \end{quote}
\item \txx{Body-part possessor ascension} alternation:
  \begin{quote}
    \eng{Kim \ul{cut} Sandy's arm} $\leftrightarrow$ \eng{Kim \ul{cut}
      Sandy on the arm} 
  \end{quote}
\end{itemize}




\myslide{Diathesis Alternations and Verb Classes}

\MyLogo{\citep{Levin:1993}}

\begin{itemize}
\item A verb's (in)compatibility with different alternations is a strong
  predictor of its lexical semantics:
  \begin{quote}\smaller[1]
    \begin{tabular}{lcccc}
      & \lex{break} & \lex{cut} & \lex{hit} & \lex{touch} \\
      Causative & YES & NO & NO & NO \\
      Middle & YES & YES & NO & NO \\
      Conative & NO & YES & YES & NO \\
      Body-part & NO & YES & YES & YES \\
    \end{tabular}
    \vspace{3ex}

    \larger[1]
    \lex{break} = \{\eng{break, chip, crack, crash, crush, ...}\}\\
    \lex{cut} = \{\eng{chip, clip, cut, hack, hew, saw, ...}\}\\
    \lex{hit} = \{\eng{bang, bash, batter, beat, bump, ...}\}\\
    \lex{touch} = \{\eng{caress, graze, kiss, lick, nudge, ...}\}
  \end{quote}

\newpage

\item We can analyze components that correlate with the alternations
  \\[2ex]
  \begin{tabular}{ll}
  \lex{break} & \textsc{cause, change}\\
  \lex{cut}   & \textsc{cause, change, contact, motion}\\ 
  \lex{hit}   & \textsc{contact, motion}\\
  \lex{touch} & \textsc{contact}
  \end{tabular}
\item The semantic class/components  predicts the syntax of novel words
\item Not all parts of meaning are relevant to  syntax 
\\[1ex]
\begin{small}
  \begin{tabular}{ll}
    has an affect & has no affect \\ \hline
    Semantic Markers & Semantic Distinguishers \\
    Grammatically Relevant Subsystem & Unrestricted Conceptual Representation \\
    Semantic Structure & Semantic Content \\
    Semantic Form & Conceptual Structure \\
    Semantic Structure & Conceptual Structure
  \end{tabular}
\end{small}

\end{itemize}

\myslide{Thematic Roles and Linking Rules}

\begin{itemize}
\item Verbs often link their thematic roles to arguments in different ways
  \begin{exe}
    \ex
    \begin{xlist}
      \ex \eng{He loaded newspapers onto the van}  \sr{\ul{AGENT}, THEME} 
      \ex \label{complete} 
      \eng{He loaded the van with newspapers} \sr{\ul{AGENT}, GOAL} 
    \end{xlist}
  \end{exe}
\item But the meanings are not identical: (\ref{complete}) implies
  completion, and the theta-grid does not deal with the adjuncts
\item We need more than just theta-grids/roles

\end{itemize}

\myslide{Movement-to-location verbs}
\begin{itemize}
\item \txx{locative alternation}
\begin{exe}
  \ex
  \begin{xlist}
    \ex \eng{Andy poured oil into the pan}
    \ex *\eng{Andy poured the pan with oil}
  \end{xlist}
  \ex
  \begin{xlist}
    \ex *\eng{Andy filled oil into the pan}
    \ex \eng{Andy filled the pan with oil}
  \end{xlist}
  \ex
  \begin{xlist}
    \ex \eng{Andy brushed oil onto the pan}
    \ex \eng{Andy brushed the pan with oil}
  \end{xlist}
  \ex
  \begin{xlist}
    \ex \sr{\ul{AGENT}, THEME, PP:GOAL}
    \ex \sr{\ul{AGENT}, PATIENT, PP:INSTRUMENT$?$}
  \end{xlist}
\end{exe}
\end{itemize}  

\myslide{Explain with verb classes}
\MyLogo{\Bad Slightly circular: alternations motivate classes which explain alternations}

\begin{itemize}
\item Verbs of movement: `X causes Y to move into/onto Z'
  \begin{enumerate}
  \item Simple motion verbs: \lex{put, push}
  \item Manner specified: \lex{pour, drip, slosh}
  \end{enumerate}
 \eng{X puts Y on Z}
\item Verbs of change of state: `X causes Z to change state by means
  of moving Y into/onto Z': \lex{fill, coat, cover}
\\ [1.5ex]\eng{X fills Z with Y}
\item Verbs of movement `X causes Y to move into/onto Z' which also
  describe a kind of motion which causes an effect on the entity Z: 
  \lex{spray, paint, brush}
\\ [1.5ex]\eng{X paints Z with Y}

\end{itemize}


\section{Components \\ and Conflation Patterns}
\MyLogo{}

\myslide{Cognitive Semantics}
\MyLogo{\citep{Talmy:2000}}
\begin{itemize}
\item Major semantic components of Motion:
\begin{itemize}
\item \txx{Figure}: object moving or located with respect to the \txx{ground} 
\item \txx{Ground}: reference object
\item \txx{Motion}: the presence of movement of location in the event
\item \txx{Path}: the course followed or site occupied by the Figure w.r.t. the Ground.
\item \txx{Manner}: the type of motion
\end{itemize}
\begin{exe}
  \ex \gll \eng{Kim} \eng{swam} \eng{away from} \eng{the crocodile} \\
  Figure Manner Path Ground \\
  \ex \gll \eng{The banana} \eng{hung} \eng{from} \eng{the tree} \\
  Figure Manner Path Ground \\
\end{exe}

\item These are lexicalized differently in different languages.
\end{itemize}
\myslide{Different Lexicalizations of Movement}
\MyLogo{}
\begin{itemize}
\item English: Manner in verb, Path as adjunct
  \begin{exe}
    \ex \eng{The bottle floated into the cave}
    \ex \eng{They rolled the keg into the party}
  \end{exe}
\item Spanish: Path in verb, Manner  as adjunct
  \begin{exe}
    \ex \gll \eng{La} \eng{botella} \eng{entr\'o} \eng{a} \eng{la} \eng{cueva} \eng{flotando} \\
    the bottle moved-in to the cave floating \\
    \trans ``The bottle entered the cave, floating''
    \ex \gll \eng{Met\'{\i}} \eng{el} \eng{barril} \eng{a} \eng{la}
    \eng{bodega} \eng{rodandolo}  \\
    I-moved-in the barrel to the storeroom rolling \\
    \trans ``I put the keg into the storeroom, rolling''
  \end{exe}
\end{itemize}

\myslide{Typology of Motion in Languages}
\bigskip
\begin{center}
\begin{tabular}{ll}
  \textbf{Language (Family)} & \textbf{Verb Conflation Pattern} \\ \hline
  Romance, Semitic, Polynesian, \ldots & Path + fact-of-Motion \\
  Indo-European ($-$ Romance), Chinese & Manner/Cause + fact-of-Motion \\
  Navajo, Atsuwegei, \ldots & Figure + fact-of-Motion 
\end{tabular}
\end{center}

\begin{itemize}
\item \txx{verb-framed} (Motion with Path)
\item \txx{satellite-framed} (Motion with Manner)
\item Which group is this from?
 \begin{exe}
    \ex \glll \ja{樽~を} \ja{倉庫~に} \ja{転がして} \ja {入れた} \\
    taru-wo souko-ni korogasite ireta \\
    barrel-\textsc{acc} storeroom-\textsc{to} rolling put\\
    \trans ``I put the keg into the storeroom, rolling''
  \end{exe}

\end{itemize}

\section{Jackendoff’s \\ Conceptual Semantics: 
  \\ Lexical Conceptual Structure} 

\myslide{Describing Mental Representations}
\MyLogo{\citep{Jackendoff:1990,Jackendoff:1997}}
\begin{itemize}
\item An attempt to explain how we think
\item \txx{Mentalist Postulate}
  \begin{quote}
    Meaning in natural language is an information structure that is
    mentally encoded by human beings
  \end{quote}
\item Try to capture regularities
  \\[2ex]\begin{tabular}{lll}
    x lifted y & entails & y rose \\
    x gave z to y & entails & y received z \\
    x persuaded y that P & entails & y came to believe P \\[2ex]
    x cause E to occur & entails & E occurs 
  \end{tabular}

\item Also linked to vision and music (through X-bar theory)
\end{itemize}

\myslide{Semantic Components}

\begin{itemize}
  \item Universal Semantic Categories
  \begin{itemize}
  \item \txx{Event}
  \item \txx{State}
  \item \txx{Material Thing/Object}
  \item \txx{Path}
  \item \txx{Place}
  \item \txx{Property}
  \end{itemize}
  \begin{exe}
  \ex 
  \begin{xlist}
    \ex {[$_S$ [$_{NP}$ Bobby] 
      [$_{VP}$ [$_{V}$ went] [$_{PP}$ [$_{P}$ into] [$_{NP}$ the house]]]]}
    \ex {[$_{Event}$ GO ([$_{Thing}$ BOBBY], 
      [$_{Path}$ TO ([$_{Place}$ IN ([$_{Thing}$ house])])])]}
  \end{xlist}
  \end{exe}
\end{itemize}

\myslide{Motion as a tree}

\begin{multicols}{2}
  \begin{exe}
    \ex \eng{Bobby went into the house}
    \ex ``Bobby traverses a path that terminates at the interior of the house''
    \ex
    \begin{tree}
      \br{Event}{\lf{GO}
        \br{Thing}{ \lf{BOBBY}}
        \br{Path}{\lf{TO}
          \br{Place}{\lf{IN}\br{Thing}{\lf{HOUSE}}}}}
    \end{tree}
    %\newpage
    \ex \eng{The car is in the garage}
    \ex ``The car is in the state located in the interior of the garage''
    \ex
    \begin{tree}
      \br{State}{\lf{BE-LOC}
        \br{Thing}{ \lf{CAR}}
        \br{Place}{\lf{IN}\br{Thing}{\lf{GARAGE}}}}
    \end{tree}
\end{exe}
\end{multicols}


\myslide{Extend Location in three ways}

\begin{small}
  \noindent\begin{tabular}{lll}
    \textbf{Semantic Field} & BE (state) & GO (event) \\
    spatial location     & \eng{Jo is in the club}             
    & \eng{Alex went into the house}          \\
    temporal location   & \eng{The exam is on Wednesday}      
    & \eng{The exam moved to Thursday}        \\
    property ascription & \eng{The class is full}             
    & \eng{The class went from full to empty} \\
    possession          & \eng{This theory belongs to Ann Elk}
    & \eng{The prize went to JC}              \\
  \end{tabular}
\end{small}

  \begin{itemize}
  \item Break down the meaning into components
  \end{itemize}

\begin{exe}
  \ex 
  \begin{xlist}
    \ex \eng{The pool emptied}
    \ex {[$_{Event}$ INCH ([$_{State}$ BE-IDENT  
      ([$_{Thing}$ POOL], [$_{Place}$ AT ([$_{Property}$ EMPTY])])])]}
  \end{xlist}
    \ex 
    \begin{xlist}
      \ex \eng{Sandy emptied the pool}
      \ex {[$_{Event}$ CAUSE ([$_{Thing}$ SANDY], [$_{Event}$ INCH ([$_{State}$ BE-IDENT  
      ([$_{Thing}$ POOL], [$_{Place}$ AT ([$_{Property}$ EMPTY])])])])]}
  \end{xlist}
\end{exe}


\myslide{THING: Boundedness and Internal Structure}
\begin{itemize}
\item Two components:
\\[2ex]  \begin{tabular}{llll}
    Boundedness & Internal Struct. & Type & Example\\ \hline
    $+$b & $-$i & \txx{individuals} & \eng{a dog}/\eng{two dogs}\\
    $+$b & $+$i & \txx{groups}      & \eng{a committee}\\
    $-$b & $-$i & \txx{substance}s  & \eng{water}\\
    $-$b & $+$i & \txx{aggregates}  & \eng{buses, cattle}
  \end{tabular}

\item This can be extended to verb aspect (the verb event is also [$\pm$b, $\pm$i]).
  \\ \lex{sleep} [$-$b], \lex{cough} [$+$b], \lex{eat}  [$\pm$b]
 \begin{exe}
   \ex \eng{Bill ate two hot dogs in two hours.}
   \ex *\eng{Bill ate hot dogs in two hours.}
   \ex $^\#$\eng{Bill ate two hot dogs for two hours.}
   \ex \eng{Bill ate hot dogs for two hours.}
\end{exe}
\end{itemize}

\myslide{Conversion: Boundedness and Internal Structure}
\MyLogo{See \citet{Bond:2005} for an extension to Japanese and computational implementation.}
\begin{itemize}
\item Including
 \\[2ex] \begin{tabular}{lll}
  \txx{plural} & {[+b, --i] \into\ [--b, +i]} &     
    \eng{brick}  \into\ \eng{bricks} \\
  \txx{composed of} &{[--b, +i] \into\ [+b, --i]} &
     \eng{bricks}  \into\ \eng{house of bricks} \\
  \txx{containing} &   {[--b, --i] \into\ [+b, --i]} &
     \eng{coffee}  \into\ \eng{a cup of coffee/a coffee}
  \end{tabular}
\item Excluding
  \\[2ex] \begin{tabular}{lll}
    \txx{element}  & {[--b,+i] \into\ [+b, --i]} &     
    \eng{grain of rice} \\
    \txx{partitive} & {[--b, $\pm$i] \into\ [+b, --i]} &     
    \eng{top of the mountain}, \\
 & &  \eng{one of the dogs} \\
    \txx{universal grinder} &  {[+b, --i] \into\ [--b, --i]} &     
    \eng{There's \ul{dog} all over the road}
  \end{tabular}
\end{itemize}

There are other types of conversion, such as type: \eng{I drank three
  [types of] beers last night: stout, lager and amber ale}.



\section{Pustejovsky’s \\Generative Lexicon}
\MyLogo{}

\myslide{The Generative Lexicon}
\MyLogo{\citet{Pustejovsky:1995}}
\begin{itemize}
\item This brings in more encyclopedic knowledge
\item Each lexical entry can have:\\
  \textsc{argument structure} \\
  \textsc{event structure}\\
  \textsc{lexical inheritance structure} \\
  \textsc{qualia structure}:\\[1ex]
  \begin{tabular}{ll}
    \textsc{constitutive} & constituent parts \\
    \textsc{formal} & relation to other things \\
    \textsc{telic} & purpose \\
    \textsc{agentive}  & how it is made
  \end{tabular}
\item Interpretation is \txx{generated} by combining word meanings
\item Lexical Inheritance shows how words are related to other concepts in the lexicon
\end{itemize}





\myslide{The ideas behind the Generative Lexicon}
\begin{itemize}
\item Word meaning is decomposed, so that it can be composed with other words
\item The range of composition teaches us something about the internal
  structure of the word
  \begin{itemize}
  \item Rich Representation: lexical decomposition
  \item Rich Rules: coercion, sub-selection, co-composition
  \end{itemize}

\end{itemize}

\myslide{Event Structure}

\begin{itemize}
\item Events have \textbf{complex} structure
  \begin{itemize}
  \item \txx{State} 
    \begin{tree}
      \br{S}{\lf{e}}
    \end{tree}
    \\[1ex]    \lex{understand, love, be tall}
  \item \txx{Process}
    \begin{tree}
      \br{P}{\tlf{e$_1$ \ldots e$_n$}}
    \end{tree}
    \\[1ex]     \lex{sing, walk, swim}
  \item \txx{Transition}
    \begin{tree}
      \br{T}{\lf{E$_1$} \lf{$\neg$E$_2$}}
    \end{tree}
    \\[1ex]     \lex{open, close, build}
    \\ For an achievement, typically  E$_1$ = $\neg$e$_1$; E$_2$ = e$_1$
  \end{itemize}
\end{itemize}

\myslide{Different Alternations}
  \begin{exe}
    \ex \eng{The door closed}
   \begin{tree}
      \br{T}{\br{P}{\br{[$\neg$ closed(door)]}{}}
        \br{S}{\br{[closed(door)]}{}}} 
    \end{tree}
    \ex \eng{Jamie closed the door}
    \begin{tree}
      \br{T}{\br{P}{\br{[act(j, door) $\wedge$ $\neg$ closed(door)]}{}}
        \br{S}{\br{[closed(door)]}{}}} 
    \end{tree}
    \ex \eng{The door is closed}
   \begin{tree}
      \br{S}{\br{e}{\br{[closed(door)]}{}}}
    \end{tree}
  \end{exe}

\myslide{Modifier Ambiguity}
  \begin{exe}
    \ex \eng{Jamie closed the door rudely}
  \begin{xlist}
    \ex \eng{Jamie closed the door in a rude way [with his foot]}
    \\\begin{tree}
      \br{T}{\br{P [rude(P)]}{\br{[act(j, door) $\wedge$ $\neg$ closed(door)]}{}}
        \br{S}{\br{[closed(door)]}{}}} 
    \end{tree}
    \ex \eng{It was rude of Jamie to close the door}
    \\\begin{tree}
      \br{T [rude(T)]}{\br{P}{\br{[act(j, door) $\wedge$ $\neg$ closed(door)]}{}}
        \br{S}{\br{[closed(door)]}{}}} 
    \end{tree}
  \end{xlist}
  \end{exe}

\myslide{Qualia Structure}
\MyLogo{See \citet{Bond:1997b} for an account of numeral classifiers using the GL}


\begin{exe}
  \ex \eng{fast typist}
  \begin{xlist}
    \ex\label{ta} a typist who is fast [at running]
    \ex\label{tb} a typist who types fast
  \end{xlist}
%%% FIXME  \ex \eng{Joan baked the potato}
\end{exe}
\begin{itemize}
\item typist 
 \begin{avm} \[
    \textsc{argstr} &
    \[
      \textsc{arg1} & \iz{x:typist}\\
    \]\\
    \textsc{qualia} &
    \[
      \textsc{formal} & \[ \iz{x}  [ $\subset$ \iz{person} ]\] \\
      \textsc{telic} & \[ \iz{type(e,x)}  \]
      \] 
 \]
\end{avm}
\item (\ref{ta}) \eng{fast} modifies $x$
\item (\ref{tb}) \eng{fast} modifies $e$
\end{itemize}


\myslide{Problems with Components of Meaning}
\begin{itemize}
\item Primitives are the same as necessary and sufficient conditions
  \\ it is impossible to agree on the definitions
  \\ but they allow us to state generalizations better
\item Don't capture all aspects of meaning
\item Psycho-linguistic evidence is weak
\item It is just \txx{markerese} which still needs to be explained,
  there is no \txx{grounding}
\item Recent work replaces components with inheritance or dimensions
  \begin{itemize}
  \item \lex{boy}$_1$ $\subset$ \lex{male}$_1$ $\wedge$  $\subset$
    \lex{child}$_1$
  \item \lex{boy}$_1$ near \lex{male}$_1$ on some dimensions; near
    \lex{child}$_1$ on others
  \item same generalizations, more psychologically plausible
  \end{itemize}
\end{itemize}

\myslide{Conclusion}

\begin{itemize}
\item Meaning can be broken up into units smaller than words:  \txx{components} 
  \begin{itemize}
  \item These can be combined to make larger meanings
  \item At least some of them influence syntax
  \item They may be psychologically real
  \item Many parts of meaning can be treated in this way
  \end{itemize}
% \item \emp{Note:} Similar facts can be handled using multiple
%   inheritance, this tends to be more common in modern approaches
\item \emp{Note:} Selectional restrictions are too strict,
  selectional preferences (giving prototypical arguments and measuring
  the similarity) are more common in modern approaches: 
\\ \txx{assigning  probabilities to interpretations}
\end{itemize}


% \myslide{Acknowledgments and References}
% \MyLogo{}
% \begin{itemize}
% \item Video from \textit{The Big Bang Theory} Season 4 Episode 7 ``The
%   Apology Insufficiency'' %% 6:22
% \end{itemize}
% \item Definitions from WordNet: \url{http://wordnet.princeton.edu/}
%  \item Some slides use material from Alexander Coupe
%  \item Strict/sloppy identity joke adapted from \textit{Literal-Minded Blog:
% Linguistic commentary from a guy who takes things too literally}.
% \\ \footnotesize\url{http://literalminded.wordpress.com/2011/03/04/you-cant-go-from-strict-to-sloppy/}
  
%  \end{itemize}
 % \item Images from
%   \begin{itemize}
%   \item the Open Clip Art Library: \url{http://openclipart.org/}
%   \item Steven Bird, Ewan Klein, and Edward Loper (2009) 
%      \textit{Natural Language Processing with Python}, O'Reilly Media
%     \\ \url{www.nltk.org/book}
% \end{itemize}
% \item Problems  partially based on exercises from Saeed (2003)
% \end{itemize}



\myslide{Fry \& Laurie: \textit{Language}}


\begin{itemize}
\item Series 1 Episode 2
\\ \url{http://abitoffryandlaurie.co.uk/sketches/language_conversation}
\item Series 2 Episode 6
\\ \url{http://abitoffryandlaurie.co.uk/sketches/beauty_and_ideas}
\item Stephen Fry on \textit{Language}
\\ \url{http://www.stephenfry.com/2008/11/04/dont-mind-your-language%E2%80%A6/}
\end{itemize}



\myslide{Bibliography}
% Reading: Jurafsky and Martin (2008) Chapter 20
\renewcommand{\section}[2]{}
\renewcommand{\baselinestretch}{0.9}
\small
\bibliographystyle{aclnat}
\bibliography{abb,mtg,nlp,ling}


\end{document}


%%% Local Variables: 
%%% coding: utf-8
%%% mode: latex
%%% TeX-PDF-mode: t
%%% TeX-engine: xetex
%%% End: 
