\documentclass[headrule,footrule]{foils}

%%
%%%  Macros
%%%
%%% fonts-sil-charis for IPA in week 5

\newcommand{\logo}{HG2002 (2021)}
\usepackage[hidelinks]{hyperref}

\newcommand{\header}[3]{%
  \title{\vspace*{-2ex} \large 
    HG2002 Semantics and Pragmatics
% \thanks{Creative
%       Commons Attribution License: 
%       you are free to share and adapt as long as you give 
%       appropriate credit and add no additional restrictions: 
%       \protect\url{https://creativecommons.org/licenses/by/4.0/}.
%     }
    \\[2ex] \Large  \emp{#2} \\ \emp{#3}}
  \author{\blu{Francis Bond}   \\ 
    \normalsize  \textbf{Division of Linguistics and Multilingual Studies}\\
    \normalsize  \url{http://www3.ntu.edu.sg/home/fcbond/}\\
    \normalsize  \texttt{bond@ieee.org}}
  \MyLogo{\logo}
  % \MyLogo{奈良女子大学:欧米言語情報理論II}
  \date{#1
    \\  \url{https://bond-lab.github.io/Semantics-and-Pragmatics/}
\\[.5ex] \footnotesize Creative  Commons Attribution License:  you are free to share and adapt 
\\[-.25ex] \footnotesize   as long as you give    appropriate credit and add no
additional restrictions: 
\\ \small  \protect\url{https://creativecommons.org/licenses/by/4.0/}.
}
  % \renewcommand{\logo}{#2}
  % \special{! /pdfmark where
  %   {pop} {userdict /pdfmark /cleartomark load put} ifelse
  %   [ /Author (Francis Bond)
  %   /Title (#1: #2)
  %   /Subject (HG2002: Semantics and Pragmatics)
  %   /Keywords (Semantics, Pragmatics, Meaning)
  %   /DOCINFO pdfmark}
  %   }
  \hypersetup{%
    final       = true,
    colorlinks  = true,
    urlcolor    = blue,
    citecolor   = blue,
    linkcolor   = MidnightBlue,
    unicode     = true,
    pdfauthor   = {Francis Bond},
    pdfkeywords = {Semantics, Pragmatics, Meaning},
    pdftitle    = {#1: #2},
    pdfsubject  = {HG2002 Semantics and Pragmatics; License CC BY 4.0}
  }
}


\usepackage[a4paper,landscape]{geometry}
%\usepackage[dvips]{xcolor}
\usepackage[dvipsnames,x11names]{xcolor}
\usepackage{graphicx}
\newcommand{\blu}[1]{\textcolor{blue}{#1}}
\newcommand{\grn}[1]{\textcolor{green}{#1}}
\newcommand{\hide}[1]{\textcolor{white}{#1}}
\newcommand{\emp}[1]{\textcolor{red}{#1}}
\newcommand{\txx}[1]{\textbf{\textcolor{blue}{#1}}}
\newcommand{\lex}[1]{\textbf{\mtcitestyle{#1}}}


\usepackage{amsmath,latexsym}
\usepackage{pifont}
\renewcommand{\labelitemi}{\textcolor{violet}{\ding{227}}}
\renewcommand{\labelitemii}{\textcolor{purple}{\ding{226}}}

\newcommand{\subhead}[1]{\noindent\textbf{#1}\\[5mm]}

\newcommand{\Bad}{\emp{\raisebox{0.15ex}{\ensuremath{\mathbf{\otimes}}}}}
\newcommand{\bad}{*}

\newcommand{\com}[1]{\hfill \textnormal{(\emp{#1})}}%
\newcommand{\cxm}[1]{\hfill \textnormal{(\txx{#1})}}%
\newcommand{\cmm}[1]{\hfill \textnormal{(#1)}}%

\usepackage{relsize,xspace}
\newcommand{\into}{\ensuremath{\rightarrow}\xspace}
\newcommand{\ent}{\ensuremath{\Rightarrow}\xspace}
\newcommand{\nent}{\ensuremath{\not\Rightarrow}\xspace}
\newcommand{\tot}{\ensuremath{\leftrightarrow}\xspace}
\usepackage{url}
\newcommand{\lurl}[1]{\MyLogo{\url{#1}}}

\usepackage{mygb4e}
\let\eachwordone=\itshape
\newcommand{\lx}[1]{\textbf{\textit{#1}}}
\newcommand{\ix}{\ex\it}

\newcommand{\cen}[2]{\multicolumn{#1}{c}{#2}}
%\usepackage{times}
%\usepackage{nttfoilhead}
\newcommand{\myslide}[1]{\foilhead[-25mm]{\raisebox{12mm}[0mm]{\emp{#1}}}\MyLogo{\logo}}
\newcommand{\myslider}[1]{\rotatefoilhead[-25mm]{\raisebox{12mm}[0mm]{\emp{#1}}}}
%\newcommand{\myslider}[1]{\rotatefoilhead{\raisebox{-8mm}{\emp{#1}}}}

\newcommand{\section}[1]{\myslide{}{\begin{center}\Huge \emp{#1}\end{center}}}



\usepackage[lyons,j,e,k]{mtg2e}
\renewcommand{\mtcitestyle}[1]{\textcolor{teal}{\textsl{#1}}}
%\renewcommand{\mtcitestyle}[1]{\textsl{#1}}
\newcommand{\ja}[1]{\mtcitestyle{\makexeCJKactive #1\makexeCJKinactive}}
\newcommand{\chn}{\mtciteform}
\newcommand{\zsm}{\mtciteform}
%\newcommand{\cmn}[1]{make\cjkactive\mtciteform#1\makecjkinactive}
\newcommand{\iz}[1]{\textup{\texttt{\textcolor{blue}{\textbf{#1}}}}}
\newcommand{\con}[1]{\textsc{#1}}
\newcommand{\gm}{\textsc}
\newcommand{\cmp}[1]{{[\textsc{#1}]}}
\newcommand{\sr}[1]{\ensuremath{\langle}#1\ensuremath{\rangle}}
\usepackage[normalem]{ulem}
\newcommand{\ul}{\uline}
\newcommand{\ull}{\uuline}
\newcommand{\wl}{\uwave}
\newcommand{\vs}{\ensuremath{\Leftrightarrow}~}
%%% theta role
\newcommand{\tr}[1]{\textcolor{Chartreuse4}{\textsc{#1}}}
%%% theta grid
\newcommand{\grid}[1]{\ensuremath{\langle}\tr{#1}{\ensuremath{\rangle}}}

%%%
%%% Bibliography
%%%
\usepackage{natbib}
%\usepackage{url}
\usepackage{bibentry}
%\usepackage{CJKutf8}


\usepackage{fontenc}
\usepackage{polyglossia}
\setmainlanguage{english}
\setotherlanguages{tamil}
\setmainfont[Ligatures=TeX]{TeX Gyre Pagella}
\setsansfont[Ligatures=TeX]{TeX Gyre Heros}
\newfontfamily\ipafont{Charis SIL}
\newcommand\ipa[1]{\mtcitestyle{\ipafont #1}}


\usepackage{xeCJK}
\makexeCJKinactive
\newcommand{\zh}[1]{\mtcitestyle{\makexeCJKactive #1\makexeCJKinactive}}
%\newcommand{\ja}[1]{\makexeCJKactive #1\makexeCJKinactive}
\setCJKmainfont{Noto Sans CJK JP}
\setCJKsansfont{Noto Sans CJK SC}
\setCJKmonofont{Noto Sans CJK SC}

\newfontfamily\tamilfont[Script=Tamil]{Noto Sans Tamil}
\newfontfamily\tamilfontsf[Script=Tamil]{Noto Sans Tamil}
\newcommand{\tam}[1]{\texttamil{#1}}
%%% From Tim
\newcommand{\WMngram}[1][]{$n$-gram#1\xspace}
\newcommand{\infers}{$\rightarrow$\xspace}


\usepackage{rtrees,qtree}
\renewcommand{\lf}[1]{\br{#1}{}}
\usepackage{avm}
%\avmoptions{topleft,center}
\newcommand{\ft}[1]{\textsc{#1}}
\renewcommand{\val}[1]{\textit{#1}}
\newcommand{\typ}[1]{\textit{#1}}
\avmfont{\sc}
\avmvalfont{\sc}
\renewcommand{\avmtreefont}{\sc}
\avmsortfont{\it}


%%% From CSLI book
\newcommand{\mc}{\multicolumn}
\newcommand{\HD}{\textbf{H}\xspace}
\newcommand{\el}{\< \>}
\makeatother
\long\def\smalltree#1{\leavevmode{\def\\{\cr\noalign{\vskip12pt}}%
\def\mc##1##2{\multispan{##1}{\hfil##2\hfil}}%
\tabskip=1em%
\hbox{\vtop{\halign{&\hfil##\hfil\cr
#1\crcr}}}}}
\makeatletter

%\usepackage{tipa}
\usepackage{multicol}


\newcommand{\task}{\marginpar{\large ~~~\textbf{?}}}
\newcommand{\sh}[1]{\href{https://www.arthur-conan-doyle.com/index.php?title=#1}{#1}}

\usepackage{tikz}
\usepackage{tikz-qtree}
\usepackage{forest}


\usepackage{tikz}

\begin{document}
\header{Lecture 10}{Formal Semantics}{}\maketitle

%\include{schedule}

\myslide{Overview}
\begin{itemize}\addtolength{\itemsep}{-1ex}
\item Revision: Components
\item Quantifiers and Higher Order Logic
\item Modality
\item (Dynamic Approaches to Discourse)
\item Next Lecture: Chapter 11 --- \emp{Cognitive Semantics}
\end{itemize}




\section{Revision: \\ Componential Analysis}


\myslide{Break word meaning into its components}
\begin{itemize} \addtolength{\itemsep}{-2ex}
\item components allow a compact description
\item interact with morphology/syntax
\item form part of our cognitive architecture
\item For example:
  \\[2ex] \begin{tabular}{lllll}
    \lex{woman} & \cmp{female} & \cmp{adult} & \cmp{human} & \\
    \lex{spinster} & \cmp{female} & \cmp{adult} & \cmp{human} & \cmp{unmarried} \\
    \lex{bachelor} & \cmp{male} & \cmp{adult} & \cmp{human} & \cmp{unmarried} \\
    \lex{wife} & \cmp{female} & \cmp{adult} & \cmp{human} & \cmp{married} \\
  \end{tabular}
\item We can make things more economical (fewer components):
  \\[2ex] \begin{tabular}{lllll}
    \lex{woman} & \cmp{+female} & \cmp{+adult} & \cmp{+human} & \\
    \lex{spinster} & \cmp{+female} & \cmp{+adult} & \cmp{+human} & \cmp{--married} \\
    \lex{bachelor} & \cmp{--female} & \cmp{+adult} & \cmp{+human} & \cmp{--married} \\
    \lex{wife} & \cmp{+female} & \cmp{+adult} & \cmp{+human} & \cmp{+married} \\
  \end{tabular}
\end{itemize}

\myslide{Defining Relations using Components}

\begin{itemize}%\addtolength{\itemsep}{-0.2em}
\item \txx{hyponymy}: 
    P is a hyponym of Q if all the components of Q are also in P.
\\ \lex{spinster} $\subset$ \lex{woman}; \lex{wife} $\subset$ \lex{woman}
\item \txx{incompatibility}:
    P is incompatible with Q if they share some
    components but differ in one or more \txx{contrasting} components
\\  \lex{spinster} $\not\approx$ \lex{wife}
\item{Redundancy Rules}
\\[2ex]  \begin{tabular}{llll}
     \cmp{+human} & \into & \cmp{+animate}  \\
     % \cmp{+adult} & \into & \cmp{+animate}  \\
      \cmp{+animate} & \into & \cmp{+concrete}  \\
     \cmp{+married} & \into & \cmp{+adult}  \\
     \cmp{+married} & \into & \cmp{+human}   & \ldots
  \end{tabular}
\item Predicates with argument structure
\\ \lex{parent (of y)}(\ul{x},y)  \into  \cmp{+parent}(\ul{x},y)
\end{itemize} 


\myslide{Katz’s Semantic Theory}

\begin{itemize}
\item Semantic rules must be recursive to deal with infinite meaning
\item Semantic rules interact with syntactic rule to build up meaning \txx{compositionally}
  \begin{itemize}
  \item A \txx{dictionary} pairs lexical items with semantic representations
 \begin{itemize}
 \item (\txx{semantic markers}) are the links that bind lexical items
   together in lexical relations
 \item {[\txx{distinguishers}]} serve to identify this particular lexical item
   \\ this information is not relevant to syntax
 \end{itemize}
\item\txx{projection rules} show how meaning is built up
  \begin{itemize}
  \item Information is passed up the tree and collected at the top.
  \item \txx{Selectional restrictions} help to reduce ambiguity and
    limit the possible readings
  \end{itemize}
\end{itemize}

  
\myslide{Verb Classification}
\MyLogo{Levin (1993)}

\begin{itemize}
\item We can investigate the meaning of a verb by looking at its
  grammatical behavior
  \begin{exe}
    \ex Consider the following transitive verbs
    \begin{xlist}
      \ex \eng{Margaret \ul{cut} the bread}
      \ex \eng{Janet \ul{broke} the vase}
      \ex \eng{Terry \ul{touched} the cat}
      \ex \eng{Carla \ul{hit} the door}
    \end{xlist}
  \end{exe}
\item These do not all allow the same argument structure alternations

\end{itemize}
\myslide{Diathesis Alternations}

\begin{itemize}
\item \txx{Causative/inchoative} alternation:
  \begin{quote}
    \eng{Kim \ul{broke} the window} $\leftrightarrow$ \eng{The window \ul{broke}}
    \\ also \eng{the window \ul{is broken}} (state)
  \end{quote}
\item \txx{Middle construction} alternation:
  \begin{quote}
    \eng{Kim \ul{cut} the bread} $\leftrightarrow$ \eng{The bread \ul{cut} easily}
  \end{quote}
\item \txx{Conative} alternation:
  \begin{quote}
    \eng{Kim \ul{hit} the door} $\leftrightarrow$ \eng{Kim \ul{hit} at the door}
  \end{quote}
\item \txx{Body-part possessor ascension} alternation:
  \begin{quote}
    \eng{Kim \ul{cut} Sandy's arm} $\leftrightarrow$ \eng{Kim} \ul{cut Sandy on the arm}
  \end{quote}
\end{itemize}




\myslide{Diathesis Alternations and Verb Classes}

\MyLogo{Levin (1993)}

\begin{itemize}
\item A verb's (in)compatibility with different alternations is a strong
  predictor of its lexical semantics:
  \begin{quote}\smaller[1]
    \begin{tabular}{lcccc}
      & \lex{break} & \lex{cut} & \lex{hit} & \lex{touch} \\
      Causative & YES & NO & NO & NO \\
      Middle & YES & YES & NO & NO \\
      Conative & NO & YES & YES & NO \\
      Body-part & NO & YES & YES & YES \\
    \end{tabular}\larger[1]
 \end{quote}
\item 
    
    \lex{break} = \{\eng{break, chip, crack, crash, crush, ...}\}\\
    \lex{cut} = \{\eng{chip, clip, cut, hack, hew, saw, ...}\}\\
    \lex{hit} = \{\eng{bang, bash, batter, beat, bump, ...}\}\\
    \lex{touch} = \{\eng{caress, graze, kiss, lick, nudge, ...}\}
  \item 
  \begin{tabular}[t]{ll}
  \lex{break} & \textsc{cause, change}\\
  \lex{cut}   & \textsc{cause, change, contact, motion}\\ 
  \lex{hit}   & \textsc{contact, motion}\\
  \lex{touch} & \textsc{contact}
  \end{tabular}
 

\end{itemize}

\myslide{Cognitive Semantics}
\MyLogo{Talmy (1975, 1983, 1985, 2000)}
\begin{itemize}
\item Major semantic components of Motion:
\begin{itemize}
\item \txx{Figure}: object moving or located with respect to the \txx{ground} 
\item \txx{Ground}: reference object
\item \txx{Motion}: the presence of movement of location in the event
\item \txx{Path}: the course followed or site occupied by the Figure
\item \txx{Manner}: the type of motion
\end{itemize}
\begin{exe}
  \ex \gll Kim swam {away from} {the crocodile} \\
  Figure Manner Path Ground \\
  \ex \gll {The banana} hung from {the tree} \\
  Figure Manner Path Ground \\
\end{exe}
\item These are lexicalized differently in different languages.
\\ 
\begin{small}
\begin{tabular}{ll}
  Language (Family) & Verb Conflation Pattern \\ \hline
  Romance, Semitic, Polynesian, \ldots & Path + fact-of-Motion \\
  Indo-European ($-$ Romance), Chinese & Manner/Cause + fact-of-Motion \\
  Navajo, Atsuwegei, \ldots & Figure + fact-of-Motion 
\end{tabular}
\end{small}
\end{itemize}

\myslide{Jackendoff’s Lexical Conceptual Structure} 

\begin{itemize}
\item An attempt to explain how we think
\item \txx{Mentalist Postulate}
  \begin{quote}
    Meaning in natural language is an information structure that is
    mentally encoded by human beings
  \end{quote}
\item Universal Semantic Categories
  \begin{itemize}
  \item \txx{Event}
  \item \txx{State}
  \item \txx{Material Thing/Object}
  \item \txx{Path}
  \item \txx{Place}
  \item \txx{Property}
  \end{itemize}
\end{itemize}

\myslide{Motion as a tree}

\begin{multicols}{2}
  \begin{exe}
    \ex \eng{Bobby went into the house}
    \ex ``Bobby traverses a path that terminates at the interior of the house''
    \ex
    \begin{tree}
      \br{Event}{\lf{GO}
        \br{Thing}{ \lf{BOBBY}}
        \br{Path}{\lf{TO}
          \br{Place}{\lf{IN}\br{Thing}{\lf{HOUSE}}}}}
    \end{tree}
    %\newpage
    \ex \eng{The car is in the garage}
    \ex ``The car is in the state located in the interior of the garage''
    \ex
    \begin{tree}
      \br{State}{\lf{BE-LOC}
        \br{Thing}{ \lf{CAR}}
        \br{Place}{\lf{IN}\br{Thing}{\lf{GARAGE}}}}
    \end{tree}
\end{exe}
\end{multicols}


\myslide{Things: Boundedness and Internal Structure}
\begin{itemize}
\item Two components:
\\[2ex]  \begin{tabular}{llll}
    Boundedness & Internal Struct. & Type & Example\\ \hline
    $+$b & $-$i & \txx{individuals} & \eng{a dog}/\eng{two dogs}\\
    $+$b & $+$i & \txx{groups}      & \eng{a committee}\\
    $-$b & $-$i & \txx{substance}s  & \eng{water}\\
    $-$b & $+$i & \txx{aggregates}  & \eng{buses, cattle}
  \end{tabular}

\item This can be extended to verb aspect (the verb event is also [$\pm$b, $\pm$i]).
  \\ \lex{sleep} [$-$b], \lex{cough} [$+$b], \lex{eat}  [$\pm$b]
 \begin{exe}
   \ex Bill ate two hot dogs in two hours.
   \ex *Bill ate hot dogs in two hours.
   \ex $^\#$Bill ate two hot dogs for two hours.
   \ex Bill ate hot dogs for two hours.
\end{exe}
\end{itemize}

\myslide{Conversion: Boundedness and Internal Structure}
\MyLogo{See Bond (2005) for an extension to Japanese and computational implementation.}
\begin{itemize}
\item Including
 \\[2ex] \begin{tabular}{lll}
  \txx{plural} & {[+b, --i] \into\ [--b, +i]} &     
    \eng{brick}  \into\ \eng{bricks} \\
  \txx{composed of} &{[--b, +i] \into\ [+b, --i]} &
     \eng{bricks}  \into\ \eng{house of bricks} \\
  \txx{containing} &   {[--b, --i] \into\ [+b, --i]} &
     \eng{coffee}  \into\ \eng{a cup of coffee/a coffee}
  \end{tabular}
\item Excluding
  \\[2ex] \begin{tabular}{lll}
    \txx{element}  & {[--b,+i] \into\ [+b, --i]} &     
    \eng{grain of rice} \\
    \txx{partitive} & {[--b, $\pm$i] \into\ [+b, --i]} &     
    \eng{top of the mountain}, \eng{one of the dogs} \\
    \txx{universal grinder} &  {[+b, --i] \into\ [--b, --i]} &     
    \eng{There's \ul{dog} all over the road}
  \end{tabular}
\end{itemize}



\myslide{Pustejovsky’s Generative Lexicon}
\MyLogo{}

\begin{itemize} 
\item Each lexical entry can have:\\
  \textsc{argument structure}\\
  \textsc{event structure}\\
  \textsc{lexical  inheritance structure}\\
  \textsc{qualia structure}:\\
  \begin{tabular}{ll}
    \textsc{constitutive} & constituent parts \\
    \textsc{formal} & relation to other things \\
    \textsc{telic} & purpose \\
    \textsc{agentive}  & how it is made
  \end{tabular}
\item Interpretation is \txx{generated} by combing word meanings
\item Events have \textbf{complex} structure
 \\ \begin{tabular}{ccc}
    \txx{State}  & \txx{Process} & \txx{Transition} \\
 \begin{tree}
    \br{S}{\lf{e}}
  \end{tree}
  &    \begin{tree}
    \br{P}{\tlf{e$_1$ \ldots e$_n$}}
  \end{tree}
  &     \begin{tree}
    \br{T}{\lf{E$_1$} \lf{$\neg$ E$_2$}}
  \end{tree}
  \\   \lex{understand, love, be tall}
  &    \lex{sing, walk, swim}
  &     \lex{open, close, build}
\end{tabular}
\end{itemize}
\end{itemize}

% \myslide{Different Alternations}
%   \begin{exe}
%     \ex \eng{The door closed}
%    \begin{tree}
%       \br{T}{\br{P}{\br{[$\neg$ closed(door)]}{}}
%         \br{S}{\br{[closed(door)]}{}}} 
%     \end{tree}
%     \ex \eng{Jamie closed the door}
%     \begin{tree}
%       \br{T}{\br{P}{\br{[act(j, door) $\wedge$ $\neg$ closed(door)]}{}}
%         \br{S}{\br{[closed(door)]}{}}} 
%     \end{tree}
%     \ex \eng{The door is closed}
%    \begin{tree}
%       \br{S}{\br{e}{\br{[closed(door)]}{}}}
%     \end{tree}
%   \end{exe}

\myslide{Modifier Ambiguity}
\MyLogo{}
  \begin{exe}
    \ex \eng{Jamie closed the door rudely}
    \ex \eng{Jamie closed the door in a rude way [with his foot]}
    \\\begin{tree}
      \br{T}{\br{P [rude(P)]}{\br{[act(j, door) $\wedge$ $\neg$ closed(door)]}{}}
        \br{S}{\br{[closed(door)]}{}}} 
    \end{tree}
    \ex \eng{It was rude of Jamie to close the door}
    \\\begin{tree}
      \br{T [rude(T)]}{\br{P}{\br{[act(j, door) $\wedge$ $\neg$ closed(door)]}{}}
        \br{S}{\br{[closed(door)]}{}}} 
    \end{tree}
  \end{exe}

\myslide{Qualia Structure}
\MyLogo{}

\begin{exe}
  \ex \eng{fast typist}
  \begin{xlist}
    \ex\label{ta} a typist who is fast [at running]
    \ex\label{tb} a typist who types fast
  \end{xlist}
%  \ex \eng{Joan baked the potato}
\end{exe}
\begin{itemize}
\item typist 
 \begin{avm} \[
    \textsc{argstr} &
    \[
      \textsc{arg1} & \iz{x:typist}\\
    \]\\
    \textsc{qualia} &
    \[
      \textsc{formal} & \[ \iz{x}  [ $\subset$ \iz{person} ]\] \\
      \textsc{telic} & \[ \iz{type(e,x)}  \]
      \] 
 \]
\end{avm}
\item (\ref{ta}) \eng{fast} modifies $x$
\item (\ref{tb}) \eng{fast} modifies $e$
\end{itemize}



\myslide{Summary}

\begin{itemize}
\item Meaning can be broken up into units smaller than words:  \txx{components} 
  \begin{itemize}
  \item These can be combined to make larger meanings
  \item At least some of them influence syntax
  \item They may be psychologically real
  \end{itemize}
\item Problems with Components of Meaning
  \begin{itemize}
  \item Primitives are no different from necessary and sufficient conditions
    \\ it is impossible to agree on the definitions
    \\ but they allow us to state generalizations better
  \item Psycho-linguistic evidence is weak
  \item It is just \txx{markerese}
  \item There is no \txx{grounding}
  \end{itemize}
\end{itemize}



\section{Word Meaning: \\ Meaning Postulates}

\myslide{Defining Relations using Logic}
\MyLogo{These are relations on predicates, not words}
\begin{itemize}
\item \txx{hyponymy}
  \begin{itemize}
  \item $\forall$x(DOG(x) \into\ ANIMAL(x))
  \end{itemize}
\item \txx{synonym}
  \begin{itemize}
  \item $\forall$x((EGGPLANT(x) \into\ BRINJAL(x)) $\wedge$ 
    (BRINJAL(x) \into EGGPLANT(x)))
  \item $\forall$x(EGGPLANT(x) $\equiv$\ BRINJAL(x))
  \end{itemize}
  \newpage
  \item \txx{antonym}
  \begin{itemize}
  \item $\forall$x(DEAD(x) \into\ $\neg$ALIVE(x));
  \item[+] $\forall$x(ALIVE(x) \into\ $\neg$DEAD(x))
  \end{itemize}
\item \txx{converse}
  \begin{itemize}
  \item $\forall$x$\forall$y(PARENT(x,y) \into\ CHILD(y,x));
  \\  $\forall$x$\forall$y(PARENT(x,y) \into\ $\neg$ CHILD(x,y))
  \item $\forall$x$\forall$y(CHILD(y,x) \into\ PARENT(x,y))
  \\ $\forall$x$\forall$y(CHILD(y,x) \into\ $\neg$  PARENT(y,x))
  \end{itemize}
\end{itemize}




\myslide{Semantic Relations as Sets ($p \subset q$ and $p \sim q$)}


\begin{tabular}{cc}
$p \subset q$ \iz{hypernym}  & $p \sim q$ \iz{synonym} \\[2ex] 
\scalebox{2}{
\begin{tikzpicture}
\filldraw[fill=white] (-2,-2) rectangle (3,2);
\scope % A \vee B
%\clip (0,0) circle (1);
\fill[pink] (0.25,0) circle (0.5);
%\fill[white] (0,0) circle (1);
\endscope
\draw (0.25,0) circle (0.5) node [text=black] {$p$}
      (1,0) circle (1.5) node [text=black,right] {$q$};
% outline
\end{tikzpicture}}
&
\scalebox{2}{
\begin{tikzpicture}
\filldraw[fill=white] (-2,-2) rectangle (3,2);
\scope % A \vee B
%\clip (0,0) circle (1);
%\fill[pink] (1,0) circle (1);
\fill[pink] (0.5,0) circle (1);
\endscope
\draw (0.5,0) circle (1) node [text=black,left] {$p$}
      (0.5,0) circle (1) node [text=black,right] {$q$};
% outline
\end{tikzpicture}}

\end{tabular}



\myslide{Logical Connectives as Sets ($p$ and $\neg p$)}


\begin{tabular}{cc}
$p$   & $\neg p$ ``not'' \\[2ex] 
\scalebox{2}{
\begin{tikzpicture}
\filldraw[fill=white] (-2,-2) rectangle (3,2);
\scope % A \cap B
%\clip (0,0) circle (1);
\fill[pink] (0,0) circle (1);
\endscope
% outline
\draw (0,0) circle (1) node [text=black,left] {$p$}
      (1,0) circle (1) node [text=black,right] {$q$};
\end{tikzpicture}} &
\scalebox{2}{
\begin{tikzpicture}
\filldraw[fill=pink] (-2,-2) rectangle (3,2);
\scope % A \vee B
%\clip (0,0) circle (1);
%\fill[pink] (1,0) circle (1);
\fill[white] (0,0) circle (1);
\endscope
\draw (0,0) circle (1) node [text=black,left] {$p$}
      (1,0) circle (1) node [text=black,right] {$q$};
% outline
\end{tikzpicture}}

\end{tabular}

\myslide{Logical Connectives as Sets ($p \wedge q$ and   $p \vee q$)}


\begin{tabular}{cc}
$p \wedge q$ ``and''  & $p \vee q$ ``or'' \\[2ex]  
\scalebox{2}{
\begin{tikzpicture}
\filldraw[fill=white] (-2,-2) rectangle (3,2);
\scope % A \cap B
\clip (0,0) circle (1);
\fill[pink] (1,0) circle (1);
\endscope
% outline
\draw (0,0) circle (1) node [text=black,left] {$p$}
      (1,0) circle (1) node [text=black,right] {$q$};
\end{tikzpicture}} &
\scalebox{2}{
\begin{tikzpicture}
\filldraw[fill=white] (-2,-2) rectangle (3,2);
\scope % A \vee B
%\clip (0,0) circle (1);
\fill[pink] (1,0) circle (1);
\fill[pink] (0,0) circle (1);
\endscope
% outline
\draw (0,0) circle (1) node [text=black,left] {$p$}
      (1,0) circle (1) node [text=black,right] {$q$};
\end{tikzpicture}}
\end{tabular}


\myslide{Logical Connectives as Sets ($p \oplus q$ and   $p \rightarrow q$)}


\begin{tabular}{cc}
$p \oplus q$ ``exclusive or''  & $p \rightarrow q$ ``if'' \\[2ex] 
\scalebox{2}{
\begin{tikzpicture}
\filldraw[fill=white] (-2,-2) rectangle (3,2);
\fill[pink] (1,0) circle (1);
\fill[pink] (0,0) circle (1);
\scope % A \cap B
\clip (0,0) circle (1);
\fill[white] (1,0) circle (1);
\endscope
% outline
\draw (0,0) circle (1) node [text=black,left] {$p$}
      (1,0) circle (1) node [text=black,right] {$q$};
\end{tikzpicture}} &
\scalebox{2}{
\begin{tikzpicture}
\filldraw[fill=pink] (-2,-2) rectangle (3,2);
\scope % A \vee B
%\clip (0,0) circle (1);
\fill[white] (0,0) circle (1);
\fill[pink] (1,0) circle (1);
\endscope
% outline
\draw (0,0) circle (1) node [text=black,left] {$p$}
      (1,0) circle (1) node [text=black,right] {$q$};
\end{tikzpicture}}
\end{tabular}



\section{Natural Language Quantifiers \\ and Higher Order Logic}

\myslide{Restricted Quantifiers}

\begin{itemize}
\item \eng{Most students read a book}
  \begin{itemize}
  \item Most(x)(S(x) $\wedge$ R(x))
    \\ \eng{most things are students and most things read books}
  \item Most(x)(S(x) \into\  R(x))
    \\ \eng{most things are such that, if they are students, they read
      books}
    \\ but also true for all things that are not students!
  \end{itemize}
\item We need to restrict the quantification
  \begin{itemize}
  \item (Most x: S(x)) R(x)
  \end{itemize}
\item Sometimes we need to decompose
  \begin{itemize}
  \item \eng{everybody} ($\forall$x: P(x))
  \item \eng{something} ($\exists$x: T(x))
  \end{itemize}
\end{itemize}

\myslide{Higher Order Logic}
\begin{itemize}
\item First-order logic over individuals
\item Second-order logic also quantifies over sets
\item Third-order logic also quantifies over sets of sets
\item Fourth-order logic also quantifies over sets of sets of sets
\item[\ldots{}]
\end{itemize}



\myslide{Higher Order Logic}
 \begin{itemize}
\item Recall  \eng{Ian sings}
  \begin{itemize}
  \item {[S(i)]$^{M_1}$ = 1 iff [i]$^{M_1} \in$ [S]$^{M_1}$} \\ The
    sentence is true if and only if the extension of \eng{Ian} is part of
    the set defined by \eng{sings} in the model $M_1$
  \item Remodel, with sing a property of Ian: i(S) \\ {[i(S)]$^{M_1}$
      = 1 iff [S]$^{M_1} \in$ [i]$^{M_1}$} \\ The sentence is true if
    and only if the denotation of the verb phrase \textit{sings} is
    part of the extension of \eng{Ian}  in the model $M_1$
  \end{itemize}
\item \eng{Ian} is a set of sets of properties: \txx{second-order logic}
\end{itemize}

\myslide{Generalized Quantifiers} 
\MyLogo{Q: Try to define \iz{many}}
\begin{itemize}
\item Q(A,B): \eng{Q A are B}
\item \iz{most(A,B)} = 1 iff $|$ A $\cap$ B $| > |$ A $-$ B $|$ 
\item \iz{all(A,B)} = 1 iff  A $\subseteq$ B  
\item \iz{some(A,B)} = 1 iff  A $\cap$ B $\ne \emptyset$ 
\item \iz{no(A,B)} = 1 iff A $\cap$ B  $= \emptyset$ 
\item \iz{fewer than x(A,B,X)} =1 iff $|$ A $\cap$ B $| <  |$ X $|$ 
\end{itemize}

\myslide{Generalized Quantifiers:  \iz{all}, \iz{most}} 
\begin{tabular}{cc}
all $p$ are $q$ & most $p$ are $q$ \\[2ex] 
\scalebox{2}{
\begin{tikzpicture}
\filldraw[fill=white] (-2,-2) rectangle (3,2);
\scope % A \vee B
%\clip (0,0) circle (1);
\fill[pink] (0.25,0) circle (0.5);
%\fill[white] (0,0) circle (1);
\endscope
\draw (0.25,0) circle (0.5) node [text=black] {$p$}
      (1,0) circle (1.5) node [text=black,right] {$q$};
% outline
\end{tikzpicture}}
&
\scalebox{2}{
\begin{tikzpicture}
\filldraw[fill=white] (-2,-2) rectangle (3,2);
\scope % A \vee B
\clip (1,0) circle (1.5);
\fill[pink] (0.2,0) circle (1);
%\fill[white] (1,0) circle (1);
\endscope
\draw (0.2,0) circle (1) node [text=black] {$p$}
      (1,0) circle (1.5) node [text=black,right] {$q$};
% outline
\end{tikzpicture}}

\end{tabular}


\myslide{Generalized Quantifiers:  \iz{some}, \iz{no}} 
\begin{tabular}{cc}
some $p$ are $q$ & no $p$ are $q$ \\[2ex] 
\scalebox{2}{
\begin{tikzpicture}
\filldraw[fill=white] (-2,-2) rectangle (3,2);
\scope % A \vee B
\clip (1,0) circle (1.5);
\fill[pink] (0.2,0) circle (1);
%\fill[white] (0,0) circle (1);
\endscope
\draw (-0.8,0) circle (1) 
      (-0.6,0) circle (1) 
      (-0.4,0) circle (1) 
      (-0.2,0) circle (1) 
      (0.0,0) circle (1) node [text=black] {$p$}
      (0.2,0) circle (1) 
%      (0.25,0) circle (0.5) 
      (1,0) circle (1.5) node [text=black,right] {$q$};
% outline
\end{tikzpicture}}
&
\scalebox{2}{
\begin{tikzpicture}
\filldraw[fill=white] (-2,-2) rectangle (3,2);
\scope % A \vee B
%\clip (1,0) circle (1.5);
\fill[pink] (-0.75,0) circle (1);
%\fill[white] (1,0) circle (1);
\endscope
\draw (-.75,0) circle (1) node [text=black] {$p$}
      (1.75,0) circle (1) node [text=black] {$q$};
% outline
\end{tikzpicture}}

\end{tabular}



\myslide{Strong/Weak Quantifiers}
\MyLogo{Q: Come up with some more strong and weak quantifiers}
\begin{exe}
  \ex only \txx{weak} quantifiers can occur in existential \eng{there} sentences
  \begin{xlist}
      \ex \eng{There is \ul{a} fox in the henhouse}
      \ex \eng{There are \ul{two} foxes in the henhouse}
      \ex \eng{*There is \ull{every} fox in the henhouse}
      \ex \eng{*There are \ull{both} foxes in the henhouse}
  \end{xlist}
\end{exe}
\begin{itemize}\addtolength{\itemsep}{-1ex}
\item \txx{symmetrical} (cardinal) quantifiers are \ul{\txx{weak}}
  \\ det(A,B) = det(B,A)
  \begin{exe}
    \ex \eng{3 lecturers are Australian}   = \eng{3 Australians are lecturers}
  \end{exe}
\item \txx{asymmetrical} (proportional) quantifiers are \ull{\txx{strong}}
  \\ det(A,B) $\ne$ det(B,A)
  \begin{exe}
    \ex \eng{most lecturers are Australian}   $\ne$ \eng{most Australians are lecturers}
  \end{exe}

\end{itemize}

\myslide{Negative Polarity Items (NPI)}
\MyLogo{Q: Come up with some NPIs and environments}
\begin{itemize}
\item Some words in English mainly appear in negative environments
  \begin{exe}
    \ex
    \begin{xlist}
      \ex \eng{Kim does\ull{n't} \ul{ever} eat dessert}
      \ex *\eng{Kim does \ul{ever} eat dessert}
    \end{xlist}
    \ex
    \begin{xlist}
      \ex \eng{Kim has\ull{n't} eaten dessert \ul{yet}}
      \ex *\eng{Kim has eaten dessert \ul{yet}}
    \end{xlist}
    \ex
    \begin{xlist}
      \ex \eng{Few people  have eaten dessert \ul{yet}}
      \ex *\eng{Many people have eaten dessert \ul{yet}}
    \end{xlist}
   \ex
    \begin{xlist}
      \ex \eng{Rarely does Kim \ul{ever} eat dessert}
      \ex *\eng{Often does Kim \ul{ever} eat dessert}
    \end{xlist}
  \end{exe}
\item Not just negation, but also some quantifiers
\end{itemize}

\myslide{Monotonicity}

\begin{itemize}
\item Some quantifiers control entailment between sets and subsets
  \begin{itemize}
  \item \txx{Upward entailment} goes from a subset to a set
  \item \txx{Downward entailment} goes from a set to a subset
  \end{itemize}
\end{itemize}

\begin{exe}
  \ex
  \begin{xlist}
    \ex \eng{Kim does\ull{n't} eat dessert} \ent  \eng{Kim does\ull{n't} eat hot dessert}
    \ex \eng{Kim does\ull{n't} eat hot dessert} \nent  \eng{Kim does\ull{n't} eat dessert}
    \trans \textbf{Downward entailment}
    \end{xlist}
    \ex
  \begin{xlist}
    \ex \eng{Kim eats some desserts} \nent  \eng{Kim eats hot desserts}
    \ex \eng{Kim eats some hot desserts} \ent  \eng{Kim  eats some desserts}
    \trans \textbf{Upward entailment}
    \end{xlist}
  \end{exe}
  \begin{itemize}
  \item \emp{Negative Polarity Items} are licensed by \emp{downward entailing expressions}
  \end{itemize}
  %
  % \end{document}
  % \section{Intensionality}

\myslide{Left and Right Monotonicity}

\begin{itemize}
\item The monotonicity may depend on the position
  \begin{exe}
    \ex 
 \begin{xlist}
    \ex \eng{Every student studies semantics} \nent  
    \eng{Every student studies formal semantics}
    \ex \eng{Every student studies formal semantics} \ent  
    \eng{Every student studies semantics}
    \trans \textbf{Upward entailment (right argument)}
    \end{xlist}
    \ex 
 \begin{xlist}
    \ex \eng{Every student studies semantics} \ent  
    \eng{Every linguistics student studies  semantics}
    \ex \eng{Every linguistic student studies  semantics} \nent  
    \eng{Every student studies semantics}
    \trans \textbf{Downward entailment (left argument)}
    \end{xlist}
\newpage
\MyLogo{Q: Make up your own example (up or down, left or right)}
  \ex 
 \begin{xlist}
    \ex \eng{Every student who has ever studied semantics loves it}
    \ex *\eng{Every student who has studied semantics ever loves it}
    \ex \eng{Few students who have ever studied semantics dislike it}
    \ex \eng{Few students who have studied semantics ever dislike it}
    \end{xlist}
  \end{exe}
\end{itemize}
\begin{itemize}
\item Formal models of quantification can be used to make predictions
  about seemingly unrelated phenomena
\end{itemize}

\myslide{In other languages too!}

\MyLogo{Thanks to Joanna Sio}
%\CJKfamily{gbsn}
\begin{exe}
\ex \glll {} \zh{我} \zh{没有} \zh{任何} \zh{朋友}  \\
        {} wǒ   méi-yǒu   rènhé   péngyǒu    \\
        {} I   \textsc{neg}-have any     friend \\
\trans ``I don’t have any friends.''
\ex \glll * \zh{我} \zh{有} \zh{任何} \zh{朋友} \\
        {} wǒ  yǒu   rènhé   péngyǒu     \\
        {} I  have any     friend \\
\trans *``I have any friends.''
\end{exe}
%\CJKfamily{min}
\section{Modality}

\myslide{Modality as a scale of Implicatures}
\MyLogo{Allan (1986): incomplete}

\begin{exe}
  \ex \eng{I know that $p$}
  \ex \eng{I am absolutely certain that  $p$}
  \ex \eng{I am almost certain that  $p$}
  \ex \eng{I believe that $p$}
  \ex \eng{I am pretty certain that $p$}
\\ $\ldots$
\item \eng{Possibly $p$}
\\ $\ldots$
  \ex \eng{It is very unlikely that $p$}
  \ex \eng{It is almost impossible that  $p$}
  \ex \eng{It is impossible that $p$}
  \ex \eng{It is not the case that $p$}
  \ex \eng{I am absolutely certain that  not-$p$}
\end{exe}

\myslide{Modal Logics}

\begin{itemize}
\item Add two modal operators for epistemic modality
  \begin{itemize}
  \item $\diamond\phi =$ \eng{it is possible that $\phi$}
  \item $\Box\phi =$ \eng{it is necessary that $\phi$}
  \end{itemize}
\item Define them in terms of \txx{possible worlds}
  \begin{itemize}
  \item $\diamond\phi$: true in at least one world
  \item $\Box\phi$: true in all worlds
  \end{itemize}
\item $M = \{W, U, F\}$: the model now has three parts
  \begin{itemize}
  \item[$W$] set of possible worlds 
  \item[$U$] domain of individuals (universe)
  \item[$F$] denotation assignment function
  \end{itemize}
\end{itemize}

\myslide{Deontic Modality}

\begin{itemize}
\item Add two modal operators for deontic modality
  \begin{itemize}
  \item P$\phi =$ \eng{it is permitted that $\phi$}
  \item O$\phi =$ \eng{it is obligatorily $\phi$}
  \end{itemize}
\item Define them in terms of \txx{possible worlds}
  \begin{itemize}
  \item P$\phi$: true in at least one legal or morally ideal world
  \item O$\phi$: true in all legal or morally ideal worlds
  \end{itemize}
\end{itemize}

\section{Dynamic Approaches \\ to Discourse}

\myslide{Anaphora}

\begin{exe}
  \ex 
  \begin{xlist}
    \ex \eng{R2D2$_i$ mistrusts itself$_i$}
    \ex M(r,r)
  \end{xlist}
 \ex 
  \begin{xlist}
    \ex \eng{Every robot mistrusts itself}
    \ex ($\forall$x: R(x)) M(x,x)
  \end{xlist}
 \ex 
  \begin{xlist}
    \ex \eng{Luke bought a robot and it doesn't work}
    \ex ($\exists$x: R(x)) B(l,x) $\wedge$ $\neg$W(x)
  \end{xlist}
 \ex 
  \begin{xlist}
    \ex \eng{Every robot went to Naboo.  ?It met Jar Jar.}
    \ex ($\forall$x: R(x)) W(x,n); M(x,j)\hfill \emp{unbound}
  \end{xlist}
 \ex 
  \begin{xlist}
    \ex \eng{A robot went to Naboo.  It met Jar Jar.}
    \ex ($\exists$x: R(x)) W(x,n); M(x,j)\hfill \emp{???}
  \end{xlist}
  \trans indefinite nominals exist beyond the sentence: \txx{discourse referents}
\ex 
  \begin{xlist}
    \ex \eng{Luke didn't buy a robot.  ?It met Jar Jar.}
  \end{xlist}
  \trans indefinite nominals scope can still be limited
\end{exe}

\myslide{Donkey Sentences}

\begin{exe}
  \ex 
  \begin{xlist}
    \ex \eng{If R2D2$_i$ owns a ship it is rich}
    \ex ($\exists$x (S(x) $\wedge$ O(r,x))) \into\ R(x)
  \end{xlist}
  \ex 
  \begin{xlist}
    \ex \eng{If a robot owns a ship it races it}
    \ex *($\exists$x$\exists$y (R(x) $\wedge$ S(y) $\wedge$ O(x,y))) \into\ R(x,y)
    \ex $\forall$x$\forall$y ((R(x) $\wedge$ S(y) $\wedge$ O(x,y)) \into\ R(x,y)
  \end{xlist}
  \trans $\exists$ needs to become $\forall$
  \ex \eng{Every farmer who owns a donkey beats it}
\end{exe}

\myslide{Discourse Representation Theory}
\begin{itemize}
\item Build up Discourse Representation Structures
\end{itemize}

\begin{exe}
   \ex 
  \begin{xlist}
    \ex \eng{Alex met a robot$_i$}
    \ex \eng{It$_i$ smiled}
  \end{xlist}
  \ex \fbox{\begin{tabular}{l}
      \cen{1}{x y} \\[2ex]
    Alex(x) \\
    robot(y) \\
    met (x,y) 
  \end{tabular}} 
\hspace{5em} \fbox{\begin{tabular}{l}
      \cen{1}{x y u} \\[2ex]
    Alex(x) \\
    robot(y) \\
    met (x,y) \\
    u = y \\
    smiled(u)
  \end{tabular}}
\end{exe}

\myslide{Negative Contexts}

\begin{exe}
   \ex 
  \begin{xlist}
    \ex \eng{Luke does not own a robot}
  \end{xlist}
  \ex 
  \fbox{\begin{tabular}{l}
      \cen{1}{x} \\
      Luke(x) \\
      $\neg$\ \  \fbox{\begin{tabular}{l}
          \cen{1}{y} \\[2ex]
          robot(y) \\
          own (x,y) 
        \end{tabular}} 
    \end{tabular}}
  \end{exe}
  \begin{itemize}
  \item The contained DRS is \txx{subordinate}
    \begin{itemize}
    \item indefinite NPs in negated subordinate structures are inaccessible
    \item names (constants) are always accessible
    \end{itemize}
  \end{itemize}

\myslide{Conditionals}

\begin{exe}
   \ex 
  \begin{xlist}
    \ex \eng{If Jo owns a robot then they are rich}
  \end{xlist}
  \ex 
  \fbox{\begin{tabular}{lll}
      & \cen{1}{x} \\
 \fbox{\begin{tabular}{l}
          \cen{1}{y} \\[2ex]
          Jo(x) \\
          robot(y) \\
          own (x,y) 
        \end{tabular}} 
& \into &
 \fbox{\begin{tabular}{l}
          \cen{1}{u} \\[2ex]
          u=x \\
          rich(u) 
        \end{tabular}} 
    \end{tabular}}
  \end{exe}
  \begin{itemize}
  \item The contained DRS is \txx{subordinate}
    \begin{itemize}
    \item indefinite NPs in the antecedent are accessible in the consequent
    \end{itemize}
  \end{itemize}
\myslide{More Conditionals}

\begin{exe}
   \ex 
  \begin{xlist}
    \ex \eng{If a Jedi owns a robot then they are rich}
  \end{xlist}
  \ex 
  \fbox{\begin{tabular}{lll}
 \fbox{\begin{tabular}{l}
          \cen{1}{x y} \\[2ex]
          jedi(x) \\
          robot(y) \\
          own (x,y) 
        \end{tabular}} 
& \into &
 \fbox{\begin{tabular}{l}
          \cen{1}{u} \\[2ex]
          u=x \\
          rich(u) 
        \end{tabular}} 
    \end{tabular}}
  \end{exe}
  \begin{itemize}
  \item The contained DRS is \txx{subordinate}
    \begin{itemize}
    \item indefinite NPs in the antecedent are accessible in the consequent
    \end{itemize}
  \end{itemize}

\myslide{More Conditionals}

\begin{exe}
   \ex 
  \begin{xlist}
    \ex \eng{If a Jedi owns a robot then they race it}
  \end{xlist}
  \ex 
  \fbox{\begin{tabular}{lll}
 \fbox{\begin{tabular}{l}
          \cen{1}{x y} \\[2ex]
          jedi(x) \\
          robot(y) \\
          own (x,y) 
        \end{tabular}} 
& \into &
 \fbox{\begin{tabular}{l}
          \cen{1}{u v} \\[2ex]
          u=x \\
          v=y \\
          race(u,v) 
        \end{tabular}} 
    \end{tabular}}
  \end{exe}
  \begin{itemize}
  \item The contained DRS is \txx{subordinate}
    \begin{itemize}
    \item indefinite NPs in the antecedent are accessible in the consequent
    \end{itemize}
  \end{itemize}

\myslide{More Conditionals}

\begin{exe}
   \ex 
  \begin{xlist}
    \ex \eng{Every Jedi who owns a robot races it}
  \end{xlist}
  \ex 
  \fbox{\begin{tabular}{lll}
 \fbox{\begin{tabular}{l}
          \cen{1}{x y} \\[2ex]
          jedi(x) \\
          robot(y) \\
          own (x,y) 
        \end{tabular}} 
& \into &
 \fbox{\begin{tabular}{l}
          \cen{1}{u} \\[2ex]
          u=y \\
          race(x,u) 
        \end{tabular}} 
    \end{tabular}}
  \end{exe}
  \begin{itemize}
  \item The contained DRS is \txx{subordinate}
    \begin{itemize}
    \item Universal Quantifiers copy the variable across the conditional
    \end{itemize}
  \end{itemize}

\myslide{Discourse Representation Theory}
\MyLogo{More in \textbf{HG4049}: Linguistics Meaning and Its Interfaces}
\begin{itemize}
\item Explains how reference occurs across clauses and sentences
  \begin{itemize}
  \item Distinguishes between names and indefinite NPS
  \item Distinguishes between positive assertions, negative sentences,
    conditional sentences, universally quantified sentences
  \item Is useful for modeling the incremental update of knowledge in a conversation
  \end{itemize}
\end{itemize}

%\myslide{Summary}

\myslide{Acknowledgments and References}
\begin{itemize}
\item Video \textit{Regency Disco} from that Mitchel and Webb Look Episode 3.3, which was first broadcast on Thursday 25th June 2009.
\end{itemize}
 
% \MyLogo{}
% \begin{itemize}
% \item Video from \textit{The Big Bang Theory} Season 4 Episode 7 ``The
%   Apology Insufficiency'' %% 6:22
% \end{itemize}
% \item Definitions from WordNet: \url{http://wordnet.princeton.edu/}
%  \item Some slides use material from Alexander Coupe
%  \item Strict/sloppy identity joke adapted from \textit{Literal-Minded Blog:
% Linguistic commentary from a guy who takes xgrepthings too literally}.
% \\ \footnotesize\url{http://literalminded.wordpress.com/2011/03/04/you-cant-go-from-strict-to-sloppy/}
  
%  \end{itemize}
 % \item Images from
%   \begin{itemize}
%   \item the Open Clip Art Library: \url{http://openclipart.org/}
%   \item Steven Bird, Ewan Klein, and Edward Loper (2009) 
%      \textit{Natural Language Processing with Python}, O'Reilly Media
%     \\ \url{www.nltk.org/book}
% \end{itemize}
% \item Problems  partially based on exercises from Saeed (2003)
% \end{itemize}

%\myslide{Bibliography}
% Reading: Jurafsky and Martin (2008) Chapter 20
% \small
% \bibliographystyle{aclnat}
% \bibliography{abb,mtg,nlp,ling}


\end{document}

%%% Local Variables: 
%%% coding: utf-8
%%% mode: latex
%%% TeX-PDF-mode: t
%%% TeX-engine: xetex
%%% End: 
%%% Local Variables: 
%%% coding: utf-8
%%% mode: latex
%%% TeX-PDF-mode: t
%%% TeX-engine: xetex
%%% End: 
