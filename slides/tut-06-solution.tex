\documentclass[a4paper]{article}

\title{\vspace*{-30mm}HG2002:  Solutions to Tutorial 6\\  Participants}
\author{Francis Bond \url{<bond@ieee.org>}}
\date{}%2011-08-15}

\usepackage{polyglossia}
\setmainlanguage{english}
\setmainfont[Ligatures=TeX]{TeX Gyre Pagella}
\setsansfont[Ligatures=TeX]{TeX Gyre Heros}
\usepackage{xeCJK}
\setCJKmainfont{Noto Sans CJK JP}
\newcommand{\ans}[1]{\hfill{#1}}
%\newcommand{\ans}[1]{}

\usepackage{multicol}
%\Restriction{}
%\rightfooter{}
%\leftheader{}
%\rightheader{}
\usepackage{mygb4e}
\newcommand{\lex}[1]{\textbf{\textit{#1}}}
\newcommand{\lx}[1]{\textbf{\textit{#1}}}
\newcommand{\eng}[1]{\textit{#1}}
\newcommand{\jpn}[1]{\textit{#1}}
\newcommand{\ix}{\ex\it}
\newcommand{\con}[1]{\textsc{#1}}
\usepackage{url}
\usepackage[normalem]{ulem}
\newcommand{\ul}[1]{\uline{#1}}
\newcommand{\txx}[1]{\textbf{#1}}

\begin{document}
\maketitle

\begin{enumerate}


\item For each of the theta-roles below, construct a sentence in any
  language that you speak, where an argument bearing that role occurs
  as subject. Use simple active sentences for this exercise.

  EXPERIENCER, PATIENT, THEME, INSTRUMENT, RECIPIENT

  \begin{itemize}
  \item EXPERIENCER   \eng{\ul{Kim} feels happy}
  \item PATIENT   \eng{\ul{Kim} fell over}
  \item THEME   \eng{\ul{Kim} is tired}
  \item INSTRUMENT   \eng{\ul{The chisel} smashed the bowl}
  \item RECIPIENT   \eng{\ul{Kim} got the letter}
  \end{itemize}


\item For each of the theta-roles below, construct a sentence in any
  language that you speak, where an argument bearing that role occurs
  as object. Use simple active sentences for this exercise.

 \begin{itemize}
  \item EXPERIENCER   \eng{Semantics fascinates \ul{Kim}}
  \item PATIENT   \eng{Sandy healed \ul{Kim}}
  \item THEME   \eng{I gave \ul{the letter} to Kim}
  \item INSTRUMENT   \eng{Kim used \ul{a knife} to open the letter}
  \item RECIPIENT   \eng{Sandy gave \ul{Kim}}
  \end{itemize}


\item Design lexical theta-grids for the underlined verbs in the
  following sentences. For example, a theta-grid for \lex{buy} in
  \\ \textit{Bobby \ul{bought} the car for Sandy} would be: 
  \\ \lex{buy} $\langle$\ul{AGENT}, THEME, BENEFICIARY$\rangle$

  \begin{exe}
    \ix Freddie \ul{drove} to the party.
    \ix Kim \ul{swatted} the fly with a newspaper.
    \ix The baboon was \ul{asleep} on the roof of my car.
    \ix The dog was \ul{killed} by Fran.
    \ix Alex \ul{gave} the doorman a tip.
  \end{exe}

  \begin{itemize}
  \item \lex{drive}  $\langle$\ul{AGENT}$\rangle$
  \item \lex{swat}  $\langle$\ul{AGENT}, PATIENT, INSTRUMENT$\rangle$
  \item \lex{asleep} $\langle$\ul{THEME}$\rangle$
  \item \lex{killed} $\langle$\ul{PATIENT}, AGENT$\rangle$
  \item \lex{give} $\langle$\ul{AGENT}, RECIPIENT, THEME $\rangle$
  \end{itemize}
  
  \newpage
  
\item Explain Dowty's argument selection principles in your own
  words. Explain the following sets of sentences using these principles?

\begin{exe}
    \ex
    \begin{xlist}
      \ix He fears Aids
      \ix Aids frightens him.
    \end{xlist}
    \trans A: Similar properties so either can be subject/object
    \ex
    \begin{xlist}
      \ix Patricia resembles Maura.
      \ix Maura resembles Patricia.
    \end{xlist}
    \trans A: Similar properties so either can be subject/object
    \ex
    \begin{xlist}
      \ix Joan bought a sportscar from Jerry.
      \ix Jerry sold a sportscar to Joan. 
    \end{xlist}
    \trans A: The same event, but different perspectives (who is doing
    the event), and so different characterisations
  \end{exe}


\item If you speak a classifier language, predict which classifier you
  would use for the following, and try to explain why:

  \begin{exe}
    \ex a live dog
    \trans 匹 \jpn{hiki} used for small animals
    \ex a dead dog
    \trans 体 \jpn{tai} used for corpses
    \ex a robot dog (Aibo)
    \trans  匹 \jpn{hiki} used by the owner, who thinks of it as an animal
    \trans 式 \jpn{shiki} used by the manufacturer, who thinks of it
    as a set
    \trans 台 \jpn{dai} used for machines
    \ex a stuffed toy dog
    \trans  匹 \jpn{hiki} used by the owner, who thinks of it as an animal
    \trans つ/個 \jpn{tsu/ko} used for general things
    \ex a dog being barbecued on a spit
    \trans つ/個 \jpn{tsu/ko} used for general things
    \ex a ghost
    \trans 匹, 人, 体 are all used!
    \ex an ogre
    \trans 匹, 人 are both used
    \ex a letter
    \trans 通 \jpn{tsuu} for correspondence generally
    \trans 枚 \jpn{mai} for flat things
    \ex an email message
    \trans 通 \jpn{tsuu} for correspondence generally
     \trans つ/個 \jpn{tsu/ko} used for general things
     \ex a text message
      \trans つ/個 \jpn{tsu/ko} used for general things
      \ex a phone call
        \trans 本 \jpn{hon} for long thin things
       \trans つ/個 \jpn{tsu/ko} used for general things
  \end{exe}

\item For each of the thematic roles AGENT, PATIENT, THEME,
  EXPERIENCER, BENEFICIARY, INSTRUMENT/MANNER, LOCATION, GOAL, SOURCE,
  STIMULUS, try to find an example of its use in the text you are
  annotating for project one, when it is assigned to you. 
  Email the examples you found to your tutor, in the following format:
  \begin{flushleft}
    SID: sentence \\
    ROLE: verb-NP/PP (just enough to identify it)
  \end{flushleft}
   For example:
  \begin{flushleft}
    11903: One day , the Lord Buddha was strolling around the edge of the lotus lake in heaven .\\
    AGENT: stroll-Buddha (or stroll-Lord Buddha)\\
    LOCATION: stroll-around the edge\\
  \end{flushleft}
%11904  御 釈迦 様 は 極楽 の 蓮 池 の ふち を 、 独り で ぶらぶら 御 歩き に なっ て いらっしゃい まし た 。 

\end{enumerate}




\vfill
\paragraph{Acknowledgments} These questions are partially
based on exercises from Saeed (2003).
\end{document}

%%% Local Variables: 
%%% coding: utf-8
%%% mode: latex
%%% TeX-PDF-mode: t
%%% TeX-engine: xetex
%%% End: 
