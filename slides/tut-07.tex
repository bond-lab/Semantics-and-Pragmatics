\documentclass[a4paper]{article}

\title{\vspace*{-30mm}HG2002: Tutorial 7\\  Context and Inference}
\author{Francis Bond \url{<bond@ieee.org>}}
\date{}%2011-08-15}

\newcommand{\ans}[1]{\hfill{#1}}
%\newcommand{\ans}[1]{}

\usepackage{multicol}
%\Restriction{}
%\rightfooter{}
%\leftheader{}
%\rightheader{}
\usepackage{mygb4e}
\newcommand{\lex}[1]{\textbf{\textit{#1}}}
\newcommand{\lx}[1]{\textbf{\textit{#1}}}
\newcommand{\ix}{\ex\it}
\newcommand{\con}[1]{\textsc{#1}}
\usepackage{url}
\usepackage[normalem]{ulem}
\newcommand{\ul}[1]{\uline{#1}}
\newcommand{\txx}[1]{\textbf{#1}}

\begin{document}
\maketitle

\begin{enumerate}

\item What are the deictic words/phrases in the following sentences?
  Identify each one as \textbf{space/time/person/social} or 
  some combination of them.

  \begin{exe}
    \ex \textit{They tried to hurt me, but he came to the rescue.}
    \ex \textit{The shop is across the street.}
    \ex \textit{She was sitting over there.}
    \ex \textit{It is raining out now, but I hope when you read this it will be sunny.}
    \ex \textit{I was born in London and I have lived there all my life.}
    \ex \textit{Wilt thou not listen to me?}
  \end{exe}
\item Describe the features used in your own language's pronoun system, with examples.
  (e.g. for Modern English: person, number, gender)

\item Give two examples each of \textbf{metonymy} and \textbf{synecdoche}.
\item Explain what is going on here 
  \\ (from \textit{The Pink Panther Strikes Again (1976))}
  \begin{description}
    \item[Clouseau:] Does your dog bite?
    \item[Hotel Clerk:] No.
    \item[Clouseau:] [bowing down to pet the dog] Nice doggie.
      \\ {}[Dog barks and bites Clouseau in the hand]
    \item[Clouseau:] I thought you said your dog did not bite!
    \item[Hotel Clerk:] That is not my dog. 
    \end{description}
  
%\item Make your own joke using strict/sloppy identity readings.

\item Hedges are used when you know you will flout a maxim.  Which
  maxim is flouted in the following hedges (and why)?:

\begin{exe}
% Manner:
\ex \textit{This may be a bit confusing, but I remember being in a car.}

%Quality:
\ex \textit{I may be mistaken, but I thought I saw a wedding ring on her finger.}

%\ex \textit{ I'm not sure if this is right, but I heard it was a secret ceremony in Hawaii.
%\ex \textit{ He couldn't live without her, I guess.

% Quantity:
%\ex \textit{  So, to cut a long story short, we grabbed our stuff and ran.
\ex \textit{I won't bore you with all the details, but it was an exciting trip.}

% Relation:
\ex \textit{I don't know if this is important, but some of the files are missing.}

%Quality:
\ex \textit{As far as I know, they're married.}

% Relation:
\ex \textit{This may sound like a dumb question, but whose handwriting is this?}
%\ex \textit{  Not to change the subject, but is this related to the budget?

% Manner:
%\ex \textit{I'm not sure if this makes sense, but the car had no lights.
\ex \textit{I don't know if this is clear at all, but I think the other car was reversing.}
% Quantity:
\ex \textit{As you probably know, I am afraid of dogs.}
\end{exe}

\item In the text you are
  annotating for project one try to find  examples of different kinds of reference:\\[2ex]
  \begin{tabular}{lll}
    PRONOUN: Pronominal reference & \textit{I saw a dog. \ul{It} bit me.} \\
    BRIDGE:  Bridging reference & \textit{I saw a dog.  \ul{The owner} was cute.} \\
    DEFINITE: Reference with an NP & \textit{I saw a dog. \ul{The animal} bit me.} \\
  \end{tabular} \\[2ex]
  Email the examples you found to your tutor, 
  in the following format at least 24 hours before the tutorial.\\
   \begin{flushleft}
    SID: sentence \\
    ROLE: NP: target (just enough to identify it: \\
~~~~~~~~~~~~~~~~~~~~~~~~~it may come from a previous sentence)
  \end{flushleft}


\bigskip

   For example:
 
  \begin{flushleft}
    11934: If he did not take immediate action, surely the string would split in half at the middle, and he would fall.\\ 
    PRONOUN: he: Kandata \\
    DEFINITE: the string: the thinly shining spider's thread
  \end{flushleft}

\end{enumerate}
\vfill
\paragraph{Acknowledgments} These questions are partially
based on exercises from Saeed (2003).
\end{document}
