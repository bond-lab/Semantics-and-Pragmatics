\documentclass[a4paper]{article}

\title{HG2002: Solutions to Tutorial 8\\  Speech Acts}
\author{Francis Bond \url{<bond@ieee.org>}}
\date{}%2011-08-15}
\usepackage[margin=25mm]{geometry}
\newcommand{\ans}[1]{\hfill{#1}}
%\newcommand{\ans}[1]{}

\usepackage{multicol}
%\Restriction{}
%\rightfooter{}
%\leftheader{}
%\rightheader{}
\usepackage{mygb4e}
\newcommand{\lex}[1]{\textbf{\textit{#1}}}
\newcommand{\lx}[1]{\textbf{\textit{#1}}}
\newcommand{\eng}[1]{\textit{#1}}
\newcommand{\ix}{\ex\it}
\newcommand{\con}[1]{\textsc{#1}}
\usepackage{url}
\usepackage[normalem]{ulem}
\newcommand{\ul}[1]{\uline{#1}}
\newcommand{\txx}[1]{\textbf{#1}}
\newcommand{\com}[1]{\hfill #1}

\begin{document}
\maketitle

\begin{enumerate}

\item The ability to insert \textit{hereby} appropriately is a test for performative utterances. For each of the following sentences below, use this as test to decide which of the following sentences would count as a performative utterance when uttered. 
  \begin{exe}
  \ex \textit{I acknowledge you as my legal heir.}\com{Y}
  \ex \textit{I give notice that I will stand down as chairman.}\com{Y}
  \ex \textit{I'm warning you that it won't end here.}\com{N}
  \ex \textit{I think you're taking this press attention too seriously.}\com{N}
  \ex \textit{I deny all knowledge of this scandal.}\com{?}
  \ex \textit{I promised them there'd be no fuss.}\com{N}
\end{exe}
\item Identify some direct illocutionary acts for the following utterances.
 \begin{exe}
   \ex \textit{Do you know what time it is?}
   \trans Asking the time
   \ex \textit{Hello! }
   \trans Greeting someone OR gaining their attention
   \ex \textit{What a beautiful day!}
   \trans Greeting someone
   \ex \textit{Accept the award!}
   \trans Telling someone to accept an award OR telling them to stop
   protesting already
   \ex \textit{Stop!}
   \trans Telling someone to stop OR asking people to stop someone 
\end{exe}

\item Below are some examples of indirect speech acts. For each one try to identify both the direct and the indirect act.
 \begin{exe}
\ex {[Travel agent to customer]}
\\ \textit{Why not think about Spain for this summer?}
\trans D: Asking a reason
\trans I: Suggesting Spain as a vacation destination

\ex {[Mother to child coming home from school]}
\\ \textit{I bet you're hungry.}
\trans D: Offering a bet
\trans I: Offering to make food

\ex {[Bank manager to applicant for an overdraft]}
\\ \textit{We regret we are unable to accede to your request.}
\trans D: Informing of their state of mind: regretful
\trans I: Refusing their request

\ex {[Someone responding to a friend's staying late]}
\\ \textit{Why don't you leave?}
\trans D: Asking for a reason as to why they don't leave
\trans I: Asking them to leave

\ex {[Doorman at a club to aspiring entrant]}
\\ \textit{Don't make me laugh.}
\trans D: Requesting someone not to cause laughter
\trans I: Refusing someone entry
\end{exe}

\item What are the felicity conditions that Searle has identified for
  requesting? Form different indirect requests with the following
  strategies below.

  \begin{enumerate}
  \item By querying the preparatory condition of the direct request
    \trans  $H$ is able to perform  $A$
    \trans \eng{Can you make me a sandwich?}
    
  \item By stating the preparatory condition of the direct request
    \trans  $H$ is able to perform  $A$
    \trans \eng{You can make me a sandwich.}

  \item By querying the propositional condition of the direct request
    \trans  $S$ predicates a future act $A$ of $H$
    \trans \eng{Are you going to make me a sandwich?}
    
  \item By stating the sincerity condition of the direct request.
    \trans  $S$ wants $H$ to do $A$ 
    \trans \eng{I want you to make me a sandwich.}
  \end{enumerate}

\item Cross-cultural differences in the use of direct versus indirect
  speech acts can lead speakers of one language to stereotype speakers
  of another language as impolite. Discuss any experience you may have
  had of such misunderstandings. Also reflect on how requests and
  other speech acts might differ in their directness in the languages
  that you speak. Try to come up with specific examples of
  differences.

  \begin{itemize}
  \item When ordering food at the food court, I feel impolite if I
    don't say \textit{Please}.  But I normally wouldn't use an
    addressee term (like \textit{uncle} or \textit{auntie}).  I notice
    many Singaporean speakers are more direct in the request, but
    start with an addressee term
    \begin{itemize}
    \item Me: \eng{rice please}
    \item Overheard: \eng{Uncle, rice}
    \end{itemize}
     \item Similarly in a taxi
    \begin{itemize}
    \item Me: \eng{turn left here please}
    \item Singapore speaker: \eng{Uncle, turn left here}
    \item Hong Kong speaker: \eng{turn left here, Sir}
    \end{itemize}
  \end{itemize}
  It still sounds weird without a \eng{please} to me, but maybe I
  sound rude because I don't say \eng{uncle}?
\end{enumerate}
\vfill
\paragraph{Acknowledgments} These questions are partially
based on exercises from Saeed (2003).
\end{document}
