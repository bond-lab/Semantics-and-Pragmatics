\documentclass[a4paper]{article}

\title{\vspace*{-30mm}HG2002: Tutorial 6\\  Participants}
\author{Francis Bond \url{<bond@ieee.org>}}
\date{}%2011-08-15}

\newcommand{\ans}[1]{\hfill{#1}}
%\newcommand{\ans}[1]{}

\usepackage{multicol}
%\Restriction{}
%\rightfooter{}
%\leftheader{}
%\rightheader{}
\usepackage{mygb4e}
\newcommand{\lex}[1]{\textbf{\textit{#1}}}
\newcommand{\lx}[1]{\textbf{\textit{#1}}}
\newcommand{\ix}{\ex\it}
\newcommand{\con}[1]{\textsc{#1}}
\usepackage{url}
\usepackage[normalem]{ulem}
\newcommand{\ul}[1]{\uline{#1}}
\newcommand{\txx}[1]{\textbf{#1}}

\begin{document}
\maketitle

\begin{enumerate}


\item For each of the theta-roles below, construct a sentence in any
  language that you speak, where an argument bearing that role occurs
  as subject. Use simple active sentences for this exercise.

EXPERIENCER, PATIENT, THEME, INSTRUMENT, RECIPIENT

\item For each of the theta-roles below, construct a sentence in any
  language that you speak, where an argument bearing that role occurs
  as object. Use simple active sentences for this exercise.

EXPERIENCER, PATIENT, THEME, INSTRUMENT, RECIPIENT



\item Design lexical theta-grids for the underlined verbs in the
  following sentences. For example, a theta-grid for \lex{buy} in
  \\ \textit{Bobby \ul{bought} the car for Sandy} would be: 
  \\ \lex{buy} $\langle$\ul{AGENT}, THEME, BENEFICIARY$\rangle$

  \begin{exe}
    \ix Freddie \ul{drove} to the party.
    \ix Kim \ul{swatted} the fly with a newspaper.
    \ix The baboon was \ul{asleep} on the roof of my car.
    \ix The dog was \ul{killed} by Fran.
    \ix Alex \ul{gave} the doorman a tip.
  \end{exe}
  

\item Explain Dowty's argument selection principles in your own
  words. Explain the following sets of sentences using these principles?

\begin{exe}
    \ex
    \begin{xlist}
      \ix He fears Aids
      \ix Aids frightens him.
    \end{xlist}
    \ex
    \begin{xlist}
      \ix Patricia resembles Maura.
      \ix Maura resembles Patricia.
    \end{xlist}
    \ex
    \begin{xlist}
      \ix Joan bought a sportscar from Jerry.
      \ix Jerry sold a sportscar to Joan. 
    \end{xlist}
  \end{exe}

\newpage

\item If you speak a classifier language, predict which classifier you
  would use for the following, and try to explain why:

  \begin{exe}
    \ex a live dog
    \ex a dead dog
    \ex a robot dog (Aibo)
    \ex a stuffed toy dog
    \ex a dog being barbecued on a spit
    \ex a ghost
    \ex an ogre
  \ex a letter
  \ex an email message
  \ex a text message
  \ex a phone call
  \end{exe}

\item For each of the thematic roles AGENT, PATIENT, THEME,
  EXPERIENCER, BENEFICIARY, INSTRUMENT/MANNER, LOCATION, GOAL, SOURCE,
  STIMULUS, try to find an example of its use in the text you are
  annotating for project one.  
  Email the examples you found to your tutor, in the following format:
  \begin{flushleft}
    SID: sentence \\
    ROLE: verb-NP/PP (just enough to identify it)
  \end{flushleft}
   For example:
  \begin{flushleft}
    11903: One day , the Lord Buddha was strolling around the edge of the lotus lake in heaven .\\
    AGENT: stroll-Buddha (or stroll-Lord Buddha)\\
    LOCATION: stroll-around the edge\\
  \end{flushleft}
%11904  御 釈迦 様 は 極楽 の 蓮 池 の ふち を 、 独り で ぶらぶら 御 歩き に なっ て いらっしゃい まし た 。 

\end{enumerate}




\vfill
\paragraph{Acknowledgments} These questions are partially
based on exercises from Saeed (2003).
\end{document}
