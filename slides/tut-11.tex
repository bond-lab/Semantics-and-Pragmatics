\documentclass[a4paper]{article}

\title{HG2002: Tutorial 11\\  Cognitive Semantics}
\author{Francis Bond \url{<bond@ieee.org>}}
\date{}%2011-08-15}
\usepackage[margin=25mm]{geometry}
\newcommand{\ans}[1]{\hfill{#1}}
%\newcommand{\ans}[1]{}

\usepackage{multicol}
%\Restriction{}
%\rightfooter{}
%\leftheader{}
%\rightheader{}
\usepackage{mygb4e}
\newcommand{\lex}[1]{\textbf{\textit{#1}}}
\newcommand{\lx}[1]{\textbf{\textit{#1}}}
\newcommand{\ix}{\ex\it}
\newcommand{\con}[1]{\textsc{#1}}
\usepackage[e,j]{mtg2e}
%\newcommand{\eng}[1]{\textit{#1}}
\usepackage{url}
\newcommand{\txx}[1]{\textbf{#1}}
\newcommand{\cmp}[1]{{[\textsc{#1}]}}
\newcommand{\sr}[1]{\ensuremath{\langle}#1\ensuremath{\rangle}}
\usepackage[normalem]{ulem}
\newcommand{\ul}{\uline}
\newcommand{\ull}{\uuline}
\newcommand{\wl}{\uwave}
\newcommand{\vs}{\ensuremath{\Leftrightarrow}~}


\begin{document}
\maketitle



\begin{enumerate}
\item What are some of the metaphors used to describe food in
  commercials and food columns?  (e.g. \eng{That chocolate cake is
  absolutely sinful}.)
\begin{exe}
  \ex \textit{Unknowingly, I popped a slice of lotus root into my mouth and immediately cringe at the sharp taste.}
  \ex \textit{It makes my mouth water while writing this review.}
  \ex \textit{Thomas Keller is a culinary colossus, and The French Laundry a regular in the top end of The World’s 50 Best Restaurants since it’s inception.}
  \ex \textit{The French Laundry successfully marries French influence with Thomas Keller's American roots.}
\end{exe}
\item Lakoff and Johnson (1980) proposed the following image schemas for love:
  \begin{itemize}
  \item LOVE IS A JOURNEY
  \item LOVE IS A FORCE (electromagnetic, gravitational)
  \item LOVE IS WAR
  \end{itemize}
Organise the following metaphors into the above three schemas.
\begin{exe}
\ex \textit{They lost their momentum}
\ex \textit{There were sparks between us}
\ex \textit{Look how far we've come}
\ex \textit{We're at a crossroad}
\ex \textit{He overpowered her}
\ex \textit{I could feel the electricity between them}
\ex \textit{We'll just have to go our separate ways}
\ex \textit{We can't turn back now}
\ex \textit{His whole life revolves around her}
\ex \textit{She is besieged by suitors}
\ex \textit{They are uncontrollably attracted to each other}
\ex \textit{He is known for his conquests}
\end{exe}
Can you think of other metaphors that do not fit into the above three schemas?

\item For any two languages that you know, discuss similarities and
  differences in conventionalized metaphors of body parts
  (e.g. \eng{hand of a watch}, \jpn[well-known (lit: face is wide)]{kao-ga hiroi}).
\newpage
\item \textbf{The Bladder Slugs of Yik}
  You are Chief Linguist Yale aboard the starship Benjamin Lee Whorf
  orbiting the gaseous giant Arcturus IV. Your most pressing task is
  to compile an analysis of the temporal system of the Yik language of
  the freefalling Bladder Slugs. Given the following translations from
  Yik, what metaphor(s) seem to prevail in the Yik temporal system?
  Offer explanations (feel free to draw diagrams).
  \begin{exe}
    \ex \textit{I perceived a large glabbage upperday.}
    \ex \textit{The time for implosion is just below us. }
    \ex \textit{The pressure increases, the light is dimming, I'm plummeting old.}
    \ex \textit{How deep until we fall on dense times.}
    \ex \textit{Three days above I consumed a large splodj.}
    \ex \textit{In the rarified days of my youth, I set my life on a helical path.}
    \ex \textit{The foolish Yik lives like a falling space rock.}
    \ex \textit{At darkest bottom, we all meet at the centre.}
    \ex \textit{The aliens, who live for eternity high above the days of our youth, believe the universe is expanding. But according to the great physicist Albort Einslug, it is merely moving up into its own ``past''.}
    \ex \textit{All lives converge. At impact, we will share our common destiny.}
    \ex \textit{I hope our bladderlings will rise into the upper reaches of the brightest past.}
    \ex \textit{I believe the shadows of our downtime bladderings fell across us lowerday.}
%%% was upperday
  \end{exe}
From:  Alan Dench, Department of Anthropology University of Western
Australia, 1991, posted on the linguist list 2.411: \url{https://linguistlist.org/issues/2/2-411.html}
\end{enumerate}

\vfill
\paragraph{Acknowledgments} These questions are partially
based on exercises from Saeed (2003).
\end{document}
