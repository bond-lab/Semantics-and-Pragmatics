\documentclass[a4paper]{article}

\title{\vspace*{-30mm}HG2002: Tutorial 5\\   Situations}
\author{Francis Bond \url{<bond@ieee.org>}}
\date{}%2011-08-15}

\newcommand{\ans}[1]{\hfill{#1}}
%\newcommand{\ans}[1]{}

\usepackage{multicol}
%\Restriction{}
%\rightfooter{}
%\leftheader{}
%\rightheader{}
\usepackage{mygb4e}
\newcommand{\lex}[1]{\textbf{\textit{#1}}}
\newcommand{\lx}[1]{\textbf{\textit{#1}}}
\newcommand{\ix}{\ex\it}
\newcommand{\con}[1]{\textsc{#1}}
\usepackage{url}
\usepackage[normalem]{ulem}
\newcommand{\ul}[1]{\uline{#1}}
\newcommand{\ull}{\uuline}
\newcommand{\txx}[1]{\textbf{#1}}

\begin{document}
\maketitle

\begin{enumerate}


\item Are the following verbs \textbf{stative} or \textbf{dynamic}?
  What are the tests that you have used in order to decide if they are
  stative or otherwise?

\lex{comprise, own, imitate, possess, know, resemble, lack, seize, last, think, lose}


\item Some verbs may describe \textbf{telic} (bounded) or \textbf{atelic} (unbounded)
  processes, depending on the form of their complements.  Below is a
  list of verb phrases. For each one, decide if it is telic or atelic,
  then see if you can change this value by altering the verb’s
  complement.

\lex{ate oranges, swim, rig an election, ripen, walk to the station}

\item Modal verbs can be used to convey \textbf{epistemic} or
  \textbf{deontic} modality. In the following sentences, discuss what
  the modal verbs tell us about the speaker’s attitude.

  \begin{exe}
    \ix This could be our bus now.
    \ix They would be very happy to meet you.
    \ix You must be the bride's father.
    \ix The bus should be here soon.
    \ix It might rain this afternoon.
    \ix I will study hard.
  \end{exe}

\item These sentences be used to convey \textbf{epistemic} or
  \textbf{deontic} modality. Explain the difference between the two
  readings, then translate the sentences into a language of your
  choice, and see if the ambiguity remains.
  \begin{exe}
    \ix You must be very tactful.
    \ix You will not leave this room early.
    \ix We should be home before five.
    \ix Students may do their homework in groups. 
  \end{exe}

\item Although English does not mark \textbf{evidentiality} grammatically, it
  can be expressed in other ways.  Consider the following situation:
$S$ ``Kim bit Sandy''.  How could you express the following situations:
  \begin{exe}
    \ix You think $S$ is true, but have no evidence
    \ix You saw $S$ occur
    \ix You saw a bite mark on Sandy, matching Kim's dental work
    \ix Someone told you $S$
    \ix You are Sandy, and you experienced $S$ 
  \end{exe}
  Are any of these expressed grammatically in a language that you speak?
\item Some verbs allow the form of the verb in an embedded
  \textit{that}-clause to be subjunctive (shown as \ull{subjunctive form}).
  \begin{exe}
  \ex \textit{Kim \ul{proposes} that the meeting \ull{be} recorded.}
  \ex *\textit{Kim \ul{thinks} that the meeting \ull{be} recorded.}
  %\ix Kim proposes that the meeting \ul{should be} recorded.
  \end{exe}
  Which of the following verbs may take the subjunctive (show with
  examples): \\ \lex{require, urge, remember, command, report,
    suggested, insist, deny, promise}

\item For each of the situation types (State, Activity, Accomplishment, Punctual, Achievement)
  try to find an example of its use in the text you are
  annotating for project one.  
  Email the examples you found to your tutor, in the following format:
  \begin{flushleft}
    SID: sentence \\
    SITUATION: verb
  \end{flushleft}
   For example:
  \begin{flushleft}
    11903: One day , the Lord Buddha was strolling around the edge of the lotus lake in heaven .\\
    ACTIVITY: stroll
  \end{flushleft}
\end{enumerate}

\vfill
\paragraph{Acknowledgments} These questions are partially
based on exercises from Saeed (2003).
\end{document}
