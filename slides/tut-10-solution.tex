\documentclass[a4paper]{article}

\title{HG2002: Solution to Tutorial 10\\  Formal Semantics}
\author{Francis Bond \url{<bond@ieee.org>}}
\date{}%2011-08-15}
\usepackage[margin=25mm]{geometry}
\newcommand{\ans}[1]{\hfill{#1}}
%\newcommand{\ans}[1]{}
\usepackage{polyglossia}
\setmainlanguage{english}
\setmainfont[Ligatures=TeX]{TeX Gyre Pagella}
\setsansfont[Ligatures=TeX]{TeX Gyre Heros}
\usepackage{multicol}
%\Restriction{}
%\rightfooter{}
%\leftheader{}
%\rightheader{}
\usepackage{mygb4e}
\newcommand{\lex}[1]{\textbf{\textit{#1}}}
\newcommand{\lx}[1]{\textbf{\textit{#1}}}
\newcommand{\ix}{\ex\it}
\newcommand{\con}[1]{\textsc{#1}}
\newcommand{\eng}[1]{\textit{#1}}
\usepackage{url}
\newcommand{\txx}[1]{\textbf{#1}}
\newcommand{\cmp}[1]{{[\textsc{#1}]}}
\newcommand{\sr}[1]{\ensuremath{\langle}#1\ensuremath{\rangle}}
\usepackage[normalem]{ulem}
\newcommand{\ul}{\uline}
\newcommand{\ull}{\uuline}
\newcommand{\wl}{\uwave}
\newcommand{\vs}{\ensuremath{\Leftrightarrow}~}
\newcommand{\into}{\ensuremath{\rightarrow}}
% \newcommand{\to}{\ensuremath{\Rightarrow}}
%%% theta role
\newcommand{\tr}[1]{\textsc{#1}}
%%% theta grid
\newcommand{\grid}[1]{\ensuremath{\langle}\tr{#1}{\ensuremath{\rangle}}}
\newcommand{\com}[1]{\hfill (#1)}


\begin{document}
\maketitle

\begin{enumerate}
\item Are the following quantifiers (i) symmetrical or asymmetrical;
  (ii) upward or downward entailing in the left (iib) or right (iic)
  argument?
  \begin{exe}
    \ex \textit{most}
    \begin{itemize}
    \item[i] \textit{Most students are youths} $\not\models$
      \textit{Most youths are students} so asymmetrical
      \\ $^*$\textit{There are most students over there}
    \item[iib] \textit{Most students are youths}  $\not\models$
      \textit{Most people are youths} \com{Upward, left}
     \\  \textit{Most students are youths}  $\not\models$
     \textit{Most linguistic students are youths}\com{Downward, left}
       \item[iic]  \textit{Most students study formal semantics}  $\not\models$
     \textit{Most students study semantics} \com{Upward, right}
     \\ \textit{Most students study semantics}  $\not\models$
      \textit{Most students study formal semantics} \com{Downward, right}
    \end{itemize}
      Neither upward or downward entailing on the left or right

    \ex \textit{many (cardinal)} ``a large number''
      \begin{itemize}
    \item[i] \textit{Many students are youths} $\models$
      \textit{Many youths are students} so symmetrical
      \\ \textit{There are many students over there}
    \item[iib] \textit{Many students are youths}  $\models$
      \textit{Many people are youths}
     \\  \textit{Many students are youths}  $\not\models$
     \textit{Many linguistic students are youths}
       \item[iic] \textit{Many students study formal semantics}  $\models$
      \textit{Many students study semantics} 
     \\  \textit{Many students study semantics}  $\not\models$
      \textit{Many students study formal semantics}
    \end{itemize}
     Upward entailing on the left and right

    \ex \textit{few (cardinal)} ``a small number'' (in comparison with another number stated or implied) 
      \begin{itemize}
    \item[i] \textit{Few students are youths} $\models$
      \textit{Few youths are students} so asymmetrical
      \\ \textit{There are few students over there}
    \item[iib] \textit{Few students are youths}  $\not\models$
      \textit{Few people are youths}
     \\  \textit{Few students are youths}  $\not\models$
     \textit{Few linguistic students are youths}
       \item[iic] \textit{Few students study semantics}  $\models$
      \textit{Few students study formal semantics}
     \\  \textit{Few students study formal semantics}  $\not\models$
      \textit{Few students study semantics} 
    \end{itemize}
     Downward entailing on the right
    
    \ex \textit{every}
    \begin{itemize}
    \item[i] \textit{Every student is a youth} $\not\models$
      \textit{Every youth is student} so asymmetrical
      \\ $^*$\textit{There is every student over there}
    \item[iib] \textit{Every student is a youth}  $\not\models$
      \textit{Every person is a youth}
     \\  \textit{Every student is a youth}  $\models$
     \textit{Every linguistic student is a youth}
       \item[iic] \textit{Every student studies formal semantics}  $\models$
      \textit{Every student studies semantics} 
     \\  \textit{Every student studies semantics}  $\not\models$
      \textit{Every student studies formal semantics}
    \end{itemize}
      Downward entailing on the left; Upward on the right

    \ex \textit{{[at least]} two}
   \begin{itemize}
    \item[i] \textit{At least two students are youths} $\models$
      \textit{At least two youths are students} so symmetrical
      \\ \textit{There are at least two students over there}
    \item[iib] \textit{At least two students are youths}  $\models$
      \textit{At least two people are youths}
     \\  \textit{At least two students are youths}  $\not\models$
     \textit{At least two linguistic students are youths}
       \item[iic] \textit{At least two students study semantics}  $\not\models$
      \textit{At least two students study formal semantics}
     \\  \textit{At least two students study  formal semantics}  $\models$
      \textit{At least two students study semantics} 
    \end{itemize}
       Upward  entailing on the left and right
    
    \ex \textit{{[exactly]} two} (no more or less)
    \begin{itemize}
    \item[i] \textit{Exactly two students are youths} $\models$
      \textit{Exactly two youths are students} so symmetrical
      \\ \textit{There are exactly two students over there}
    \item[iib] \textit{Exactly two students are youths}  $\not\models$
      \textit{Exactly two people are youths}
     \\  \textit{Exactly two students are youths}  $\not\models$
     \textit{Exactly two linguistic students are youths}
       \item[iic] \textit{Exactly two students study formal semantics}  $\not\models$
      \textit{Exactly two students study semantics} 
     \\  \textit{Exactly two students study semantics}  $\not\models$
      \textit{Exactly two students study formal semantics}
    \end{itemize}
    Neither upward or downward entailing on the left or right
    
  \end{exe}
% \item Using the DRT rules described in Saeed (2003, \S~10.9), try to
%   identify which NPs in the following sentences are accessible for
%   coreference with pronouns in subsequent sentences.
%  \begin{exe}
%    \ex \textit{If Kim drinks a beer they are happy
%    \ex \textit{Sandy does not own a scanner
%    \ex \textit{Every student who answers a question enjoys it
%   \end{exe}
\item Using the formulae of meaning postulates, represent the
  semantic relations between the following word pairs:
  \\  Also give the Theta-grid for the predicates.
  \begin{exe}
    \ex \textit{couch/sofa}
    \begin{itemize}
    \item $\forall$x((COUCH(x) \into\ SOFA(x)) $\wedge$ 
      $\forall$x ((SOFA(x) \into COUCH(x))
    \end{itemize}
    \ex \textit{accepted/rejected}
    \begin{itemize}
    \item $\forall$x(ACCEPTED(x) \into\ $\neg$REJECTED(x));
    \item[+] $\forall$x(REJECTED(x) \into\ $\neg$ACCEPTED(x))
    \end{itemize}
        \lex{X be accepted} \grid{\ul{theme}}
  \\  \lex{X be rejected} \grid{\ul{theme}}
    \ex \textit{student/person}
    \begin{itemize}
    \item $\forall$x((STUDENT(x) \into\ PERSON(x))
    \end{itemize}

    \ex \textit{on/off (of a switch)}
    \begin{itemize}
    \item $\forall$x(ON(x) \into\ $\neg$ON(x));
    \item[+] $\forall$x(OFF(x) \into\ $\neg$OFF(x))
    \end{itemize}
     \lex{X be on} \grid{\ul{theme}}
  \\  \lex{X be off} \grid{\ul{theme}}
    \ex \textit{buy/sell}

    \begin{itemize}
  \item $\forall$x$\forall$y(BUY(x,z,y) \into\ SELL(y,z,x));
  \\  $\forall$x$\forall$y(BUY(x,z,y) \into\ $\neg$ SELL(x,z,y))
  \item $\forall$x$\forall$y(SELL(y,z,x) \into\ BUY(x,z,y))
  \\ $\forall$x$\forall$y(SELL(y,z,x) \into\ $\neg$  BUY(y,z,x))
  \end{itemize}
 \lex{X buy Z from Y} \grid{\ul{agent}, theme, source}
  \\  \lex{X sell Z to Y} \grid{\ul{agent}, theme, goal}
      \ex \textit{computer/laptop}
     \begin{itemize}
    \item $\forall$x((LAPTOP(x) \into\ COMPUTER(x))
    \end{itemize}

    \ex \textit{give/receive}
    \begin{itemize}
    \item $\forall$x$\forall$y(GIVE(x,z,y) \into\ RECEIVE(y,z,x));
  \\  $\forall$x$\forall$y(GIVE(x,z,y) \into\ $\neg$ RECEIVE(x,z,y))
  \item $\forall$x$\forall$y(RECEIVE(y,z,x) \into\ GIVE(x,z,y))
  \\ $\forall$x$\forall$y(RECEIVE(y,z,x) \into\ $\neg$  GIVE(y,z,x))
  \end{itemize}
  \lex{X give Z to Y} \grid{\ul{agent}, theme, goal}
  \\  \lex{X receive Z from Y} \grid{\ul{agent}, theme, source}

  \ex \textit{Monday/Tuesday/Wednesday/Thursday/Friday}
    \begin{itemize}
    \item $\forall$x(MONDAY(x) \into\
      ($\neg$TUESDAY(x) $\vee$ $\neg$WEDNESDAY(x)
       $\vee$ $\neg$THURSDAY(x) $\vee$ $\neg$FRIDAY(x));
    \item[+] $\forall$x(TUESDAY(x) \into\
      ($\neg$MONDAY(x) $\vee$ $\neg$WEDNESDAY(x)
       $\vee$ $\neg$THURSDAY(x) $\vee$ $\neg$FRIDAY(x));
       \item[+] \ldots
     \end{itemize}
   \end{exe}
 
\newpage
\item Using set notation, define \textup{few(A,B)} (cardinal) and
  \textup{few\_of(A,B)} (proportional).


  \begin{itemize}
  \item   few(A,B) = 1 iff |A ∩ B| $< n$
\\  where $n$ is a contextually defined number that denotes a small
number without relating it to the size of A or B.
\item  few\_of(A,B) = 1 iff |A ∩ B| $< |A|/n$
\\  n is a contextually defined number >1 that denotes the proportion in relation to A's size.
\end{itemize}
\end{enumerate}
\vfill
\paragraph{Acknowledgments} These questions are partially
based on exercises from Saeed (2003).
\end{document}


%%% Local Variables: 
%%% coding: utf-8
%%% mode: latex
%%% TeX-PDF-mode: t
%%% TeX-engine: xetex
%%% End: 
