\documentclass[a4paper]{article}

\title{HG2002: Tutorial 10\\  Formal Semantics}
\author{Francis Bond \url{<bond@ieee.org>}}
\date{}%2011-08-15}
\usepackage[margin=25mm]{geometry}
\newcommand{\ans}[1]{\hfill{#1}}
%\newcommand{\ans}[1]{}

\usepackage{multicol}
%\Restriction{}
%\rightfooter{}
%\leftheader{}
%\rightheader{}
\usepackage{mygb4e}
\newcommand{\lex}[1]{\textbf{\textit{#1}}}
\newcommand{\lx}[1]{\textbf{\textit{#1}}}
\newcommand{\ix}{\ex\it}
\newcommand{\con}[1]{\textsc{#1}}
\newcommand{\eng}[1]{\textit{#1}}
\usepackage{url}
\newcommand{\txx}[1]{\textbf{#1}}
\newcommand{\cmp}[1]{{[\textsc{#1}]}}
\newcommand{\sr}[1]{\ensuremath{\langle}#1\ensuremath{\rangle}}
\usepackage[normalem]{ulem}
\newcommand{\ul}{\uline}
\newcommand{\ull}{\uuline}
\newcommand{\wl}{\uwave}
\newcommand{\vs}{\ensuremath{\Leftrightarrow}~}


\begin{document}
\maketitle

\begin{enumerate}
% \item Assume  the following model
% \\  U = \{Ian, Bernard, Peter, Stephen, Tony, Martin\}
%    \\ F(i) = Ian; F(b) = Bernard; F(p) = Peter; F(s) = Stephen; F(t) = Tony;  F(m) = Martin
%  \\ \begin{tabular}{lllll}
%       F(J) & = & in Joy Division & = & \{Ian, Bernard, Peter, Stephen\} \\
%       F(S) & = & sings & = & \{Ian, Peter\} \\      
%       F(G) & = & plays guitar & = & \{Bernard, Ian\} \\
%       F(B) & = & plays bass & = & \{Peter\} \\
%       F(D) & = & plays drums & = & \{Stephen\} \\
%       F(M) & = & is a manager & = & \{Tony\} \\
%       F(P) & = & is a producer & = & \{Martin\} \\
%       F(F) & = & fires at & = & \{$<$Martin, Tony$>$\} \\
%       F(O) & = & (over) produces & = & \{$<$Martin, Ian$>$, $<$Martin, Bernard$>$, \\
%       &&&&  $<$Martin, Peter$>$, $<$Martin, Stephan$>$\} \\
%     \end{tabular}
% \\ Paraphrase the following sentences in English and calculate their truth values
% \begin{exe}
%   \ex O(m,b)
%   \ex O(m,t)
%   \ex ($\exists$x) S(x) $\wedge$ G(x)
%   \ex ($\exists$x) B(x) $\wedge$ G(x)
%   \ex ($\forall$x: $\neg$J(x)) M(x) $\vee$ P(x)
%   \ex ($\forall$x: M(x)) F(m,x)
%   \ex ($\neg\forall$x) D(x)
%   \ex ($\forall$x: J(x)) G(x)  $\vee$ B(x)  $\vee$ D(x)
% \end{exe}
% \newpage
\item Are the following quantifiers (i) symmetrical or asymmetrical;
  (ii) upward or downward entailing in the left (iib) or right (iic)
  argument?
  \begin{exe}
    \ex \textit{most}
    \ex \textit{many (cardinal)}
    \ex \textit{few (cardinal)}
    \ex \textit{every }
    \ex \textit{{[at least]} two}
    \ex \textit{{[exactly]} two}
  \end{exe}
% \item Using the DRT rules described in Saeed (2003, \S~10.9), try to
%   identify which NPs in the following sentences are accessible for
%   coreference with pronouns in subsequent sentences.
%  \begin{exe}
%    \ex \textit{If Kim drinks a beer they are happy
%    \ex \textit{Sandy does not own a scanner
%    \ex \textit{Every student who answers a question enjoys it
%   \end{exe}
\item Using the formulae of meaning postulates, represent the
  semantic relations between the following word pairs:
  \begin{exe}
    \ex \textit{couch/sofa}
    \ex \textit{accepted/rejected}
    \ex \textit{student/person}
    \ex \textit{on/off (of a switch)}
    \ex \textit{buy/sell}
    \ex \textit{computer/laptop}
    \ex \textit{give/receive}
    \ex \textit{Monday/Tuesday/Wednesday/Thursday/Friday}
  \end{exe}
  Also give the Theta-grid for the predicates.
\item Using set notation, define \textup{few(A,B)} (cardinal) and
  \textup{few\_of(A,B)} (proportional).

% few(A,B) = 1 iff |A ∩ B| < n

% where n is an arbitrarily defined number that denotes a small number without relating it to the size of A or B.
% few(A,B) = 1 iff |A ∩ B| < (A/n)

% n is a arbitrarily defined number >1 that denotes the proportion in relation to A's size.
  
\end{enumerate}
\vfill
\paragraph{Acknowledgments} These questions are partially
based on exercises from Saeed (2003).
\end{document}
