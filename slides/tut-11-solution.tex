\documentclass[a4paper]{article}

\title{HG2002: Solution to Tutorial 11\\  Cognitive Semantics}
\author{Francis Bond \url{<bond@ieee.org>}}
\date{}%2011-08-15}
\usepackage[margin=25mm]{geometry}
\newcommand{\ans}[1]{\hfill{#1}}
%\newcommand{\ans}[1]{}
\usepackage{polyglossia}
\setmainlanguage{english}
\setmainfont[Ligatures=TeX]{TeX Gyre Pagella}
\setsansfont[Ligatures=TeX]{TeX Gyre Heros}
\usepackage{xeCJK}
\setCJKmainfont{Noto Sans CJK JP}
%\usepackage{multicol}
%\Restriction{}
%\rightfooter{}
%\leftheader{}
%\rightheader{}
\usepackage{mygb4e}
\newcommand{\lex}[1]{\textbf{\textit{#1}}}
\newcommand{\lx}[1]{\textbf{\textit{#1}}}
\newcommand{\ix}{\ex\it}
\newcommand{\con}[1]{\textsc{#1}}
\usepackage[e,j]{mtg2e}
%\newcommand{\eng}[1]{\textit{#1}}
\usepackage{url}
\newcommand{\txx}[1]{\textbf{#1}}
\newcommand{\cmp}[1]{{[\textsc{#1}]}}
\newcommand{\sr}[1]{\ensuremath{\langle}#1\ensuremath{\rangle}}
\usepackage[normalem]{ulem}
\newcommand{\ul}{\uline}
\newcommand{\ull}{\uuline}
\newcommand{\wl}{\uwave}
\newcommand{\vs}{\ensuremath{\Leftrightarrow}~}


\begin{document}
\maketitle



\begin{enumerate}
\item What are some of the metaphors used to describe food in
  commercials and food columns?  (e.g. \eng{That chocolate cake is
    absolutely sinful}.)  Please try to find your own examples, these
  are some we found.  What is the source and target?
  
\begin{exe}
  \ex \eng{Fourrier make some of the finest, lightest-boned and most
    ethereal Gevreys of all.} (referring to a light wine)
  \\ DRINK is HUMAN
  \ex \eng{Big-boned petite sirah packs a punch!} (referring to a
  powerful, deep and rich wine)
  \\ DRINK is HUMAN
  \ex \eng{\ldots{},  this is quite a voluptuous wine, \ldots}
  (mid-weight wine)
  \\ DRINK is HUMAN
  \ex \eng{A provocative fusion of sultry red varietals, this wine
    knows how to have a good time.} (intense, medium-bodied wine)
  \\ DRINK is HUMAN
  \ex \eng{It is criminally delicious and once you have made it you
    will be addicted.} (rendang)
  \\ FOOD is DRUG
  \ex \eng{\ldots{}, it was love at first smell.} (rendang)
   \\ FOOD is HUMAN
  \ex \eng{Very addictive beef rendang, cosy environment}
  \\ FOOD is DRUG
  \ex \eng{The sedap  beef rendang has an addictive rempah sauce prepared with fresh ingredients every day}
  \\ FOOD is DRUG
\end{exe}
\item Lakoff and Johnson (1980) proposed the following image schemas for love:
  \begin{itemize}
  \item LOVE is a JOURNEY
  \item LOVE is a FORCE (electromagnetic, gravitational)
  \item LOVE is WAR
  \end{itemize}
Organise the following metaphors into the above three schemas.
\begin{exe}
  \ex \textit{They lost their momentum}
  \\  LOVE is a FORCE
  \ex \textit{There were sparks between us}
  \\  LOVE is a FORCE
  \ex \textit{Look how far we've come}
  \\  LOVE is a JOURNEY
\ex \textit{We're at a crossroad}
  \\  LOVE is a JOURNEY
  \ex \textit{He overpowered her}
  \\ LOVE is WAR
\ex \textit{I could feel the electricity between them}
  \\  LOVE is a FORCE
\ex \textit{We'll just have to go our separate ways}
  \\  LOVE is a JOURNEY
\ex \textit{We can't turn back now}
  \\  LOVE is a JOURNEY
\ex \textit{His whole life revolves around her}
  \\  LOVE is a FORCE
\ex \textit{She is besieged by suitors}
  \\ LOVE is WAR
\ex \textit{They are uncontrollably attracted to each other}
  \\  LOVE is a FORCE
\ex \textit{He is known for his conquests}
  \\ LOVE is WAR
\end{exe}
Can you think of other metaphors that do not fit into the above three schemas?

\item For any two languages that you know, discuss similarities and
  differences in conventionalized metaphors of body parts
  (e.g. \eng{hand of a watch}, \jpn[well-known (lit: face is wide)]{kao-ga hiroi}).
  \begin{itemize}
  \item \jpn{atama ga katai} (頭がかたい) ``stubborn --- head is hard''
    \\ HEAD is MIND; THOUGHT is OBJECT
  \item \jpn{kuchi ga karui} (口が軽い) ``loose lipped --- mouth is
    light''
    \\ can't keep a secret
    \\ MOUTH is VOICE; THOUGHT is OBJECT
  \item \jpn{kuchi ga katai} (口がかたい) ``close mouthed --  mouth is
    hard''
    \\ can keep a secret
    \\ MOUTH is VOICE; THOUGHT is OBJECT

 \item \jpn{kuchi ga omoi} (口が重い)``close mouthed --  mouth is
   heavy''
   \\ generally keeps quiet
   \\ MOUTH is VOICE; THOUGHT is OBJECT
  \item \jpn{koshi ga karui} (腰が軽い) ``flighty --- hips are  light''
    \\ quick to do something, easy
    \\ BODY-PART is PERSONALITY
 \item \jpn{koshi ga omoi} (腰が重い)``reluctant --  hips are heavy''
   \\ slow to do something
   \\ BODY-PART is PERSONALITY
  \end{itemize}
  \newpage
\item \textbf{The Bladder Slugs of Yik}
  You are Chief Linguist Yale aboard the starship Benjamin Lee Whorf
  orbiting the gaseous giant Arcturus IV. Your most pressing task is
  to compile an analysis of the temporal system of the Yik language of
  the freefalling Bladder Slugs. Given the following translations from
  Yik, what metaphor(s) seem to prevail in the Yik temporal system?
  Offer explanations (feel free to draw diagrams).
  \begin{exe}
    \ex \textit{I perceived a large glabbage upperday.}
    \ex \textit{The time for implosion is just below us. }
    \ex \textit{The pressure increases, the light is dimming, I'm plummeting old.}
    \ex \textit{How deep until we fall on dense times.}
    \ex \textit{Three days above I consumed a large splodj.}
    \ex \textit{In the rarified days of my youth, I set my life on a helical path.}
    \ex \textit{The foolish Yik lives like a falling space rock.}
    \ex \textit{At darkest bottom, we all meet at the centre.}
    \ex \textit{The aliens, who live for eternity high above the days of our youth, believe the universe is expanding. But according to the great physicist Albort Einslug, it is merely moving up into its own ``past''.}
    \ex \textit{All lives converge. At impact, we will share our common destiny.}
    \ex \textit{I hope our bladderlings will rise into the upper reaches of the brightest past.}
    \ex \textit{I believe the shadows of our downtime bladderings fell across us upperday.}
%%% was upperday
  \end{exe}
From:  Alan Dench, Department of Anthropology University of Western
Australia, 1991, posted on the linguist list 2.411: \url{https://linguistlist.org/issues/2/2-411.html}

\begin{itemize}
\item FUTURE is DOWN; PAST is UP
\item DEPTH is TIME
\item IMPACT is DEATH
\item LIGHT is YOUTH
\item LIFE is a JOURNEY
\end{itemize}
\end{enumerate}

\vfill
\paragraph{Acknowledgments} These questions are partially
based on exercises from Saeed (2003).
\end{document}

%%% Local Variables: 
%%% coding: utf-8
%%% mode: latex
%%% TeX-PDF-mode: t
%%% TeX-engine: xetex
%%% End: 
