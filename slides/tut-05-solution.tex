\documentclass[a4paper]{article}

\title{\vspace*{-30mm}HG2002: Solutions to Tutorial 5\\   Situations}
\author{Francis Bond \url{<bond@ieee.org>}}
\date{}%2011-08-15}

\usepackage{polyglossia}
\setmainlanguage{english}
\setmainfont[Ligatures=TeX]{TeX Gyre Pagella}
\setsansfont[Ligatures=TeX]{TeX Gyre Heros}
\usepackage{xeCJK}
\setCJKmainfont{Noto Sans CJK JP}
\newcommand{\ans}[1]{\hfill{#1}}
%\newcommand{\ans}[1]{}

\usepackage{multicol}
%\Restriction{}
%\rightfooter{}
%\leftheader{}
%\rightheader{}
\usepackage{mygb4e}
\newcommand{\lex}[1]{\textbf{\textit{#1}}}
\newcommand{\lx}[1]{\textbf{\textit{#1}}}
\newcommand{\eng}[1]{\textit{#1}}
\newcommand{\ix}{\ex\it}
\newcommand{\con}[1]{\textsc{#1}}
\usepackage{url}
\usepackage[normalem]{ulem}
\newcommand{\ul}[1]{\uline{#1}}
\newcommand{\ull}{\uuline}
\newcommand{\txx}[1]{\textbf{#1}}

\begin{document}
\maketitle

\begin{enumerate}


\item Are the following verbs \textbf{stative} or \textbf{dynamic}?
  What are the tests that you have used in order to decide if they are
  stative or otherwise?

  \begin{itemize}
  \item \textbf{stative:} \lex{own, possess, know, last, ?lack}
  \item \textbf{dynamic:} \lex{comprise, imitate, resemble, seize,
      think, lose}
  \end{itemize}
  Tests: can it take progressive? can you use it in imperative?
%\lex{comprise, own, imitate, possess, know, resemble, lack, seize, last, think, lose}
% stative own, possess, know  *? lack, last 


\item Some verbs may describe \textbf{telic} (bounded) or \textbf{atelic} (unbounded)
  processes, depending on the form of their complements.  Below is a
  list of verb phrases. For each one, decide if it is telic or atelic,
  then see if you can change this value by altering the verb’s
  complement.
  \begin{itemize}
  \item \textbf{telic:} \lex{rig an election, ripen, walk to the
      station}
    \\ \lex{ate two oranges, swim a mile}
  \item \textbf{atelic:} \lex{ate oranges, swim}
    \\   \lex{rig elections,  walk toward the  station}
  \end{itemize}
  Tests: does it combine with \eng{in 10 minutes}/ \eng{for 10 minutes}
%\lex{ate oranges, swim, rig an election, ripen, walk to the station}

\item Modal verbs can be used to convey \textbf{epistemic} or
  \textbf{deontic} modality. In the following sentences, discuss what
  the modal verbs tell us about the speaker’s attitude.

  \begin{exe}
    \ex \eng{This could be our bus now.}
    \\ E: maybe it is, I can't see clearly
    \\ D: if we paid the deposit 
    \ex \eng{They would be very happy to meet you.}
    \\ E: if you took the time to meet them
    \ex \eng{You must be the bride's father.}
    \\ E: I think it is the case that you are the bride's father
    \\ D: I need an actor to play this role, you do it!
    \ex \eng{The bus should be here soon.}
    \\ E: I think it will be here soon
    \\ D: It is due, it has an obligation to be here
    \ex \eng{It might rain this afternoon.}
    \\ E: It is possible that it will rain this afternoon
    \ex \eng{I will study hard.}
    \\ E: tomorrow, I am gonna study hard
    \\ D: I intend to study hard
  \end{exe}
\newpage
\item These sentences be used to convey \textbf{epistemic} or
  \textbf{deontic} modality. Explain the difference between the two
  readings, then translate the sentences into a language of your
  choice, and see if the ambiguity remains.
  \begin{exe}
    \ex \eng{You must be very tactful.}
    \\ D: From what I've heard it is the case that you are very tactful
    \\ E: You have an obligation to be very tactful
    \ex \eng{You will not leave this room early.}
    \\ D: My belief is that you will remain in this room until
      the appointed time
    \\ E: I am telling you to remain in this room until
      the appointed time
    \ex \eng{We should be home before five.}
    \\ D: Given the circumstances, I expect that we will be home by five
    \\ E: We are under an obligation to be home by five
    \ex \eng{Students may do their homework in groups. }
    \\ D: It might be the case that students will do their homework in groups
    \\ E: Students have permission to do their homework in groups
  \end{exe}


\item Although English does not mark \textbf{evidentiality} grammatically, it
  can be expressed in other ways.  Consider the following situation:
$S$ ``\eng{Kim bit Sandy} ''.  How could you express the following situations:
  \begin{exe}
    \ex You think $S$ is true, but have no evidence
    \\ \eng{I think Kim bit Sandy}
    \ex You saw $S$ occur
    \\ \eng{I saw Kim bite Sandy}
    \ex You saw a bite mark on Sandy, matching Kim's dental work
    \\ \eng{I deduce Kim bit Sandy from the bitemark}
    \\ キムが サンディを 噛だ ようだ 
    \ex Someone told you $S$
    \\ \eng{I heard that Kim bit Sandy}
    \\ キムが サンディを 噛だ そうだ 
    \ex You are Sandy, and you experienced $S$ 
    \\ \eng{Kim bit me}
  \end{exe}
  Are any of these expressed grammatically in a language that you
  speak? 

  \newpage
\item Some verbs allow the form of the verb in an embedded
  \textit{that}-clause to be subjunctive (shown as \ull{subjunctive form}).
  \begin{exe}
  \ex \textit{Kim \ul{proposes} that the meeting \ull{be} recorded.}
  \ex *\textit{Kim \ul{thinks} that the meeting \ull{be} recorded.}
  %\ix Kim proposes that the meeting \ul{should be} recorded.
  \end{exe}
  Which of the following verbs may take the subjunctive (show with
  examples): \\ \lex{require, urge, remember, command, report,
    suggested, insist, deny, promise}
  \begin{itemize}
  \item \textbf{subjunctive:}\lex{require, urge, command, 
      suggested, insist}
     \begin{itemize}
     \item \eng{Kim requires that the meeting be/$^?$is recorded}
        \item \eng{Kim suggested that the meeting be recorded}
      ``that the meeting should be recorded in the future''
    \end{itemize}
  \item \textbf{no subjunctive:}\lex{remember, report,
    suggested,   deny, promise}
    \begin{itemize}
    \item \eng{Kim remembers that the meeting was/$^*$be recorded}
    \item \eng{Kim suggested that the meeting is/was recorded}
      ``that the meeting has been recorded already''
    \end{itemize}
  \end{itemize}
  Isn't it amazing that we have this stored in our brain somehow!


% \item For each of the situation types (State, Activity,
%   Accomplishment, Punctual, Achievement) try to find an example of its
%   use in the text you are annotating for project one.  Email the
%   examples you found to your tutor, in the following format:
%   \begin{flushleft}
%     SID: sentence \\
%     SITUATION: verb
%   \end{flushleft}
%    For example:
%   \begin{flushleft}
%     11903: One day , the Lord Buddha was strolling around the edge of the lotus lake in heaven .\\
%     ACTIVITY: stroll
%   \end{flushleft}
\end{enumerate}


\end{document}

%%% Local Variables: 
%%% coding: utf-8
%%% mode: latex
%%% TeX-PDF-mode: t
%%% TeX-engine: xetex
%%% End: 
