\documentclass[a4paper]{article}

\title{HG2002: Tutorial 9\\  Componential Analysis}
\author{Francis Bond \url{<bond@ieee.org>}}
\date{}%2011-08-15}
\usepackage[margin=25mm]{geometry}
\newcommand{\ans}[1]{\hfill{#1}}
%\newcommand{\ans}[1]{}

\usepackage{multicol}
%\Restriction{}
%\rightfooter{}
%\leftheader{}
%\rightheader{}
\usepackage{mygb4e}
\newcommand{\lex}[1]{\textbf{\textit{#1}}}
\newcommand{\lx}[1]{\textbf{\textit{#1}}}
\newcommand{\ix}{\ex\it}
\newcommand{\con}[1]{\textsc{#1}}
\newcommand{\eng}[1]{\textit{#1}}
\usepackage{url}
\newcommand{\txx}[1]{\textbf{#1}}
\newcommand{\cmp}[1]{{[\textsc{#1}]}}
\newcommand{\sr}[1]{\ensuremath{\langle}#1\ensuremath{\rangle}}
\usepackage[normalem]{ulem}
\newcommand{\ul}{\uline}
\newcommand{\ull}{\uuline}
\newcommand{\wl}{\uwave}
\newcommand{\vs}{\ensuremath{\Leftrightarrow}~}


\begin{document}
\maketitle

\begin{enumerate}
\item Using semantic components, analyze the following words:
  \begin{quote}
  \lex{son, daughter, child, mother, father, parent, grandfather,
    grandmother, grandparent}
  \end{quote}
  Discuss whether a binary format would be an advantage here.
  \\ You may use two place relations in your descriptions (e.g. \cmp{sibling-of[x,y]}.
  \\ If you speak a language that makes additional distinctions in this area, also describe them (e.g. maternal grandmother, \ldots).


% child (X,Y): +CHILD-OF(X,Y) | or  +CHILD-OF
% son (X,Y): +MALE(X) +CHILD-OF(X,Y) | or  +MALE +CHILD-OF
% daughter (X,Y): -MALE(X) +CHILD-OF(X,Y) | or  -MALE +CHILD-OF
% parent (X,Y): +CHILD-OF(Y,X) | or  +PARENT-OF <-note we need a new relation
% father (X,Y): +MALE(X) +CHILD-OF(Y,X) | or  +MALE +PARENT-OF
% mother (X,Y): -MALE(X) +CHILD-OF(Y,X) | or  -MALE +PARENT-OF
% grandparent (X,Z): +CHILD-OF(Y,X) +CHILD-OF(Y,Z)  | or  +GRANDPARENT-OF ? not so good
% grandfather (X,Z): +MALE(X) +CHILD-OF(Y,X)  +CHILD-OF(Y,Z)
% grandmother (X,Z): -MALE(X) +CHILD-OF(Y,X)  +CHILD-OF(Y,Z)

%%% Note that in English, fact that there are two arguments shows up in det. choice
%%% "My son is sleepy" (not "a/the son is sleepy")
%%% can also do with parent-of and reversed elements, likewise +/-FEMALE
%%% really need two place predicates to get a good def of grandparents
%%% if there is time at the end, maybe consider /ancestor/ :-)
%%% maybe discuss markedness/defaults

\item  \label{cia} Which of the following participate in the
  \textbf{causative/inchoative alternation}.
  \begin{exe}
    \ex \textit{The goalkeeper bounced the ball.}
    \ex \textit{The assassin murdered the general.}
    \ex \textit{The waiter melted the chocolate.}
    \ex \textit{Charlie built the new swimming pool.}
    \ex \textit{The people lowered the boat.}
    \ex \textit{Kim worried Sandy.}
    \ex \textit{The censors destroyed the film.}
    \ex \textit{Jo dried the clothes.}
  \end{exe}
  For those verbs that do undergo the alternation, translate them
  into a language of your choice and report on whether the
  translations undergo a similar alternation.
\item Levin and Rapaport Hovav (1995: 102--5) argue that transitive
  verbs which do not undergo the \textbf{causative/inchoative
    alternation} need an intentional and volitional Agent.  In contrast,
  verbs that undergo this alternation should also allow a non-Agent subject:
  \begin{enumerate}
  \item \textit{John broke the window with a rock} \hfill Agent Subject
  \item \textit{The rock broke the window} \hfill Non-Agent (Instrument) Subject 
  \item \textit{The window broke} \hfill Inchoative Alternation
  \end{enumerate}
  Test this hypothesis on the sentences from Question~\ref{cia}.
\item Consider the following semantic and syntactic tests for countability:
  \begin{itemize}
  \item Semantic: Can it be divided and still use the same name 
    (\textbf{divisibility}):
    \begin{itemize}
    \item Mass:     \eng{half some gold} is \eng{gold}
    \item Count: \eng{half a dog} is not \eng{a dog} 
    \end{itemize}
  \item Syntactic: does it co-occur with \eng{much} or \eng{many}
    (\textbf{enumerability}):
    \begin{itemize}
    \item Mass:  \eng{I don't have \ul{much gold}}
    \item Count: \eng{I don't have \ul{many dogs}} 
    \end{itemize}
  \end{itemize}
  Classify the following nouns using these tests:
  \begin{quote}
    \lex{monkey, program, software, chair, furniture, 
      beer, icecream, curry, chocolate,
      chicken, salmon, potato, rice, oats, mink}
  \end{quote}
  Do the tests always give unique results?  If not, why not?
\end{enumerate}
\vfill
\paragraph{Acknowledgments} These questions are partially
based on exercises from Saeed (2003).
\end{document}
