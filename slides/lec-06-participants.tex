\documentclass[headrule,footrule]{foils}

%%
%%%  Macros
%%%
%%% fonts-sil-charis for IPA in week 5

\newcommand{\logo}{HG2002 (2021)}
\usepackage[hidelinks]{hyperref}

\newcommand{\header}[3]{%
  \title{\vspace*{-2ex} \large 
    HG2002 Semantics and Pragmatics
% \thanks{Creative
%       Commons Attribution License: 
%       you are free to share and adapt as long as you give 
%       appropriate credit and add no additional restrictions: 
%       \protect\url{https://creativecommons.org/licenses/by/4.0/}.
%     }
    \\[2ex] \Large  \emp{#2} \\ \emp{#3}}
  \author{\blu{Francis Bond}   \\ 
    \normalsize  \textbf{Division of Linguistics and Multilingual Studies}\\
    \normalsize  \url{http://www3.ntu.edu.sg/home/fcbond/}\\
    \normalsize  \texttt{bond@ieee.org}}
  \MyLogo{\logo}
  % \MyLogo{奈良女子大学:欧米言語情報理論II}
  \date{#1
    \\  \url{https://bond-lab.github.io/Semantics-and-Pragmatics/}
\\[.5ex] \footnotesize Creative  Commons Attribution License:  you are free to share and adapt 
\\[-.25ex] \footnotesize   as long as you give    appropriate credit and add no
additional restrictions: 
\\ \small  \protect\url{https://creativecommons.org/licenses/by/4.0/}.
}
  % \renewcommand{\logo}{#2}
  % \special{! /pdfmark where
  %   {pop} {userdict /pdfmark /cleartomark load put} ifelse
  %   [ /Author (Francis Bond)
  %   /Title (#1: #2)
  %   /Subject (HG2002: Semantics and Pragmatics)
  %   /Keywords (Semantics, Pragmatics, Meaning)
  %   /DOCINFO pdfmark}
  %   }
  \hypersetup{%
    final       = true,
    colorlinks  = true,
    urlcolor    = blue,
    citecolor   = blue,
    linkcolor   = MidnightBlue,
    unicode     = true,
    pdfauthor   = {Francis Bond},
    pdfkeywords = {Semantics, Pragmatics, Meaning},
    pdftitle    = {#1: #2},
    pdfsubject  = {HG2002 Semantics and Pragmatics; License CC BY 4.0}
  }
}


\usepackage[a4paper,landscape]{geometry}
%\usepackage[dvips]{xcolor}
\usepackage[dvipsnames,x11names]{xcolor}
\usepackage{graphicx}
\newcommand{\blu}[1]{\textcolor{blue}{#1}}
\newcommand{\grn}[1]{\textcolor{green}{#1}}
\newcommand{\hide}[1]{\textcolor{white}{#1}}
\newcommand{\emp}[1]{\textcolor{red}{#1}}
\newcommand{\txx}[1]{\textbf{\textcolor{blue}{#1}}}
\newcommand{\lex}[1]{\textbf{\mtcitestyle{#1}}}


\usepackage{amsmath,latexsym}
\usepackage{pifont}
\renewcommand{\labelitemi}{\textcolor{violet}{\ding{227}}}
\renewcommand{\labelitemii}{\textcolor{purple}{\ding{226}}}

\newcommand{\subhead}[1]{\noindent\textbf{#1}\\[5mm]}

\newcommand{\Bad}{\emp{\raisebox{0.15ex}{\ensuremath{\mathbf{\otimes}}}}}
\newcommand{\bad}{*}

\newcommand{\com}[1]{\hfill \textnormal{(\emp{#1})}}%
\newcommand{\cxm}[1]{\hfill \textnormal{(\txx{#1})}}%
\newcommand{\cmm}[1]{\hfill \textnormal{(#1)}}%

\usepackage{relsize,xspace}
\newcommand{\into}{\ensuremath{\rightarrow}\xspace}
\newcommand{\ent}{\ensuremath{\Rightarrow}\xspace}
\newcommand{\nent}{\ensuremath{\not\Rightarrow}\xspace}
\newcommand{\tot}{\ensuremath{\leftrightarrow}\xspace}
\usepackage{url}
\newcommand{\lurl}[1]{\MyLogo{\url{#1}}}

\usepackage{mygb4e}
\let\eachwordone=\itshape
\newcommand{\lx}[1]{\textbf{\textit{#1}}}
\newcommand{\ix}{\ex\it}

\newcommand{\cen}[2]{\multicolumn{#1}{c}{#2}}
%\usepackage{times}
%\usepackage{nttfoilhead}
\newcommand{\myslide}[1]{\foilhead[-25mm]{\raisebox{12mm}[0mm]{\emp{#1}}}\MyLogo{\logo}}
\newcommand{\myslider}[1]{\rotatefoilhead[-25mm]{\raisebox{12mm}[0mm]{\emp{#1}}}}
%\newcommand{\myslider}[1]{\rotatefoilhead{\raisebox{-8mm}{\emp{#1}}}}

\newcommand{\section}[1]{\myslide{}{\begin{center}\Huge \emp{#1}\end{center}}}



\usepackage[lyons,j,e,k]{mtg2e}
\renewcommand{\mtcitestyle}[1]{\textcolor{teal}{\textsl{#1}}}
%\renewcommand{\mtcitestyle}[1]{\textsl{#1}}
\newcommand{\ja}[1]{\mtcitestyle{\makexeCJKactive #1\makexeCJKinactive}}
\newcommand{\chn}{\mtciteform}
\newcommand{\zsm}{\mtciteform}
%\newcommand{\cmn}[1]{make\cjkactive\mtciteform#1\makecjkinactive}
\newcommand{\iz}[1]{\textup{\texttt{\textcolor{blue}{\textbf{#1}}}}}
\newcommand{\con}[1]{\textsc{#1}}
\newcommand{\gm}{\textsc}
\newcommand{\cmp}[1]{{[\textsc{#1}]}}
\newcommand{\sr}[1]{\ensuremath{\langle}#1\ensuremath{\rangle}}
\usepackage[normalem]{ulem}
\newcommand{\ul}{\uline}
\newcommand{\ull}{\uuline}
\newcommand{\wl}{\uwave}
\newcommand{\vs}{\ensuremath{\Leftrightarrow}~}
%%% theta role
\newcommand{\tr}[1]{\textcolor{Chartreuse4}{\textsc{#1}}}
%%% theta grid
\newcommand{\grid}[1]{\ensuremath{\langle}\tr{#1}{\ensuremath{\rangle}}}

%%%
%%% Bibliography
%%%
\usepackage{natbib}
%\usepackage{url}
\usepackage{bibentry}
%\usepackage{CJKutf8}


\usepackage{fontenc}
\usepackage{polyglossia}
\setmainlanguage{english}
\setotherlanguages{tamil}
\setmainfont[Ligatures=TeX]{TeX Gyre Pagella}
\setsansfont[Ligatures=TeX]{TeX Gyre Heros}
\newfontfamily\ipafont{Charis SIL}
\newcommand\ipa[1]{\mtcitestyle{\ipafont #1}}


\usepackage{xeCJK}
\makexeCJKinactive
\newcommand{\zh}[1]{\mtcitestyle{\makexeCJKactive #1\makexeCJKinactive}}
%\newcommand{\ja}[1]{\makexeCJKactive #1\makexeCJKinactive}
\setCJKmainfont{Noto Sans CJK JP}
\setCJKsansfont{Noto Sans CJK SC}
\setCJKmonofont{Noto Sans CJK SC}

\newfontfamily\tamilfont[Script=Tamil]{Noto Sans Tamil}
\newfontfamily\tamilfontsf[Script=Tamil]{Noto Sans Tamil}
\newcommand{\tam}[1]{\texttamil{#1}}
%%% From Tim
\newcommand{\WMngram}[1][]{$n$-gram#1\xspace}
\newcommand{\infers}{$\rightarrow$\xspace}


\usepackage{rtrees,qtree}
\renewcommand{\lf}[1]{\br{#1}{}}
\usepackage{avm}
%\avmoptions{topleft,center}
\newcommand{\ft}[1]{\textsc{#1}}
\renewcommand{\val}[1]{\textit{#1}}
\newcommand{\typ}[1]{\textit{#1}}
\avmfont{\sc}
\avmvalfont{\sc}
\renewcommand{\avmtreefont}{\sc}
\avmsortfont{\it}


%%% From CSLI book
\newcommand{\mc}{\multicolumn}
\newcommand{\HD}{\textbf{H}\xspace}
\newcommand{\el}{\< \>}
\makeatother
\long\def\smalltree#1{\leavevmode{\def\\{\cr\noalign{\vskip12pt}}%
\def\mc##1##2{\multispan{##1}{\hfil##2\hfil}}%
\tabskip=1em%
\hbox{\vtop{\halign{&\hfil##\hfil\cr
#1\crcr}}}}}
\makeatletter

%\usepackage{tipa}
\usepackage{multicol}


\newcommand{\task}{\marginpar{\large ~~~\textbf{?}}}
\newcommand{\sh}[1]{\href{https://www.arthur-conan-doyle.com/index.php?title=#1}{#1}}

\usepackage{tikz}
\usepackage{tikz-qtree}
\usepackage{forest}

% \usepackage{tree-dvips}
% \newcommand{\sa}[2]{\node{c#1}{\iz{#2}}}%\nodebox{c#1}}
\usepackage{pst-node}
\usepackage{pst-plot}

\usepackage{tree-dvips} % ALT Ontology
\newcommand{\sa}[2]{\texttt{\small #1}:\iz{#2}} %\nodebox{c#1}}
%\newcommand{\sagt}[2]{\rnode{c#1}{\iz{#2}}} %\nodebox{c#1}}
\newcommand{\sagt}[2]{\rnode{c#1}{\izn{#1}{#2}}} %\nodebox{c#1}}
\newcommand{\izn}[2]{\iz{#1:#2}}



\begin{document}
%\begin{CJK}{UTF8}{min}
\header{Lecture 6}{Participants}{}\maketitle

%\include{schedule}

\myslide{Overview}

\begin{itemize}\addtolength{\itemsep}{-1ex}
\item Revision: Situations
  \begin{itemize}
  \item Verb Types
  \item TAM: Tense, Aspect and Modality
  \item Mood and Evidentiality
  \end{itemize}
\item Thematic Roles
  \begin{itemize}
  \item Grammatical Relations and Thematic Roles
  \item Verbs and Thematic Role Grids
  \item Problems with Thematic Roles
  \item The Motivation for Identifying Thematic Roles
 \item Voice
 \end{itemize}
\item Classifiers and Noun Classes

\item Next Lecture: Chapter 7: Context and Inference
\end{itemize}

%%%
%%% this changes each year, so keep separate
%%%
\include{schedule}


\section{Revision: Situations}

\myslide{Summary of Situation}
\begin{itemize}\addtolength{\itemsep}{-1ex}
\item Verb/Situation Types
\begin{itemize}
\item Stative 
\item  Dynamic
  \begin{itemize}
  \item  Punctual
  \item  Durative
    \begin{itemize}
    \item  Telic/Resultative 
    \item  Atelic
    \end{itemize}
  \end{itemize}
\end{itemize}
\item Tense/Aspect and Time: R, S and E
\item Modality
  \begin{itemize}
  \item Epistemic
  \item Deontic: Permission, Obligation
  \end{itemize}
\item Evidentiality
\end{itemize}


\myslide{Situation Types}

\noindent\begin{tabular}{lcccl}
Situations     & Stative & Durative & Telic & Examples \\ \hline
State          & +       & +        &       & \eng{desire, know} \\
Activity       & $-$       & +        & $-$     & \eng{run, drive a car} \\
Accomplishment & $-$       & +        & +     & \eng{bake, walk to school, build} \\
Punctual       & $-$       & $-$        & $-$     & \eng{knock, flash} \\
Achievement    & $-$       & $-$        & +     & \eng{win,  start}  
\end{tabular}


\myslide{Tense and Time}
\begin{itemize}
\item  Locate a situation to a point in time: 
  \\ S = speech point;  R = reference time:  E = event time
\begin{itemize}
\item  Simple Tense
  \begin{itemize}
  \item Past ($R = E < S$) \eng{saw}
\item  Present ($R = S = E$) \eng{see} 
\item  Future ($S < R = E$) \eng{will see}
\end{itemize}
\item Complex Tense
\begin{itemize}
\item  Past Perfect ($E < R < S$) \eng{had seen}
\item  Present Perfect ($E < R =S$) \eng{had seen}
\item  Future Perfect ($S< E < R$) \eng{had seen}
\end{itemize}
\end{itemize}
\end{itemize}

\myslide{Aspect in General}

\begin{itemize}
\item  \txx{Perfective} focus on the end point
  \begin{itemize}
  \item  \txx{Completive} \eng{I built the building}
  \item  \txx{Experiential} \eng{I have built the building} %(E < R = S) ???
  \end{itemize}
\item  \txx{Imperfective} 
  \begin{itemize}
  \item  \txx{Progressive} \eng{I was listening/I am listening}
  \item  \txx{Habitual} \eng{I listen to the Goon Show}
\end{itemize}
\item Different languages grammaticalize different things
\end{itemize}


\myslide{Mood: Knowledge vs Obligation}
\begin{itemize}
\item  \txx{Epistemic modality}: Speaker signals degree of  
knowledge.
\begin{exe}
  \ex\eng{You can drive this car} \cmm{You are able to}
\end{exe}
\item  \txx{Deontic modality}: Speaker signals his/her attitude  
to social factors of obligation and permission.
\begin{itemize}
\item \txx{Permission}
  \begin{exe}
    \ex\eng{You can drive this car}  \cmm{You have permission to}
    \ex\eng{You may drive this car}  
  \end{exe}
\item \txx{Obligation}
  \begin{exe}
    \ex\eng{You must drive this car}  \cmm{You have an obligation to}
    \ex\eng{You ought to drive this car}
  \end{exe}
% \item  Question: How do you use `must', `confirm', and 
% `sure' in Singlish? ???
\end{itemize}
\end{itemize}

\myslide{Mood more Generally}

\begin{itemize}
\item Grammatical Inflection used to mark modality is called \txx{mood}
  \begin{itemize}
  \item \txx{indicative} expresses factual statements
  \item \txx{conditional} expresses events dependent on a condition
  \item \txx{imperative} expresses commands
  \item \txx{injunctive} expresses pleading, insistence, imploring
  \item \txx{optative} expresses hopes, wishes or commands 
  \item \txx{potential} expresses something likely to happen
  \item \txx{subjunctive} expresses  hypothetical events; opinions or emotions
  \item \txx{interrogative} expresses questions
\end{itemize}
\item English only really marks imperative and subjunctive, and then only on \lex{be}
  \begin{exe}
    \ix \eng{\ul{Be} good!}
    \ix \eng{If I \ul{were} a rich man}
  \end{exe}
\end{itemize}



\section{Participants} 

\myslide{Thematic Roles}

In this section we talk about the relations between the participants
in a situation and the situation itself.

\begin{itemize}
\item  \txx{Thematic roles} are the roles played by the parts of the sentence that 
correspond to the participants in the situation described

\item  They classify relations between entities in a situation

\item  Also known as
\begin{itemize}
\item  Deep case \citep{Fillmore:1968}
\item  Thematic roles; Theta roles;   $\theta$-roles
\item  Semantic Roles; Participant Roles
\end{itemize}
\end{itemize}
 
\myslide{Roles link different alternations}
\begin{exe}
  \ex\eng{Kim patted Sandy}
  \ex\eng{Sandy was patted by Kim}
\end{exe}
\begin{itemize}
\item  Which is the \txx{Subject} and which the \txx{Object} in these sentences?\task
\item  What are the thematic roles of Kim and Sandy?\task
\end{itemize}

\myslide{Thematic Roles}
\MyLogo{Some prose and examples from \citet{Bender:2013}}
\begin{itemize}
\item \txx{\tr{agent}} (takes \eng{deliberately, on purpose, what did X do?})
  \begin{quote}
    A participant which the meaning of the verb specifies as doing or causing something,
    possibly intentionally. 
  \end{quote}
  \begin{itemize}
  \item The initiator, performer of controller of an action; typically volitional, typically animate
  \item Typically \textsc{subject}
  \end{itemize}
  \begin{exe}
      \ex\eng{\eng{\ul{Kim} kicked Sandy}}
      \ex\eng{\eng{\ul{The ogre} leaped into the fray}}
      \ex\eng{\eng{\ul{The student} watched the video}}
    \end{exe}
\item (\txx{\tr{actor}}) generalization of \txx{\tr{agent}} that allows non-volitional, non-actor:
   if you use this, then \txx{\tr{agent}} is restricted to animate, volitional participants
\newpage
\item  \txx{\tr{patient}} (\eng{What happened to X?})
  \begin{quote}
    A participant which the verb characterizes as having something
    happen to it, and as being affected by what happens to it.
    %Examples: objects of kill, eat, smash but not those of watch, hear
    %and love.
  \end{quote}
  \begin{itemize}
  \item The undergoer of an action
  \item  Undergoes change in state usually, both animate and 
    inanimate
  \item Typically \textsc{object}
  \end{itemize}
  \begin{exe}
    \ex\eng{\eng{Kim kicked \ul{Sandy}}}
    \ex\eng{\eng{The ogre ate  \ul{the dog}}}
    \ex\eng{$^\#$\eng{The student watched \ul{the video}}}
    \ex\eng{$^\#$\eng{I heard \ul{a sound}}}
  \end{exe}
\newpage
\item  \txx{\tr{theme}}
  \begin{quote}
     A participant which is characterized as changing its position or condition, or as
being in a state or position. 
  \end{quote}
  \begin{itemize}
  \item  Moved, location or state is described
  \item Typically \textsc{object}
\end{itemize}
\begin{exe}
  \ex \eng{\eng{Hiromi put \ul{the book} on the shelf}}
  \ex \eng{\eng{Freddy gave you \ul{the chocolate}}}
  \ex \eng{\eng{\ul{The book} is on the shelf}}
  \ex \eng{\eng{\ul{The protagonist} died}}
  \ex *\eng{\eng{\ul{The dog} walked home}}
\end{exe}
\newpage

\item  \txx{\tr{experiencer}}
  \begin{quote}
    A participant who is characterized as aware of something.
  \end{quote}
  \begin{itemize}
  \item   Non-volitional, displaying awareness of action, state
  \item Typically \textsc{subject}
  \end{itemize}
  \begin{exe}
  \ex\eng{\ul{Liling} heard thunder}
  \ex\eng{\ul{Jo} felt sick}
  \ex\eng{The lecturer annoyed \ul{the students}}
\end{exe}
\newpage
\item  \txx{\tr{beneficiary}}
  \begin{itemize}
  \item   for whose benefit the action was performed
  \item   Typically indexed by \eng{for} PP in English
    \\ or \textsc{object} in ditransitive verbs
  \end{itemize}
  \begin{exe}
  \ex\eng{They made \ul{me} a present}
  \ex\eng{They made a present \ul{for me}}
\end{exe}

\item  \txx{\tr{location}}
  \begin{itemize}
  \item  Place
  \item Typically indexed by locative PPs in English
  \end{itemize}
  \begin{exe}
  \ex\eng{I am living \ul{in Indonesia}}
  \ex\eng{It is \ul{on the table}}
\end{exe}
\newpage  

\item  \txx{\tr{goal}}
  \begin{itemize}
  \item  towards which something moves (lit or metaphor)
 \item  Typically indexed by \eng{to} PP in English 
   \\ or \textsc{object} in ditransitive
  \end{itemize}
  \begin{exe}
  \ex\eng{She handed the form \ul{to him}}
  \ex\eng{She handed \ul{him} her form}
  \end{exe}
\item  \txx{\tr{source}}
 \begin{itemize}
 \item  from which something moves or originates
 \item  Typically indexed by \eng{from} PP in English
 \end{itemize}
  \begin{exe}
 \ex\eng{We gleaned this \ul{from the Internet}}  
\end{exe}
\newpage
\item  \txx{\tr{stimulus}}
  \begin{itemize}
  \item  Usually used in connection with \tr{experiencer}
  \end{itemize}
  \begin{exe}
  \ex\eng{\ul{The lightning} scared them}
  \ex\eng{I don't like \ul{the lightning}}
\end{exe}
\item  \txx{\tr{instrument/manner}}
  \begin{itemize}
  \item  Means by which action is performed
  \item  Can be indexed by \eng{with} PP in English
  \end{itemize}
  \begin{exe}
    \ex\eng{I ate breakfast \ul{with chopsticks}}
  \end{exe}
\end{itemize}

\myslide{Split Themes}

\begin{itemize}
\item \citet{Jackendoff:1990} suggests
  \begin{itemize}
  \item \txx{action tier} (actor-patient)
    \\ \tr{actor, agent, experiencer, patient, beneficiary, instrument}
  \item \txx{thematic tier} (spatial)
  \\ \tr{theme, goal, source, location}
\end{itemize}
\end{itemize}

% \myslide{Case Grid}
% {\small
% \noindent\begin{tabular}{lllll}
%             & Source  & Path 
%             & Goal & Local\\ \hline
%   Active    & INSTIGATOR  & MEANS & RECIPIENT & PATIENT   \\
%   Objective & MATERIAL & INSTRUMENT & RESULT & CHANGED   \\
%   Dative    & STIMULUS & MEDIUM & EXPERIENCER & CONTENT \\ 
%             & OWNER &   PRICE   & RECIPIENT & TRANSFERRED \\
%   Locative  & SOURCE   & PATH  & GOAL & POINT  \\
%   Temporal  & FROM & DURATION& UNTIL & WHEN \\
%   Ambient   & REASON&  MANNER & AIM & CONDITION 
%   % Particle& \multicolumn{2}{c}{\cm{ga}, wa, towa} & 
%   % \multicolumn{2}{c}{\cm{de}, to, o} & \multicolumn{2}{c}{\cm{ni} e}
%   % & \multicolumn{2}{c}{\cm{o}, to} \\
%   %           & \multicolumn{2}{c}{kara, yori} &  &       &
%   %           \multicolumn{2}{c}{made to}  & \multicolumn{2}{c}{ni, de,
%   %             nituite}  \\
%   % Preposition & \multicolumn{2}{c}{\cm{from}} &
%   % \multicolumn{2}{c}{\cm{by}, with} & \multicolumn{2}{c}{\cm{to},
%   %   until}  & \multicolumn{2}{c}{\cm{in/at/on}}\\
%   %             & \multicolumn{2}{c}{for}  & \multicolumn{2}{c}{for,
%   %               around}  & \multicolumn{2}{c}{as, for} &
%   %             \multicolumn{2}{c}{with, about} 
% \end{tabular}}

% \citet{Somers:1987}: \textit{Valency and Case in Computational Linguistics}

% \begin{itemize}
% \item Combines the two tiers into a matrix
% \end{itemize}

\myslide{Theta-Grid}
\begin{itemize}\addtolength{\itemsep}{-1ex}
% \item  Goal: Explanation for the syntax-semantics 
% interface.
\item Have a semantic \txx{valence}
  (\txx{theta-grid}) %, \txx{subcategorization})

 \begin{itemize}
 \item  \lex{give}: V \grid{\ul{agent}, theme, beneficiary}
 \item  underlined role maps to subject
 \item  order of roles allows prediction of grammatical function
 \end{itemize}
\item  This is used to link the meaning with the realization
\item Distinguish between
  \begin{itemize}
  \item \txx{participant roles} depend on the verb --- in the grid (\txx{arguments})
    \begin{itemize}
    \item In general, if it takes part in an alternation: it should be in the grid.
    \end{itemize}
  \item \txx{non-participant roles} combine freely --- not in the grid (\txx{adjuncts})
    \begin{itemize}
    \item If there can be multiple instances: it should not be in the grid.
    \end{itemize}
 
  \end{itemize}
\end{itemize}
  


\myslide{Theta-Grids (continued)}

\begin{itemize}
\item Theta Grids/subcategorization are properties of meta-lexemes
  \begin{itemize}
  \item For a given sense they are constant:
    \\ \textbf{\textit{hand}}: V \grid{\ul{agent}, theme,
    beneficiary} (NP, NP, NP)
    \begin{itemize}
    \item \textit{I handed Kim the book}: 
      %\\ Sbj=AGENT, Obj=THEME, IObj =  BENEFICIARY
    \end{itemize}
  \item passivization changes the grid: 
      \\ \textbf{\textit{handed}}:  V \grid{\ul{beneficiary},
      theme, agent} (NP, NP, PP:by)
  \begin{itemize}
    \item \textit{Kim was handed the book by me}: 
      %\\ Sbj=BENEFICIARY, Obj=THEME, PP =  AGENT
    \end{itemize}
  \item Can change with alternations, voice, \ldots
  \end{itemize}
\item Theta Roles are semantic NOT syntactic
  \begin{itemize}
  \item Never \textsc{subject, object, adjective, \ldots}
  \end{itemize}
\end{itemize}


\myslide{Some Issues}
\begin{itemize}
\item  Every theory has a different set of roles

\item  From 8 to 42! (two groups at NTT) 
\item  How useful is the notion of \tr{patient} if it 
encompasses all these?
\begin{exe}
\ex\eng{The genie touched \ul{the lamp} with their nose.}
\ex\eng{The baby rubbed \ul{the lamp} with its hands.}
\ex\eng{The baby squeezed \ul{the rubber toy} with its hands.}
\ex\eng{She cracked \ul{the mirror} with a stone.}
\end{exe}
\end{itemize}



\myslide{Linking Grammatical Relations and Thematic Roles}
\begin{itemize}
\item    Thematic roles typically map onto grammatical 
  functions systematically
  \begin{itemize}
  \item  \tr{agent} is usually the subject
  \item  \tr{patient} is usually the object
  \end{itemize}
\item  It is possible to predict how arguments are linked to 
the verb from their thematic roles, and hence their 
grammatical functions.
\item Different languages show these in different ways:
  \begin{itemize}
  \item English uses position for SUBJ/OBJ and prepositions
  \item Japanese uses postpositions
    \newpage
    \MyLogo{Ablative includes source, passive-by, comparative-than, location, \ldots}
  \item Latin inflects:  \eng{familia} ``family, household''
\\[1ex]    \begin{tabular}{lll}
 & Singular & 	Plural \\ \hline
Nominative & 	familia & 	familiae\\
Accusative & 	familiam & 	familiās\\
Genitive & 	\raisebox{-1ex}[0ex][0ex]{familiae} & 	familiārum\\
Dative &    &		\raisebox{-1ex}[0ex][0ex]{familiīs}  \\
Ablative & 	familiā & 	 \\
    \end{tabular}

  \end{itemize}
\item Most language mark arguments and adjuncts slightly differently
  \begin{itemize}
  \item There are far fewer arguments (typically not more than 4)
  \item There are more adjuncts, so they are typically marked with a
    contentful marker
  \end{itemize}
\end{itemize}

\myslide{Many verbs allow alternations}
\begin{exe}
  \ex \eng{Jo broke the ice with a pickaxe}
  \\   \grid{\ul{agent}, patient,  instrument} (NP, NP, PP:with)
  \ex \eng{The pickaxe broke the ice}
  \\  \grid{\ul{instrument}, patient} (NP, NP)
  \ex \eng{The ice broke}  
  \\ \grid{\ul{patient}} (NP)
\end{exe}


\myslide{Other Predicates}

\begin{itemize}
\item Adjectives  (normally theme)
  \begin{exe}
  \ex \eng{John is tall} \grid{\ul{theme}}
  \ex \eng{John is cold [to touch]} \grid{\ul{theme}}
  \ex \eng{John is/feels cold} \grid{\ul{experiencer}}
    \\ different adjectives in e.g., Japanese:
    \\  \eng{冷たい} \jpn[cold (to touch)]{tsumetai}  vs \eng{寒い} \jpn[(feel) cold]{samui}
  \end{exe}
\item Predicative Copula (treat second NP as predicate)
  \begin{exe}
  \ex \eng{John is a boy} \grid{\ul{theme}}
  \end{exe}
\item Identity Copula (reversible)
  \begin{exe}
  \ex \eng{Kim is my teacher} \grid{\ul{theme}, theme}?
  \ex \eng{My teacher is Kim} \grid{\ul{theme}, theme}?
  \end{exe}


\end{itemize}



\myslide{Thematic Hierarchy}

\begin{itemize}
\item  The higher you are in the hierarchy the more likely 
to be subject (then object, then indirect, then 
argument PP, then adjunct PP)
\end{itemize}

\[  \hspace*{-3em}\mbox{\tr{agent}} > 
\left\{\begin{array}[c]{l} \mbox{\tr{goal/recipient}} \\ \mbox{\tr{beneficiary}} \end{array} \right\}  >
\left\{\begin{array}[c]{l} \mbox{\tr{theme}} \\ \mbox{\tr{patient}} \end{array} \right\}  >
\mbox{\tr{instrument}} > 
\mbox{\tr{location}} \]

\begin{itemize}
\item  Generally true across languages
\end{itemize}



\myslide{Dowty's Proto-Arguments}
\MyLogo{\citet{Dowty:1991}}
\begin{itemize}
\item  The \tr{Agent} Proto-Role 
\begin{itemize}
\item  Volitional
\item  Sentient (and/or perceptive)
\item  Causes event or change of state; 
\item  Movement
\end{itemize}
\item  The \tr{Patient} Proto-Role
\begin{itemize}
\item  Change of state
\item  Incremental theme (i.e. determines aspect)
\item  Causally affected by event
\item  Stationary (relative to movement of proto-agent).     
\end{itemize}
\end{itemize}

\myslide{Dowty's Argument Selection Principle}
\begin{itemize}
\item   when a verb takes a subject and an object
  \begin{itemize}
  \item  the argument with the greatest number of Proto-Agent 
    properties will be the one selected as \textsc{subject}
  \item  the one with the greatest number of Proto-Patient properties 
    will be selected as \textsc{object}
  \end{itemize}
\item   Try: \lex{threw} --- ball, the man, the dog

\item   Relatively predictive, but what about sentences 
such as:
\\  \eng{The hunger killed him}?
\end{itemize}


\myslide{Alternations}
 \MyLogo{English Verb Classes and Alternation \citep{Levin:1993}}
\begin{itemize}
\item  Many verbs have multiple theta-grids
  \begin{exe}
    \ex
    \begin{xlist}
      \ex\eng{Kim broke the window with the hammer}
      \trans \grid{\ul{agent}, patient, instrument}
      \ex\eng{The hammer broke the window}
      \trans \grid{\ul{instrument}, patient}
      \ex\eng{The window broke}
      \trans \grid{\ul{patient}}
    \end{xlist}
    \ex
    \begin{xlist}
      \ex\eng{I cut the cake with the knife}
      \trans \grid{\ul{agent}, patient, instrument}
      \ex\eng{This cake cuts easily}
      \trans \grid{\ul{patient}}
    \end{xlist}
  \end{exe}  
\item  The relations between them are called \txx{alternations}

\end{itemize}


\myslide{Voice}
\MyLogo{These are also alternations for Levin}
\begin{itemize}
\item  Another alternation that changes the number of arguments is 
\txx{voice}: passive, middle
\begin{exe}
  \ex \txx{Transitive Passive}
  \\ makes the \tr{patient} more salient
  \begin{xlist}
    \ex\eng{Kim ate Sandy}
    \ex\eng{Sandy was eaten (by Kim)}
  \end{xlist}
  \ex \txx{Ditransitive Passive} 
  \\ can make the \tr{theme} or the \tr{goal} more salient
  \begin{xlist}
    \ex\eng{Abraham gave Brown chocolate}
    \ex\eng{Abraham gave chocolate to Brown}
    \ex\eng{Chocolate was given to Brown (by Abraham)}
    \ex\eng{Brown was given chocolate (by Abraham)}
  \end{xlist}
\newpage
  \ex \txx{Transitive Middle} 
  \\ requires an adverbial, becomes a timeless generic statement 
  \begin{xlist}
    \ex\eng{They open the gate very quietly} (active)
    \ex\eng{The gate opens very quietly} (middle)
    \ex\eng{The gate opened very quietly} (inchoative)
  \end{xlist}
  \ex \txx{Intransitive Middle}
 \\ requires an adverbial, becomes a timeless generic statement 
  \begin{xlist}
    \ex\eng{The knife cuts the cake well}
    \ex\eng{The knife cuts well}
  \end{xlist}
\end{exe}
\end{itemize}

\myslide{Why so many possibilities?}

\begin{itemize}
\item So we can emphasize different participants
\item We may not know all the participants
\item We may not care about all the participants
\item There are also lexical alternations
\end{itemize}
\begin{exe}
\ex \eng{Kim \ul{killed} Sandy} vs \eng{Sandy \ul{dies}}
\ex c.f. \eng{Kim \ul{melted} the ice} vs \eng{the ice \ul{melted}}
\makexeCJKactive
\ex \glll \zh{金が} 氷を \ul{溶かした}  {~~~vs~~~}  氷が \ul{溶けた} \\ 
\makexeCJKinactive
Kim-ga koori-wo \ul{tokashita}  {} koori-ga \ul{toketa} \\
  Kim-\textsc{sbj} ice-\textsc{obj} \textit{melt:trans} 
{} ice-\textsc{sbj} \textit{melt:intrans} \\
\end{exe}


\section{Classifiers}

\myslide{Classifiers and Noun Classes}
\MyLogo{}
\begin{itemize}
\item  Many languages include special ways to classify 
nouns
\begin{itemize}
\item  Noun Classifiers (Bantu, {\ipafont Yidiɲ}, \ldots) 
\item  Numeral Classifiers (Chinese, Malay, Japanese, \ldots)
  \begin{itemize}
  \item English group nouns: \lex{flock, mob, group, pack, \ldots}
  \end{itemize}
\item  Gender (German, Spanish, \ldots)  
\end{itemize}
\item Classifiers can be marked on the noun, on the verb, on a
  separate word (a classifier) or on all words
\end{itemize}

\myslide{Examples}
\begin{exe}
  \ex \gll Bulumba walba      malan \\
  CL:HABITABLE CL:STONE  flat.rock \\
  \trans ``a flat rock for camping'' \hfill {\ipafont Yidiɲ}  \citep{Dixon:1977}
  \ex \eng[1.CL:round apple]{se-biji epel} \hfill Malay
  \ex \zh{一张\,纸} \eng[1.CL:flat paper]{yi-zhang zhi} \hfill Mandarin
  \ex \eng[the:male dog]{der Hund} \hfill German
  \ex \eng[the:neuter girl]{den Madchen}\hfill German
\end{exe}

\myslide{What gets Classified?}
\MyLogo{Allan (2001)}
\begin{itemize}
\item \txx{Taxonomic Class}: Human, Animal, Tree, Female
\item \txx{Function}: piercing, cutting, writing instrument, for eating/drinking
\item \txx{Shape}: long, flat, round (1D, 2D, 3D)
\item \txx{Consistency}: rigid, flexible
\item \txx{Size}: grab in fingers, hand, $<$ human, $>$ human
\item \txx{Location}: towns
\item \txx{Arrangement}: row, coil, heap
\item \txx{Quanta}: head, pack, flock
\end{itemize}

\myslide{Noun Classes in Bantu}

\begin{tabular}{ll}
  Class &  Semantics \\
\hline
1/2 	& sg/pl   human \\
3/4 	& sg/pl   plants, foods, non-paired body parts \\
5/6 	& sg/pl  fruits, paired body parts, \ldots \\
7/8 	& sg/pl  inanimate \\
9/10 	& sg/pl  animals \\
11/12 	& sg/p   long objects, abstracts \\
13      & small objects, birds \\
14      & masses \\
15      & infinitives
\end{tabular}
\newpage
Other elements in the sentence agree with the noun (class 8)
\begin{exe}
  \ex \gll Vi-su vidogo viwili hi-vi amba-vy-o nili-vi-nunua ni vi-kali sana \\
  vi-knife vi-small vi-teo this-vi which-vi 1.s-vi-buy be vi-sharp very \\
\trans These two small knives which I bought are very sharp
\end{exe}

\myslide{Classification}
\MyLogo{}
\begin{itemize}
\item    Is there a system for classifying nouns in a 
language that you speak? \task
\begin{itemize}
\item  What are the criteria for classification? \task
\end{itemize}
\item  Semantic change?  
\begin{itemize}
\item  How do you classify \iz{watermelon}?   (or what gender is $\sim$)\task
\item  How do you classify a \iz{grain (of rice)} \task
\item  How do you classify a \iz{human} \task
\item  How do you classify a \iz{robot} \task
\end{itemize}
\end{itemize}



\myslide{Classifiers in Japanese and Chinese}
\MyLogo{\citet{Paik:Bond:2002}}
\begin{itemize}
\item Modeling Classifier use in  Japanese and Chinese:
  \begin{itemize}
  \item Associate classifiers with semantic classes (in an ontology) by hand
  \item Most sortal classifiers select for some kind of semantic class
  \item 20\% of the classes require more than one classifier\\
    choose the most common one
  \item class \iz{961:weapon}:\\
    \jpn[knives]{-ch\=o}, \jpn[long thin things]{-hon},
    \jpn[swords]{-furi}, \jpn[machines]{-ki}
  \end{itemize}
\item Each language took around two weeks
\item Currently redoing this with WordNet and associating
  semi-automatically from a corpus (URECA projects available)
\end{itemize}

\myslide{Top four levels of the Goi-Taikei (\zh{語彙大系}) Ontology}
\begin{small}
\hspace*{-3em}\begin{tikzpicture}[grow'=right]
  \tikzset{level distance=8em,sibling distance=0em}
  \tikzset{every tree node/.style={anchor=base west}}
  \Tree 
  [.{\sa{1}{noun}} 
    [.{\sa{2}{concrete}}
      [.{\sa{3}{agent}} 
        [.{\sa{4}{person}} ]
        [.{\sa{362}{organization}} ] ]
      [.{\sa{388}{place}}
        [.{ \sa{389}{facility}} ]
        [.{\sa{458}{region}} ]
        [.{\sa{468}{natural place}} ] ]
      [.{\sa{533}{object}}
        [.{\sa{534}{animate}} ]
        [.{\sa{706}{inanimate}} ]
      ]
    ]
 [.{\sa{1000}{abstract}}  
 [.\sa{1001}{\shortstack{abstract\\thing}} 
 [.\sa{1002}{mental state} ]
 [.\sa{1154}{action} ] ]
 [.\sa{1235}{event} 
[.\sa{1236}{human activity} ]
[.\sa{2054}{phenomenon} ]
[.\sa{2304}{natural phenomenon} ]
]
 [.\sa{2422}{relation}  
[.\sa{2423}{existence} ]
[.\sa{2432}{system} ]
[.\ldots{} ] ]
]
]
\end{tikzpicture}
 \end{small}
\vspace*{-4.5ex}\begin{itemize}\addtolength{\itemsep}{-0.5ex}
\item A rich ontology for Japanese, English, Chinese and Malay 
\item \emp{2,710} semantic classes (12-levels) for common  nouns
  % \item \emp{200} classes (9-level tree structure) for proper nouns:
  %   \item Japanese word semantic dictionary has:
  %   \begin{itemize}
  %   \item \emp{100,000} common nouns
  %   \item \emp{70,000} technical terms
  %   \item \emp{200,000} proper nouns
  %   \item \emp{30,000} other words
  %   \end{itemize}
  %  \item \emp{108} classes for predicates: \emp{17,000} entries
\end{itemize}


   % The solid arrows show the direct parent and child relationship.
   % Dashed arrows show indirect relationships (skipping some levels).



\myslide{Japanese Classifiers}
\vspace*{-5mm}\begin{center}
    \begin{tabular}{lclrrlrr}
      \multicolumn{2}{l}{\textsc{classifier}} & Referents classified &  
      No. & \% &  Sample Class \\ \hline% & No. & \% \\ \hline
      \multicolumn{2}{l}{None} & Uncountable & 794 & 29.3 
      & \iz{3:agent} \\% & 34,548& 20.0\\
      \jpn{-kai} & (\zh{回}) &  events & 703& 25.9& \iz{1699:visit}
      \\%& 35,050 & 20.3\\
      \jpn{-tsu} & (\zh{つ})&  abstract/general & 565& 20.9&\iz{2:concrete} 
      
      \\%& 52,921& 30.1      \\
      \jpn{-nin} & (\zh{人}) &  people & 298 & 11.0 & \iz{5:person}
      \\%& 8,545& 4.9       \\   
      \jpn{-ko} & (\zh{個})&  concrete objects & 124& 4.6& \iz{854:fruit} 
      \\%& 14,380& 8.3      \\
      \jpn{-hon} & (\zh{本})&  long thin objects & 52& 1.9& \iz{673:tree}
      \\%& 3,775& 2.1       \\
      \jpn{-mai} & (\zh{枚})&  flat objects & 32& 1.2& \iz{770:paper}
      \\%&2,807 & 1.6      \\
      \jpn{-teki} & (\zh{滴})&  liquid & 21& 0.8& \iz{652:tear}
      \\%&1,219 & 0.7       \\
      \jpn{-dai} & (\zh{台})&  mechanical items& 18 & 0.7&
      \iz{962:machinery} \\
      & & furniture
      \\%& 5,087& 2.9      \\
      \jpn{-hiki} & (\zh{匹})&  animals & 12& 0.6& \iz{537:beast}
      \\%& 1,361& 0.8       \\
      \multicolumn{2}{l}{Other} & 38 classifiers & 91 & 3.4 &
\\\hline
      \multicolumn{2}{l}{Total} & 47 classifiers & 2,710 & 100 &
      \\%& 12,813&  7.4      \\
      \end{tabular}
    \end{center}

% \myslide{Korean Classifiers}
%   \begin{center}
%      \begin{tabular}{lclrrlrr}
%       \multicolumn{2}{l}{\textsc{classifier}} & Referents classified &  
%       No. & \% &  Sample  Class \\ \hline% & No. & \% \\ \hline
%       \multicolumn{2}{l}{None} & Uncountable & 799 & 29.5 
%       & \iz{3:agent} \\% & 34,548& 20.0\\
%       \kor{-kae} & \emp{(개)}& abstract/general & \emp{737} & 27.1 &
%       \iz{2:concrete} \\
%       \kor{-hyoi} & (회) &  events & 707& 26.1& \iz{1699:visit}
%       \\%& 35,050 & 20.3\\
%       \kor{-myong} & (명) &  people & 296 & 10.9 & \iz{5:person}
%       \\%& 8,545& 4.9  
%       \kor{-bangul} & (방울) & liquid & 26 & 1.0 & \iz{652:tear}
%      \\  
%       \kor{-jang} & (장)&  flat objects & 24 & 0.9 & \iz{770:paper}
%       \\%&2,807 & 1.6      \\ 
%  %     \kor{-dae} & (대)&  long thin objects & 21& 1.9& \iz{673:tree}
%   %    \\%& 3,775& 2.1       \\
%        \kor{-dae} & (대)&  mechanical items & 20 & 0.7&
%        \iz{962:machinery} \\   
%        & & furniture \\
%       \kor{-keun} & (건)&  incidents & 14& 0.5 & \iz{1717:contract}
%       \\%&1,219 & 0.7       \\

%      % \\%& 5,087& 2.9      \\
%        \kor{-mari} & (마리)&  animals & 14& 0.5 & \iz{535:animal}
%       \\%& 1,361& 0.8       \\
%        \multicolumn{2}{l}{Other} & 26 classifiers & 73 & 2.7 & 
% \\\hline
%       \multicolumn{2}{l}{Total} & 34 classifiers & 2,710 & 100 &
%      \\%& 12,813&  7.4      \\
%       \end{tabular}
% \end{center}

\myslide{Chinese Classifiers}
\vspace*{-1ex}  \begin{center}
    \begin{tabular}{lclrrlrr}
      \multicolumn{2}{l}{\textsc{classifier}} & Referents classified &  
      No. & \% &  Sample Class \\ \hline% & No. & \% \\ \hline
 \multicolumn{2}{l}{None} & Uncountable referents& 765  & 28.2 & \iz{3:agent} \\% & 34,548& 20.0\\
      \chn{-ci4} & (\zh{次}) &  events & 692 & 25.5 & \iz{1699:visit}
      \\%& 35,050 & 20.3\\
%      \chn{-zhong3} & (\zh{种}) & types & 71 & ?? &  \iz{767:product}
%      \\%& 8,545& 4.9  
   \chn{-ge4} & (\zh{个})& general/people & 655 & 24.1 &
   \iz{2:concrete} \\
      \chn{-wei4} & (\zh{位}) & people (\zh{honored}) & 68 & 2.5 &  \iz{228:doctor}
     \\  
      \chn{-quai4} & (\zh{块}) & big objects & 61 & 2.2 &  \iz{461:land}
     \\  
      \chn{-ren2} & (\zh{人}) & people & 39 & 1.4 &  \iz{92:descendants}
     \\  
      \chn{-tiao2} & (\zh{条})& long thin objects & 33 & 1.2 &\iz{417:route}
      \\%&2,807 & 1.6      \\ 
      \chn{-pian4} & (\zh{片}) & parts/pieces  & 25  & 0.9 & \iz{2578:flake}
      \\%&1,219 & 0.7       \\
      \chn{-zhang1} & (\zh{张})& big flat objects  & 23 & 0.8 & \iz{773:board} \\   
     % \\%& 5,087& 2.9 \\ 
      \chn{-ming2} & (\zh{名}) & people (\zh{respected}) & 22 & 0.8 &  \iz{351:expert}
     \\  
      \chn{-di1} & (\zh{滴})& liquid & 20 & 0.7 & \iz{652:tear} \\
     %& 1,361& 0.8 \\ 
   \chn{-jian4} & (\zh{件})& incidents & 19 & 0.7 & \iz{1717:contract} 
           \\%& 1,361& 0.8 \\ 
%    \chn{-??} & (\zh{??})& mechanic items/furniture & 19 & ?? & \iz{962:machinery} \\%& 1,361& 0.8 \\ ask Yao-san
       \multicolumn{2}{l}{Other} & 70 classifiers & 293 & 10.8  &
     \\%& 12,813&  7.4      \\
\hline      \multicolumn{2}{l}{Total} & 81 classifiers & 2,710 & 100 &
      \\%& 12,813&  7.4      \\
      \end{tabular}
%    \caption{Chinese Numeral Classifiers and associated Semantic Classes}
%    \label{tab:nc-chn}
  \end{center}
%\end{table*}

\myslide{Language Differences}

\begin{itemize}
\item   %\emp{34} Korean classifiers at the level of semantic classes\\
  \emp{47} Japanese classifiers at the level of semantic classes\\
  \emp{81} Chinese classifiers at the level of semantic classes
  \begin{itemize}
  \item Around the number a human typically uses (30--80)\\
    More classifiers at the noun level (default classifiers)
  \item Chinese uses more classifiers than Japanese\\
        Chinese has more specific classifiers

  % \item Japanese uses more kinds of classifiers than Korean\\
  %   \cll{つ \jpn{-tsu}, 個 \jpn{-ko}, 本 \jpn{-hon}} \into  \cll{개
  %     \kor{-kae}}
   \end{itemize}
\item No classifiers assigned to \emp{800} semantic classes %need to
                                                            %check for Chinese
  \begin{itemize}
    \item Uncountable, abstract nouns (e.g.  \eng{greed, lethargy}) 
    \item Empty classes
  \end{itemize}
\end{itemize}

\myslide{Noun Classes vs Classifiers}
\MyLogo{\citet{Dixon:1986}}

\noindent\begin{tabular}{lll}
  & \textbf{Noun classes} & \textbf{Classifiers} \\
Size & Small Finite Set & Large Number (low hundreds) \\
Realization & Closed Grammatical System & Separate Morpheme \\
Marking & Also outside the noun word & Only in the noun phrase
\end{tabular}

\begin{itemize}
\item Gender (noun class in e.g., German)
  \begin{itemize}
  \item typically 3 (Masculine, Feminine, Neuter)
  \item marked as inflection
  \item marked on determiners, adjective and nouns
  \end{itemize}
\item Numeral Classifiers (in e.g., Japanese)
  \begin{itemize}
  \item typically 30-80 in common use, hundreds exist
  \item separate classifier phrase (numeral/interrogative+classifier)
  \item classifier phrase modifies noun
  \end{itemize}
\end{itemize}

\myslide{Summary}
\MyLogo{}
\begin{itemize}
\item  Semantics motivates syntax
  \begin{itemize}
  \item  But most generalizations fail to cover all examples
  \end{itemize}
\item Argument structure and thematic roles link predicates and their arguments
  \begin{itemize}
  \item \blu{Remember the basic roles and examples}
  \end{itemize}
%\item Grammar relations and thematic roles
\item Dowty's Argument Selection Principle
  \\ prototypical agents and patients are subjects and objects 
\item Problems with thematic roles
\item Noun Classes and Classifiers
\end{itemize}






\myslide{Acknowledgments and References}
\MyLogo{}
\begin{itemize}
% \item Definitions from WordNet: \url{http://wordnet.princeton.edu/}
% \item Images from
%   \begin{itemize}
%   \item the Open Clip Art Library: \url{http://openclipart.org/}
%   \item Steven Bird, Ewan Klein, and Edward Loper (2009) 
%      \textit{Natural Language Processing with Python}, O'Reilly Media
%     \\ \url{www.nltk.org/book}
% \end{itemize}
 \item Video: \textit{Does your dog bite} excerpt from \textit{The Pink Panther Strikes Again} directed by Blake Edwards, starring Peter Sellers.  The Pink Panther Strikes Again is the fifth film in The Pink Panther series and was released in 1976.
   \begin{itemize}
   \item It shows issues of reference and cooperation in dialog
   \end{itemize}
   \begin{small}
\begin{verbatim}
Closeau
   Good day.
   My name is Professor Guy Gabroir...
   medieval castle authority from Marseilles.
   Tell me, do you have a room?
Clerk
    I do not know what a "reum" is.
Closeau
    A Zimmer.
Clerk
    Ah! A room!
Closeau
    That is what I have been saying, you idiot.
    Room.

    Does your dog bite?
Clerk
    No.
Closeau
    Nice doggy.
Dog
    Grrrr <BITE>
Closeau
    I thought you said your dog did not bite.
Clerk
    That is not my dog.
\end{verbatim}
   \end{small}
 \end{itemize}

%\myslide{Bibliography}
% Reading: Jurafsky and Martin (2008) Chapter 20
\small
\bibliographystyle{aclnat}
\bibliography{abb,mtg,nlp,ling}

\clearpage
%\end{CJK}
\end{document}


%%% Local Variables: 
%%% coding: utf-8
%%% mode: latex
%%% TeX-PDF-mode: t
%%% TeX-engine: xetex
%%% End: 

