\documentclass[headrule,footrule]{foils}

%\usepackage{nttfoilhead}
%\newcommand{\myslide}[1]{\foilhead[-25mm]{\raisebox{12mm}[0mm]{\emp{#1}}}}
%\newcommand{\myslider}[1]{\rotatefoilhead[-25mm]{\raisebox{12mm}[0mm]{\emp{#1}}}}
%\newcommand{\myslider}[1]{\rotatefoilhead{\raisebox{-8mm}{\emp{#1}}}}

%\newcommand{\TIPA}[1]{\textipa{\mtcitestyle{#1}}}%
\newcommand{\TIPA}[1]{\mtcitestyle{#1}}

%%
%%%  Macros
%%%
%%% fonts-sil-charis for IPA in week 5

\newcommand{\logo}{HG2002 (2021)}
\usepackage[hidelinks]{hyperref}

\newcommand{\header}[3]{%
  \title{\vspace*{-2ex} \large 
    HG2002 Semantics and Pragmatics
% \thanks{Creative
%       Commons Attribution License: 
%       you are free to share and adapt as long as you give 
%       appropriate credit and add no additional restrictions: 
%       \protect\url{https://creativecommons.org/licenses/by/4.0/}.
%     }
    \\[2ex] \Large  \emp{#2} \\ \emp{#3}}
  \author{\blu{Francis Bond}   \\ 
    \normalsize  \textbf{Division of Linguistics and Multilingual Studies}\\
    \normalsize  \url{http://www3.ntu.edu.sg/home/fcbond/}\\
    \normalsize  \texttt{bond@ieee.org}}
  \MyLogo{\logo}
  % \MyLogo{奈良女子大学:欧米言語情報理論II}
  \date{#1
    \\  \url{https://bond-lab.github.io/Semantics-and-Pragmatics/}
\\[.5ex] \footnotesize Creative  Commons Attribution License:  you are free to share and adapt 
\\[-.25ex] \footnotesize   as long as you give    appropriate credit and add no
additional restrictions: 
\\ \small  \protect\url{https://creativecommons.org/licenses/by/4.0/}.
}
  % \renewcommand{\logo}{#2}
  % \special{! /pdfmark where
  %   {pop} {userdict /pdfmark /cleartomark load put} ifelse
  %   [ /Author (Francis Bond)
  %   /Title (#1: #2)
  %   /Subject (HG2002: Semantics and Pragmatics)
  %   /Keywords (Semantics, Pragmatics, Meaning)
  %   /DOCINFO pdfmark}
  %   }
  \hypersetup{%
    final       = true,
    colorlinks  = true,
    urlcolor    = blue,
    citecolor   = blue,
    linkcolor   = MidnightBlue,
    unicode     = true,
    pdfauthor   = {Francis Bond},
    pdfkeywords = {Semantics, Pragmatics, Meaning},
    pdftitle    = {#1: #2},
    pdfsubject  = {HG2002 Semantics and Pragmatics; License CC BY 4.0}
  }
}


\usepackage[a4paper,landscape]{geometry}
%\usepackage[dvips]{xcolor}
\usepackage[dvipsnames,x11names]{xcolor}
\usepackage{graphicx}
\newcommand{\blu}[1]{\textcolor{blue}{#1}}
\newcommand{\grn}[1]{\textcolor{green}{#1}}
\newcommand{\hide}[1]{\textcolor{white}{#1}}
\newcommand{\emp}[1]{\textcolor{red}{#1}}
\newcommand{\txx}[1]{\textbf{\textcolor{blue}{#1}}}
\newcommand{\lex}[1]{\textbf{\mtcitestyle{#1}}}


\usepackage{amsmath,latexsym}
\usepackage{pifont}
\renewcommand{\labelitemi}{\textcolor{violet}{\ding{227}}}
\renewcommand{\labelitemii}{\textcolor{purple}{\ding{226}}}

\newcommand{\subhead}[1]{\noindent\textbf{#1}\\[5mm]}

\newcommand{\Bad}{\emp{\raisebox{0.15ex}{\ensuremath{\mathbf{\otimes}}}}}
\newcommand{\bad}{*}

\newcommand{\com}[1]{\hfill \textnormal{(\emp{#1})}}%
\newcommand{\cxm}[1]{\hfill \textnormal{(\txx{#1})}}%
\newcommand{\cmm}[1]{\hfill \textnormal{(#1)}}%

\usepackage{relsize,xspace}
\newcommand{\into}{\ensuremath{\rightarrow}\xspace}
\newcommand{\ent}{\ensuremath{\Rightarrow}\xspace}
\newcommand{\nent}{\ensuremath{\not\Rightarrow}\xspace}
\newcommand{\tot}{\ensuremath{\leftrightarrow}\xspace}
\usepackage{url}
\newcommand{\lurl}[1]{\MyLogo{\url{#1}}}

\usepackage{mygb4e}
\let\eachwordone=\itshape
\newcommand{\lx}[1]{\textbf{\textit{#1}}}
\newcommand{\ix}{\ex\it}

\newcommand{\cen}[2]{\multicolumn{#1}{c}{#2}}
%\usepackage{times}
%\usepackage{nttfoilhead}
\newcommand{\myslide}[1]{\foilhead[-25mm]{\raisebox{12mm}[0mm]{\emp{#1}}}\MyLogo{\logo}}
\newcommand{\myslider}[1]{\rotatefoilhead[-25mm]{\raisebox{12mm}[0mm]{\emp{#1}}}}
%\newcommand{\myslider}[1]{\rotatefoilhead{\raisebox{-8mm}{\emp{#1}}}}

\newcommand{\section}[1]{\myslide{}{\begin{center}\Huge \emp{#1}\end{center}}}



\usepackage[lyons,j,e,k]{mtg2e}
\renewcommand{\mtcitestyle}[1]{\textcolor{teal}{\textsl{#1}}}
%\renewcommand{\mtcitestyle}[1]{\textsl{#1}}
\newcommand{\ja}[1]{\mtcitestyle{\makexeCJKactive #1\makexeCJKinactive}}
\newcommand{\chn}{\mtciteform}
\newcommand{\zsm}{\mtciteform}
%\newcommand{\cmn}[1]{make\cjkactive\mtciteform#1\makecjkinactive}
\newcommand{\iz}[1]{\textup{\texttt{\textcolor{blue}{\textbf{#1}}}}}
\newcommand{\con}[1]{\textsc{#1}}
\newcommand{\gm}{\textsc}
\newcommand{\cmp}[1]{{[\textsc{#1}]}}
\newcommand{\sr}[1]{\ensuremath{\langle}#1\ensuremath{\rangle}}
\usepackage[normalem]{ulem}
\newcommand{\ul}{\uline}
\newcommand{\ull}{\uuline}
\newcommand{\wl}{\uwave}
\newcommand{\vs}{\ensuremath{\Leftrightarrow}~}
%%% theta role
\newcommand{\tr}[1]{\textcolor{Chartreuse4}{\textsc{#1}}}
%%% theta grid
\newcommand{\grid}[1]{\ensuremath{\langle}\tr{#1}{\ensuremath{\rangle}}}

%%%
%%% Bibliography
%%%
\usepackage{natbib}
%\usepackage{url}
\usepackage{bibentry}
%\usepackage{CJKutf8}


\usepackage{fontenc}
\usepackage{polyglossia}
\setmainlanguage{english}
\setotherlanguages{tamil}
\setmainfont[Ligatures=TeX]{TeX Gyre Pagella}
\setsansfont[Ligatures=TeX]{TeX Gyre Heros}
\newfontfamily\ipafont{Charis SIL}
\newcommand\ipa[1]{\mtcitestyle{\ipafont #1}}


\usepackage{xeCJK}
\makexeCJKinactive
\newcommand{\zh}[1]{\mtcitestyle{\makexeCJKactive #1\makexeCJKinactive}}
%\newcommand{\ja}[1]{\makexeCJKactive #1\makexeCJKinactive}
\setCJKmainfont{Noto Sans CJK JP}
\setCJKsansfont{Noto Sans CJK SC}
\setCJKmonofont{Noto Sans CJK SC}

\newfontfamily\tamilfont[Script=Tamil]{Noto Sans Tamil}
\newfontfamily\tamilfontsf[Script=Tamil]{Noto Sans Tamil}
\newcommand{\tam}[1]{\texttamil{#1}}
%%% From Tim
\newcommand{\WMngram}[1][]{$n$-gram#1\xspace}
\newcommand{\infers}{$\rightarrow$\xspace}


\usepackage{rtrees,qtree}
\renewcommand{\lf}[1]{\br{#1}{}}
\usepackage{avm}
%\avmoptions{topleft,center}
\newcommand{\ft}[1]{\textsc{#1}}
\renewcommand{\val}[1]{\textit{#1}}
\newcommand{\typ}[1]{\textit{#1}}
\avmfont{\sc}
\avmvalfont{\sc}
\renewcommand{\avmtreefont}{\sc}
\avmsortfont{\it}


%%% From CSLI book
\newcommand{\mc}{\multicolumn}
\newcommand{\HD}{\textbf{H}\xspace}
\newcommand{\el}{\< \>}
\makeatother
\long\def\smalltree#1{\leavevmode{\def\\{\cr\noalign{\vskip12pt}}%
\def\mc##1##2{\multispan{##1}{\hfil##2\hfil}}%
\tabskip=1em%
\hbox{\vtop{\halign{&\hfil##\hfil\cr
#1\crcr}}}}}
\makeatletter

%\usepackage{tipa}
\usepackage{multicol}


\newcommand{\task}{\marginpar{\large ~~~\textbf{?}}}
\newcommand{\sh}[1]{\href{https://www.arthur-conan-doyle.com/index.php?title=#1}{#1}}

\usepackage{tikz}
\usepackage{tikz-qtree}
\usepackage{forest}




\begin{document}
%\begin{CJK}{UTF8}{min}
\header{Lecture 5}{Situations}{}\maketitle

%\include{schedule}

\myslide{Overview}

\begin{itemize}\addtolength{\itemsep}{-1ex}
\item Revision: Truth
  \begin{itemize}
  \item Logic and Truth
  \item Entailment
  \item Presupposition
  \end{itemize}
\item TAM: Tense, Aspect and Modality
\item Mood and Evidentiality
\item Next week: Chapter 6: Participants
\end{itemize}

%%%
%%% this changes each year, so keep separate
%%%
\include{schedule}

\section{Revision: Sentence Relations and Truth}

\myslide{Logic}

\begin{itemize}
\item Classical logic is an attempt to find valid principles of argument and inference.
\\[2ex]
\begin{tabular}{llr}
  $a$ & If something is human then it is mortal & \txx{premise}\\
  $b$ & Socrates is human & \txx{premise}\\ \hline
  $c$ & Socrates is mortal & \txx{conclusion}
\end{tabular}
\item Can we go from $a$ and $b$ to $c$? \hfill {\large Yes}
\item Truth is \txx{empirical}: The premises need to correspond with
  the facts of the world
  \begin{itemize}
  \item Sentences have \txx{truth values} (true, false or unknown)
  \item The state of the world that makes a sentence true or false are its \txx{truth conditions}
  \end{itemize}
\end{itemize}


\myslide{Methods of Argument}

\begin{itemize}
\item \txx{Modus Ponens}
\\[2ex]
  \begin{tabular}{ll}
    $a$ & If something is human then it is mortal \\
    $b$ & Socrates is human \\ \hline
    $c$ & Socrates is mortal
  \end{tabular}
\\ $p \rightarrow q, p \vdash q$
\item \txx{Modus tollens}
\\[2ex]
  \begin{tabular}{ll}
    $a$ & If something is human then it is mortal \\
    $b$ & Zeus is not mortal \\ \hline
    $c$ & Zeus is not human
  \end{tabular}
\\ $p \rightarrow q, \neg q \vdash \neg p$

\newpage
\item \txx{Hypothetical syllogism}
\\[2ex]
 \begin{tabular}{ll}
    $a$ & If something is human then it is mortal \\
    $b$ & If something is mortal then it dies \\ \hline
    $c$ & If something is human then it dies
  \end{tabular}
\\ $p \rightarrow q, q \rightarrow r \vdash p \rightarrow r$
\item \txx{Disjunctive syllogism}
\\ (modus tollendo ponens: affirm by denying)
\\[2ex]
 \begin{tabular}{ll}
    $p$ & Either a human is mortal or a human is immortal \\
    $q$ & A human is not immortal \\ \hline
    $r$ & A human is mortal
  \end{tabular}
\\ $p \oplus q, \neg p \vdash q$
\end{itemize}


\myslide{Empirical truths and connectives}
% \begin{itemize}
% \item \txx{and} ($p \wedge q$)
% \item \txx{or}  ($p \vee q$: disjunction, inclusive or)
% \item \txx{xor} ($p \oplus q$: exclusive or, either or)
% \item \txx{if} ($p \rightarrow q$: if then, material implication)
% % If it doesn’t rain, p = F, the conditional claim cannot be
% % invalidated by whatever is done. (q= T or q =F). p is a
% % sufficient but not necessary condition for q. Some other
% % factor might want to cause me to go to the movies!
% % But not all if.. then…constructions work like that. What are
% % some counterexamples that you can think of? If she is smart,
% % then I would win the Nobel prize (counterfactuals)
% \item \txx{iff} ($p \equiv q$: if and only if) 
%   ($(p \rightarrow q) \wedge (q \rightarrow p)$)
% \item \txx{not} ($\neg p$: contradiction)
% \end{itemize}

% \myslide{Truth Tables}
\begin{center}
  \begin{tabular}{|c|c|c|c|c|c|c|c|}
    \hline
    $p$ & $q$ & $p \rightarrow q$ & $p \wedge q$ & $p \vee q$ 
    & $p \oplus q$ & $p \equiv q$ & $\neg p$\\
    \hline
    &   & if & and & or &  XOR & iff & not  \\
    \hline
    T & T & T & T & T & F & T & F \\ 
    T & F & F & F & T & T & F & F \\  
    F & T & T & F & T & T & F & T\\ 
    F & F & T & F & F & F & T & T\\ \hline
%    \hline
  \end{tabular}
  \begin{itemize}
  \item Words themselves often carry more implications
    \\ \eng{I did A and B} often implies \eng{I did A first}
  \item There are many ways of saying the operations
  \end{itemize}
\end{center}


\myslide{Necessary Truth, A Priori Truth and Analyticity}
\begin{itemize}
\item Arguments from the speaker's knowledge
\begin{itemize}
\item \txx{A priori} truth is truth that is known without experience.
\item \txx{A posteri} truth is truth known from empirical testing.
\end{itemize}
\item Arguments from the facts of the world
  \begin{itemize}
  \item \txx{Necessary truth} is truth that cannot be denied without forcing a
    contradiction.
  \item \txx{Contingent truth} can be contradicted depending on the facts.
  \end{itemize}
\item Arguments from our model of the world
\begin{itemize}
\item \txx{Analytic truth} Truth follows from meaning relations  within the sentence.
\\ \emp{can include word meaning}
\item \txx{Synthetic truth} Agrees with facts of the world.
\end{itemize}
%\item Normally these give the same results, but not always.  Why?
% \\ \textit{Consuming gum is allowed in Singapore}
% \\ Contingent truth? Why?
\end{itemize}

% If we include our model of word meaning in our reasoning, then \eng{an
%   apple is a fruit} is \textbf{analytic}.  So it is important to have
% an explicit model: these models are typically called \txx{ontologies}.

% \begin{itemize}
% \item What about \eng{the apple of my eye}?
% \end{itemize}

% Building an \txx{inference engine} is actually very, very hard, \ldots
% \\ But very useful for question answering

\myslide{Entailment}


\begin{itemize}
\item \txx{Entailment} \\[2ex]
  \begin{tabular}{ll}
    $a$ & The evil overlord  assassinated the man in the red shirt. \\ \hline
    $b$ &  The man  in the red shirt died.
  \end{tabular}
  \\[2ex]
  A sentence $p$ entails a sentence $q$ when the truth of the first ($p$)
  guarantees the truth of the second ($q$), and the falsity of the
  second ($q$) guarantees the falsity of the first ($p$).
\item Sources of Entailment
\begin{itemize}
\item Hyponyms
  \begin{exe}
    \ix I rescued a dog today. \textnormal{vs} I rescued an animal today.
  \end{exe}
\item Paraphrases
  \begin{exe}
    \ix My mom baked a cake.\textnormal{vs}  A cake was baked by my mom.
  \end{exe}
\end{itemize}
\end{itemize}

\myslide{Presuppositions}

\begin{itemize}
\item Many statements assume the truth of something else
  \begin{exe}
    \ex   \begin{xlist}
    \ix Mary's sister bakes the best pies. %(presupposing sentence p)
    \ix Mary has a sister. %(presupposition q)
    \end{xlist}
  \end{exe}
\item Negating the presupposing sentence $a$ doesn't affect the presupposition $b$
 whereas negating an entailing sentence destroys the entailment.
\item Sources of Presuppositions
  \begin{itemize}
  \item Names presuppose that their referents' exist
  \item Clefts (\eng{it was X that Y}); Time adverbial; Comparative
  \item Factive verbs: \eng{realize}; 
    some judgement verbs: \eng{blame}; \ldots
%    some change of state: \eng{stop}
  \end{itemize}
\item  Presupposition is one aspect of a speaker’s strategy of
organizing information for maximum clarity for the listener.
\end{itemize}

\myslide{Language meets Logic (again)}

\begin{itemize}
\item \txx{formal semantics} is also known as
  \begin{itemize}
  \item \txx{truth-conditional semantics}
  \item \txx{model-theoretic semantics}
  \item \txx{Montague Grammar}
  \item \txx{logical semantics}
  \end{itemize}
\item A general attempt to link the meaning of sentences to the
  circumstances of the world: \txx{correspondence theory}
  \begin{itemize}
  \item If the meaning of the sentence and the state of the world
    \emp{correspond} then the sentence is \textbf{true}
  \end{itemize}
\end{itemize}

\myslide{Model-Theoretical Semantics}

\begin{enumerate}
\item Translate from a natural language into a logical language
  with explicitly defined syntax and semantics
\item Establish a mathematical model of the situations that the
  language describes
\item Establish procedures for checking the mapping between the
  expressions in the logical language and the modeled situations.
\end{enumerate}


\myslide{Translating English into a Logical Metalanguage}
%\myslide{Empirical truths and connectives}
\MyLogo{Recall lecture 4}
\begin{itemize}
\item Consider simple sentences
  \begin{itemize}
  \item Represent the predicates by a capital \txx{predicate letter}
    \\ these can be n-ary
  \item Represent the \txx{individual constants} by lower case letters
  \item Represent \txx{variables} by lower case letters (x,y,z)
  \end{itemize}
\item Join simple sentences with logical connectives
\\ treat relative clauses as \txx{and}
  \begin{exe}
    \ex \eng{Bobbie who is asleep writhes}: A(b) $\wedge$ W(b)
    \ex \eng{Bobbie is asleep and Freddie drinks}: A(b) $\wedge$ D(f)
    \ex \eng{Freddie drinks and sleeps}: D(f) $\wedge$ S(f)
    \ex \eng{Freddie doesn't drink beer}: $\neg$ D(f,b)
    \ex \eng{If Freddie drinks whiskey Bobbie sleeps}: D(f,w) \into S(b)
%    \ex \eng{x is asleep}: A(x)
  \end{exe}
\end{itemize}

\myslide{Quantifiers in Predicate Logic}
\MyLogo{$\forall$ and \into; $\exists$ and $\wedge$}

\begin{itemize}
  \item Quantifiers bind variables and scope over predications
    \begin{itemize}
    \item \txx{Universal Quantifier} ($\forall$: \eng{each, every, all})
    \item \txx{Existential Quantifier} ($\exists$: \eng{some, a})
    \end{itemize}
    \begin{exe}
      \ex \eng{All students learn logic}: $\forall$x (S(x)  $\into$ L(x,l))
      \ex \eng{A student learns logic}: $\exists$x (S(x)  $\wedge$ L(x,l))
      \ex \eng{Some students learn logic}: $\exists$x (S(x)  $\wedge$ L(x,l))
      \ex \eng{No students learn logic}: $\neg\exists$x (S(x)  $\wedge$ L(x,l))
      \ex \eng{All students don't learn logic}: $\forall$x (S(x)  $\into$ $\neg$L(x,l))
    \end{exe}
  \item All variables must be bound
\end{itemize}

\myslide{Tutorial Solutions}
\MyLogo{$\forall$ and \into; $\exists$ and $\wedge$}
 \begin{exe}
    \ex \eng{Lancelot hated all dragons} \\
    $\forall$x (D(x)  $\into$ H(l,x))
    \ex \eng{Every dragon feared Lancelot} \\
    $\forall$x (D(x)  $\into$ F(x,l))
    \ex \eng{One dragon feared every knight.} \\
    $\exists$x  (D(x)  $\wedge$ $\forall$y(K(y)  $\into$ F(x,y)))
    \\ or  $\exists$x$\forall$y (D(x)  $\wedge$ (K(y)  $\into$ F(x,y)))
 \\   $\forall$y(K(y)   $\into$ $\exists$x  (D(x)  $\wedge$  F(y,x)))
 \\ or $\forall$y$\exists$x(K(y)   $\into$   (D(x)  $\wedge$  F(y,x)))
    \ex \eng{Somebody searched for the Holy Grail} \\
    $\exists$x  (P(x)  $\wedge$ S(x,h))
\newpage
    \ex \eng{Every dragon did not like spinach} \\
    $\forall$x (D(x)  $\into$ $\neg$L(x,l))
    \\ $\neg\forall$x (D(x)  $\into$ L(x,l))
    \ex \eng{Every dragon who did not like spinach feared Lancelot} \\
    $\forall$x ((D(x)  $\wedge$ $\neg$L(x,l)) $\into$  F(x,l))
    \\ I would accept  $\forall$x ((D(x)  $\into$ $\neg$L(x,l)) $\into$  F(x,l))

    \ex \eng{Not every one searched for the Holy Grail} \\
    $\neg\forall$x (P(x)  $\into$ S(x,h))
    \ex No dragon searched for Lancelot
\\
    $\neg\exists$x (D(x)  $\wedge$ S(x,l))
  \end{exe}

\myslide{Some Advantages in Translating to Predicate Logic}
\MyLogo{}

\begin{itemize}
\item Explicit representation of scope ambiguity
  \begin{exe}
    \ex \eng{Everyone doesn't love semantics}
    \begin{xlist}
          \ex \eng{It is not the case that all people love semantics}:  
          \\ $\neg\forall$x (L(x,s))
          \ex \eng{All people have the property of not loving semantics}: 
          \\ $\forall$x($\neg$L(x,s))
    \end{xlist}
  \end{exe}
\item But the big advantage is in reasoning with the real world
   \\ \txx{denotational semantic analysis}
\end{itemize}





\section{Situations}
\MyLogo{Chapter 5}


\myslide{Situations}

Here we look at the meanings of situations described by sentences: 
in particular how we can talk about time and belief.

 \begin{itemize}
 \item  How are situations classified?
 \item  How does classification affect the way we can talk 
   about these situations?
 \item  How are different types of verbs lexically biased  
   towards describing situation types?
 \end{itemize}

\myslide{Stative or Dynamic}

\begin{itemize}
\item  Differences in states  
  \begin{exe}
    \ex \eng{The museum is open.}
    \ex \eng{The museum opens at nine.}
    \ex \eng{The fruit is ripe.}
    \ex \eng{The fruit is ripening.}
  \end{exe}
\item  A situation can be 
  \begin{itemize}
  \item  \txx{Static}: stable for its duration
  \item  \txx{Dynamic}: change over time
  \item  Which of the above are stative and which dynamic?    
  \end{itemize}
\end{itemize}

\myslide{Semantics motivates Syntax}

\begin{itemize}
\item  There is typically  a correlation between states and 
adjectives, and between verbs and dynamic 
situations.
\begin{exe}
\ex \eng{I am writing a paper.}
\ex \eng{The paper is hard to read.}
\ex \eng{Kim poured water into the glass.}
\ex \eng{The glass is full.}
\end{exe}
\item There are exceptions
\begin{exe}
\ex \eng{Be brave!}
\ex \eng{Sandy is being foolish.}
\ex \eng{She knows what semantics is.}
\ex \eng{He loves cats.}
\ex \eng{The cat has green eyes.}
\end{exe}


\myslide{Different Verb Classes}

\item  Verbs differ in whether they are stative or dynamic.
\begin{exe}
\ex \eng{John knows how to drive.}
\ex \eng{John learned how to drive.}
\end{exe}
\item  \txx{Stative}
  \begin{itemize}
  \item  Steady situation, relatively unchanging
  \item  no reference to an explicit start or endpoint
  \end{itemize}
\item  \txx{Dynamic}
  \begin{itemize}
  \item   Situations that have internal phases    
  \end{itemize}
\end{itemize}

\myslide{Properties of Stative Verbs/Adjectives}

\begin{itemize}
\item    Usually incompatible with progressive aspect
  \begin{exe}
    \ex \eng{John is learning German.}
    \ex \eng{*John is knowing German.}
  \end{exe}
\item  Usually strange with imperatives
  \begin{exe}
    \ex \eng{Learn German!}
    \ex \eng{? Know German!}
  \end{exe}
\item  Exceptions: \lex{remain, have, \ldots}
\end{itemize}

\myslide{Dynamic Verbs}

\begin{itemize}
\item \txx{Durative} vs. \txx{Punctual}
\begin{itemize}
\item  whether situation described by verb lasts for a period of 
  time or not
  \begin{exe}
    \ex \eng{John blinked.} \cmm{punctual}
    \ex \eng{John slept.}  \cmm{durative}
  \end{exe}
\end{itemize}
\item  \txx{Telic/Bounded/Resultative} vs. \txx{Atelic/Unbounded}
\begin{itemize}
\item  whether situation described by verb has a natural point of 
  completion
  \begin{exe}
    \ex \eng{John built a raft.}  \cmm{telic}
    \ex \eng{John gazed at the clouds.}  \cmm{atelic}
  \end{exe}
  If you interrupt a telic process, then it may not finish.
  \item Typically test with \eng{in/for 10 minutes}: telic/atelic
\end{itemize}

\myslide{Depends on the whole sentence}
\begin{exe}
  \ex \eng{John was swimming.} \cmm{atelic}
  \ex \eng{John was swimming in the biathlon.} \cmm{telic}
\end{exe}
\end{itemize}
%\newpage
\begin{itemize}
\item    There is a derivational process to turn atelic into 
telic verbs in some languages.
\begin{itemize}
\item  German: \eng[eat]{essen} \into \eng[finish eating]{aufessen}
\end{itemize}
\item  It can also be done with an auxiliary
\begin{itemize}
\item  Japanese: \eng[write]{kaku}  \into \eng[finish writing]{kaki-oeru}
\end{itemize}
\item  It can also be done with a particle
\begin{itemize}
\item  English: \eng{eat}  \into \eng{eat up}
\end{itemize}
\end{itemize}

\myslide{Punctual verbs}
\begin{itemize}
\item  \txx{Punctual verbs} (Semelfactive) describe events that 
  occur for a brief moment 
\item  They can get an \txx{iterative} interpretation if the 
  duration is prolonged
\begin{exe}
  \ex \eng{John coughed.}
  \ex \eng{John coughed all night.}
  \ex \eng{The traffic lights flashed.}
  \ex \eng{The traffic lights flashed the entire time.}
\end{exe}
\end{itemize}

\myslide{Situation Types}

\begin{tabular}{lcccl}
Situations     & Stative & Durative & Telic & Examples \\ \hline
State         & +       & +        &       & \lex{desire, know} \\
Activity     & $-$       & +        & $-$     & \lex{run, drive a car} \\
Accomplishment & $-$       & +        & +     & \lex{bake, walk to school, build} \\
Punctual       & $-$       & $-$        & $-$     & \lex{knock, flash} \\
Achievement    & $-$       & $-$        & +     & \lex{win,  start}  
\end{tabular}
\begin{exe}
  \ex \eng{Kim desires more cowbell}
  \ex \eng{Sandy drives to school}
  \ex \eng{Hiromi compiled a lexicon}
  \ex \eng{Bobby tapped on the window}
  \ex \eng{Alex lost the race}
\end{exe}
% \myslide{Verb Types}
% \begin{itemize}
% \item Stative know
% \item  Dynamic
%   \begin{itemize}
%   \item  Punctual  flash
%   \item  Durative
%     \begin{itemize}
%     \item  Telic  build
%     \item  Atelic gaze
%     \end{itemize}
%   \end{itemize}
% \end{itemize}
% % Situations Stative Durative Telic Exe
% % States + + desire
% % Activities - + - run
% % Accomplishment - + + bake
% % Punctual - - - knock
% % Achievement - - + win
\section{Tense}

\myslide{TAM}
\begin{center}
  \LARGE \emp{Tense, Aspect and Modality}
\end{center}
\begin{itemize}
\item  We need to distinguish grammatical expression from 
  meaning
  \begin{itemize}
  \item  Tense vs Time
  \item  Grammatical Aspect vs Semantic Aspect
  \item  Mood vs Modality
  \item  Surface Case vs Deep Case
  \end{itemize}
\item  The relation between them is refered to as
  \begin{itemize}
  \item  linking; syntax-semantics interface; grammar
  \end{itemize}
\end{itemize} 
 

\myslide{How Universal is Tense?}
\begin{itemize}
\item  Grammatical tense is different from semantic time
\item  English has \iz{past/non-past}
\item  Latin marks \iz{past/present/future}
\item  Chibemba (Bantu) has \txx{metrical tense}
  \begin{multicols}{2}
  \begin{itemize}
  \item  Remote Past ($<$ yesterday)
  \item  Removed Past (yesterday)
  \item  Near Past (today)
  \item  Immediate Past (past few hours)
  \item Immediate Future (next few hours)
  \item Near Future (today)
  \item Removed Future (tomorrow)
  \item Remote Future ($>$ tomorrow)
  \end{itemize}
\end{multicols}
\end{itemize}

\myslide{Tense and Time}
\begin{itemize}
\item  Locate a situation to with respect to a point in time
  \begin{itemize}
  \item  S = speech point
  \item  R = reference time
  \item  E = event time
  \end{itemize}
\item   Hans Reichenbach (1947)
\end{itemize}

\myslide{Simple Tense}

\begin{itemize}
\item  Past ($R = E < S$) \eng{saw} \hfill
\begin{tabular}[t]{ccc|ccc|ccc}
  \mc{3}{c}{past} &  \mc{3}{c}{present} & \mc{3}{c}{future} \\
&R=E&&&S&&&& \\ \hline
&&&&&&& \\ 
\end{tabular}
\item  Present ($R = S = E$) \eng{see} \hfill
\begin{tabular}[t]{ccc|ccc|ccc}
  \mc{3}{c}{past} &  \mc{3}{c}{present} & \mc{3}{c}{future} \\
&&&&S=R=E&&&&  \\ \hline
&&&&&&& \\ 
\end{tabular}
\item  Future ($S < R = E$) \eng{will see}\hfill
\begin{tabular}{ccc|ccc|ccc}
  \mc{3}{c}{past} &  \mc{3}{c}{present} & \mc{3}{c}{future} \\
&&&&S&&&R=E& \\ \hline
&&&&&&& \\ 
\end{tabular}
\end{itemize}

\myslide{Complex Tense}

\begin{itemize}
\item  Past Perfect ($E < R < S$) \eng{had seen} \hfill
\begin{tabular}[t]{ccc|ccc|ccc}
  \mc{3}{c}{past} &  \mc{3}{c}{present} & \mc{3}{c}{future} \\
E&R&&&S&&&& \\ \hline
&&&&&&& \\ 
\end{tabular}
\\ \eng{By 1939 my Father had seen many arrests}
\item  Future Perfect ($S< E < R$) \eng{will have seen} \hfill
\begin{tabular}[t]{ccc|ccc|ccc}
  \mc{3}{c}{past} &  \mc{3}{c}{present} & \mc{3}{c}{future} \\
&&&&S&&&E&R \\ \hline
&&&&&&& \\ 
\end{tabular}
\\ \eng{By 2039 my son will have seen many things}
\end{itemize}

\myslide{Aspect in English}
\begin{itemize}
\item  Finer grained talking about time!
\item  \txx{Progressive}  is used for ongoing processes (unfinished)
  \begin{itemize}
  \item  \txx{Past Progressive} \eng{I was building the building}
  \item  \txx{Present Progressive} \eng{I am building the building}
  \item  \txx{Future Progressive} \eng{I will be building the building}  %(E < R = S) ???
  \end{itemize}
\item  \txx{Perfect} compares the time to the reference point
 \begin{itemize}
  \item  \txx{Past Perfect} \eng{I had built the building} ($E<R<S$)
  \item  \txx{Present Perfect} \eng{I have built the building}  ($E<R=S$)
  \item  \txx{Future Perfect} \eng{I will have built the building}  ($S<E<R$)
  \end{itemize}
\end{itemize}

\myslide{Aspect more Generally}

\begin{itemize}
\item  \txx{Perfective} focuses on the end point
  \begin{itemize}
  \item  \txx{Completive} \eng{I built the building}
  \item  \txx{Experiential} \eng{I have built the building} %(E < R = S) ???
  \end{itemize}
\item  \txx{Imperfective} 
  \begin{itemize}
  \item  \txx{Progressive} \eng{I was listening/I am listening}
  \item  \txx{Habitual} \eng{I listen to the Goon Show}
\end{itemize}
\item Different languages grammaticalize different things
\end{itemize}

% \item  Progressive
% \item  Habitual
% \item  Completive
% \item  Experiential
\section{Mood}

\myslide{Mood and Modality}
\begin{itemize}
\item  Modality expresses varying degrees of the speaker's 
commitment and belief
\begin{exe}
  \ex \eng{She has left by now.}
  \ex \eng{She must have left by now.}
  \ex \eng{She could have left by now.}
  \ex \eng{She needn't have left by now.}
  \ex \eng{She couldn't have left by now.}
  \ex \eng{She has to leave by now.}
  \ex \eng{She must leave by now.}
  \ex \eng{She can leave now. }
\end{exe}
\end{itemize}

\myslide{Other means of expression}
\begin{itemize}
\item Explicit External Verb
  \begin{exe}
    \ex \eng{I know that $S$}
    \ex \eng{I believe that $S$}
  \end{exe}
\item Adverb or Adjective
  \begin{exe}
    \ex \eng{It is certain that $S$}
    \ex \eng{It is likely that $S$}
    \ex \eng{I will probably $S$}
    \ex \eng{I will definitely $S$}
  \end{exe}

\end{itemize}

\myslide{Knowledge vs Obligation}
\begin{itemize}
\item  \txx{Epistemic modality}: Speaker signals degree of  
knowledge.
\begin{exe}
  \ex \eng{You can drive this car} \cmm{You are able to}
\end{exe}
\item  \txx{Deontic modality}: Speaker signals his/her attitude  
to social factors of obligation and permission.
\begin{itemize}
\item \txx{Permission}
  \begin{exe}
    \ex \eng{You can drive this car}  \cmm{You have permission to}
    \ex \eng{You may drive this car}  
  \end{exe}
\item \txx{Obligation}
  \begin{exe}
    \ex \eng{You must drive this car}  \cmm{You have an obligation to}
    \ex \eng{You ought to drive this car} 
  \end{exe}
% \item  Question: How do you use `must', `confirm', and 
% `sure' in Singlish? ???
\end{itemize}
\end{itemize}

\myslide{Possible Worlds}

\begin{itemize}
\item We can analyze these in terms of \txx{possible worlds}
\item We mark how close a hypothetical case is to reality:
  \begin{exe}
    \ex \eng{It must be/might be/is/can't be \ul{hot outside}}
  \end{exe}
\item Similarly for \txx{conditionals} (condition/consequence)
  \begin{exe}
    \ex \eng{If it is Singapore, it will be hot outside}
    \ex \eng{If it were Singapore, it would be hot outside}
    \ex \eng{If you should go to Singapore, take some cool clothes}
  \end{exe}
\end{itemize}

\myslide{Real vs Hypothetical}
\begin{itemize}
\item  \txx{Realis} is used for things that occur
\item  \txx{Irrealis} is used for things that are not claimed to 
occur (hypotheticals, negation, future)
\item  English doesn't mark this normally
  \begin{exe}
    \ex \eng{If I were to go} \textnormal (subjunctive)
  \end{exe}
\item  What about Singlish? ???
 \begin{exe}
    \ex \eng{I got go.}
    \ex \eng{I sure confirm go.}
    \ex \eng{I maybe go.}
  \end{exe}
\end{itemize}

\myslide{Mood more Generally}
\MyLogo{See Saeed for more examples}
\begin{itemize}
\item Grammatical Inflection used to mark modality is called \txx{mood}
  \begin{itemize}
  \item \txx{indicative} expresses factual statements
  \item \txx{conditional} expresses events dependent on a condition
  \item \txx{imperative} expresses commands
  \item \txx{injunctive} expresses pleading, insistence, imploring
  \item \txx{optative} expresses hopes, wishes or commands 
  \item \txx{potential} expresses something likely to happen
  \item \txx{subjunctive} expresses  hypothetical events; opinions or emotions
  \item \txx{interrogative} expresses questions
\end{itemize}
\item English only really marks imperative and subjunctive
  morphologically on \lex{be} 
  \begin{exe}
    \ex \eng{\ul{Be} good!}
    \ex \eng{If I \ul{were} a rich man}
  \end{exe}
\end{itemize}
\section{Evidentiality}

\myslide{Evidentiality}
\MyLogo{Eastern Pomo (McLendon 2003)}
\begin{itemize}
\item  Some languages must show you gained the evidence
\begin{itemize}\addtolength{\itemsep}{1.5ex}
\item  \txx{nonvisual sensory}:  speaker felt the sensation
  \begin{itemize}
  \item  \TIPA{/p\super ha$\cdot$b\'ek\super h-\ul{ink'e}/}  ``burned, I felt it''
  \end{itemize}
\item  \txx{inferential}: speaker saw circumstantial evidence 
  \begin{itemize}
  \item  \TIPA{/p\super ha$\cdot$b\'ek\super h-\ul{ine}/}  ``must have burned''
  \end{itemize}
\item  \txx{hearsay (reportative)}:   speaker is reporting what was told
  \begin{itemize}
  \item  \TIPA{/p\super ha$\cdot$b\'ek\super h-\ul{le}/} ``burned, they say''
  \end{itemize}
\item  \txx{direct knowledge}:   speaker has direct evidence, probably visual 
  \begin{itemize}
  \item \TIPA{/p\super ha$\cdot$b\'ek\super h-\ul{a}/} ``burned, I saw it''
  \end{itemize}
\end{itemize}
\end{itemize}

\myslide{Evidentiality in English}

We can, and often do, mark evidentiality in English, although it is
not strongly grammaticalized.

\begin{exe}
\ex \eng{Bob is hungry.}
\ex \eng{Bob looks hungry.}
\ex \eng{Bob seems hungry.}
\ex \eng{Bob is apparently hungry.}
\ex \eng{Bob would be hungry by now.}
\ex \eng{Look at those clouds! It's going to rain!}
\ex \eng{Look at those clouds! $^\#$ It will rain!.}
\end{exe}

\myslide{Summary of Situations}
\begin{itemize}\addtolength{\itemsep}{-1ex}
\item Verb/Situation Types
\begin{itemize}
\item Stative 
\item  Dynamic
  \begin{itemize}
  \item  Punctual
  \item  Durative
    \begin{itemize}
    \item  Telic/Resultative 
    \item  Atelic
    \end{itemize}
  \end{itemize}
\end{itemize}
\item Tense/Aspect and Time: R, S and E
\item Modality
  \begin{itemize}
  \item Epistemic
  \item Deontic: Permission, Obligation
  \end{itemize}
\item Evidentiality
\end{itemize}

\myslide{ObJoke}
\MyLogo{\url{http://languagelog.ldc.upenn.edu/nll/?p=15495}}

\begin{itemize}
\item \con{past}, \con{present}, and \con{future} walked into a bar. It was \txx{tense}.
\item Luckily, auxiliary \lex{have} got a booth with a past participle. It was \txx{perfect}.
\end{itemize}



\myslide{Acknowledgments and References}
\begin{itemize}
\item \textit{Anne Elk's Theory on Brontosauruses} is a sketch from the
  thirty-first Monty Python's Flying Circus episode, \textit{The All-England
  Summarize Proust Competition}.
\item Why we do active learning ``Active learning is an approach to instruction that involves actively engaging students with the course material through discussions, problem solving, case studies, role plays and other methods.'':
  \\ \url{https://www.pnas.org/content/early/2019/09/03/1821936116}
  \\ You learn better (even though it may not feel that way)
\end{itemize}

% % \MyLogo{}
% % \begin{itemize}
% % \item Definitions from WordNet: \url{http://wordnet.princeton.edu/}
% % \item Images from
% %   \begin{itemize}
% %   \item the Open Clip Art Library: \url{http://openclipart.org/}
% %   \item Steven Bird, Ewan Klein, and Edward Loper (2009) 
% %      \textit{Natural Language Processing with Python}, O'Reilly Media
% %     \\ \url{www.nltk.org/book}
% % \end{itemize}
% % \item Problems  partially based on exercises from Saeed (2003)
% % \end{itemize}

% %\myslide{Bibliography}
% % Reading: Jurafsky and Martin (2008) Chapter 20
% \small
% \bibliographystyle{aclnat}
% \bibliography{abb,mtg,nlp,ling}



\clearpage
%\end{CJK}
\end{document}


%%% Local Variables: 
%%% coding: utf-8
%%% mode: latex
%%% TeX-PDF-mode: t
%%% TeX-engine: xetex
%%% End: 

