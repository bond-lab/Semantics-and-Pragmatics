\documentclass[headrule,footrule]{foils}
%\usepackage{times}
%\usepackage{multicol}

%\usepackage{nttfoilhead}
%\newcommand{\myslide}[1]{\foilhead[-25mm]{\raisebox{12mm}[0mm]{\emp{#1}}}}
%\newcommand{\myslider}[1]{\rotatefoilhead[-25mm]{\raisebox{12mm}[0mm]{\emp{#1}}}}
%\newcommand{\myslider}[1]{\rotatefoilhead{\raisebox{-8mm}{\emp{#1}}}}

%%% delimiters
\usepackage{stmaryrd}
\newcommand{\den}[2][]{\ensuremath{\llbracket\textnormal{\textcolor{teal}{#2}}\rrbracket^{\ensuremath{#1}}}}

%%
%%%  Macros
%%%
%%% fonts-sil-charis for IPA in week 5

\newcommand{\logo}{HG2002 (2021)}
\usepackage[hidelinks]{hyperref}

\newcommand{\header}[3]{%
  \title{\vspace*{-2ex} \large 
    HG2002 Semantics and Pragmatics
% \thanks{Creative
%       Commons Attribution License: 
%       you are free to share and adapt as long as you give 
%       appropriate credit and add no additional restrictions: 
%       \protect\url{https://creativecommons.org/licenses/by/4.0/}.
%     }
    \\[2ex] \Large  \emp{#2} \\ \emp{#3}}
  \author{\blu{Francis Bond}   \\ 
    \normalsize  \textbf{Division of Linguistics and Multilingual Studies}\\
    \normalsize  \url{http://www3.ntu.edu.sg/home/fcbond/}\\
    \normalsize  \texttt{bond@ieee.org}}
  \MyLogo{\logo}
  % \MyLogo{奈良女子大学:欧米言語情報理論II}
  \date{#1
    \\  \url{https://bond-lab.github.io/Semantics-and-Pragmatics/}
\\[.5ex] \footnotesize Creative  Commons Attribution License:  you are free to share and adapt 
\\[-.25ex] \footnotesize   as long as you give    appropriate credit and add no
additional restrictions: 
\\ \small  \protect\url{https://creativecommons.org/licenses/by/4.0/}.
}
  % \renewcommand{\logo}{#2}
  % \special{! /pdfmark where
  %   {pop} {userdict /pdfmark /cleartomark load put} ifelse
  %   [ /Author (Francis Bond)
  %   /Title (#1: #2)
  %   /Subject (HG2002: Semantics and Pragmatics)
  %   /Keywords (Semantics, Pragmatics, Meaning)
  %   /DOCINFO pdfmark}
  %   }
  \hypersetup{%
    final       = true,
    colorlinks  = true,
    urlcolor    = blue,
    citecolor   = blue,
    linkcolor   = MidnightBlue,
    unicode     = true,
    pdfauthor   = {Francis Bond},
    pdfkeywords = {Semantics, Pragmatics, Meaning},
    pdftitle    = {#1: #2},
    pdfsubject  = {HG2002 Semantics and Pragmatics; License CC BY 4.0}
  }
}


\usepackage[a4paper,landscape]{geometry}
%\usepackage[dvips]{xcolor}
\usepackage[dvipsnames,x11names]{xcolor}
\usepackage{graphicx}
\newcommand{\blu}[1]{\textcolor{blue}{#1}}
\newcommand{\grn}[1]{\textcolor{green}{#1}}
\newcommand{\hide}[1]{\textcolor{white}{#1}}
\newcommand{\emp}[1]{\textcolor{red}{#1}}
\newcommand{\txx}[1]{\textbf{\textcolor{blue}{#1}}}
\newcommand{\lex}[1]{\textbf{\mtcitestyle{#1}}}


\usepackage{amsmath,latexsym}
\usepackage{pifont}
\renewcommand{\labelitemi}{\textcolor{violet}{\ding{227}}}
\renewcommand{\labelitemii}{\textcolor{purple}{\ding{226}}}

\newcommand{\subhead}[1]{\noindent\textbf{#1}\\[5mm]}

\newcommand{\Bad}{\emp{\raisebox{0.15ex}{\ensuremath{\mathbf{\otimes}}}}}
\newcommand{\bad}{*}

\newcommand{\com}[1]{\hfill \textnormal{(\emp{#1})}}%
\newcommand{\cxm}[1]{\hfill \textnormal{(\txx{#1})}}%
\newcommand{\cmm}[1]{\hfill \textnormal{(#1)}}%

\usepackage{relsize,xspace}
\newcommand{\into}{\ensuremath{\rightarrow}\xspace}
\newcommand{\ent}{\ensuremath{\Rightarrow}\xspace}
\newcommand{\nent}{\ensuremath{\not\Rightarrow}\xspace}
\newcommand{\tot}{\ensuremath{\leftrightarrow}\xspace}
\usepackage{url}
\newcommand{\lurl}[1]{\MyLogo{\url{#1}}}

\usepackage{mygb4e}
\let\eachwordone=\itshape
\newcommand{\lx}[1]{\textbf{\textit{#1}}}
\newcommand{\ix}{\ex\it}

\newcommand{\cen}[2]{\multicolumn{#1}{c}{#2}}
%\usepackage{times}
%\usepackage{nttfoilhead}
\newcommand{\myslide}[1]{\foilhead[-25mm]{\raisebox{12mm}[0mm]{\emp{#1}}}\MyLogo{\logo}}
\newcommand{\myslider}[1]{\rotatefoilhead[-25mm]{\raisebox{12mm}[0mm]{\emp{#1}}}}
%\newcommand{\myslider}[1]{\rotatefoilhead{\raisebox{-8mm}{\emp{#1}}}}

\newcommand{\section}[1]{\myslide{}{\begin{center}\Huge \emp{#1}\end{center}}}



\usepackage[lyons,j,e,k]{mtg2e}
\renewcommand{\mtcitestyle}[1]{\textcolor{teal}{\textsl{#1}}}
%\renewcommand{\mtcitestyle}[1]{\textsl{#1}}
\newcommand{\ja}[1]{\mtcitestyle{\makexeCJKactive #1\makexeCJKinactive}}
\newcommand{\chn}{\mtciteform}
\newcommand{\zsm}{\mtciteform}
%\newcommand{\cmn}[1]{make\cjkactive\mtciteform#1\makecjkinactive}
\newcommand{\iz}[1]{\textup{\texttt{\textcolor{blue}{\textbf{#1}}}}}
\newcommand{\con}[1]{\textsc{#1}}
\newcommand{\gm}{\textsc}
\newcommand{\cmp}[1]{{[\textsc{#1}]}}
\newcommand{\sr}[1]{\ensuremath{\langle}#1\ensuremath{\rangle}}
\usepackage[normalem]{ulem}
\newcommand{\ul}{\uline}
\newcommand{\ull}{\uuline}
\newcommand{\wl}{\uwave}
\newcommand{\vs}{\ensuremath{\Leftrightarrow}~}
%%% theta role
\newcommand{\tr}[1]{\textcolor{Chartreuse4}{\textsc{#1}}}
%%% theta grid
\newcommand{\grid}[1]{\ensuremath{\langle}\tr{#1}{\ensuremath{\rangle}}}

%%%
%%% Bibliography
%%%
\usepackage{natbib}
%\usepackage{url}
\usepackage{bibentry}
%\usepackage{CJKutf8}


\usepackage{fontenc}
\usepackage{polyglossia}
\setmainlanguage{english}
\setotherlanguages{tamil}
\setmainfont[Ligatures=TeX]{TeX Gyre Pagella}
\setsansfont[Ligatures=TeX]{TeX Gyre Heros}
\newfontfamily\ipafont{Charis SIL}
\newcommand\ipa[1]{\mtcitestyle{\ipafont #1}}


\usepackage{xeCJK}
\makexeCJKinactive
\newcommand{\zh}[1]{\mtcitestyle{\makexeCJKactive #1\makexeCJKinactive}}
%\newcommand{\ja}[1]{\makexeCJKactive #1\makexeCJKinactive}
\setCJKmainfont{Noto Sans CJK JP}
\setCJKsansfont{Noto Sans CJK SC}
\setCJKmonofont{Noto Sans CJK SC}

\newfontfamily\tamilfont[Script=Tamil]{Noto Sans Tamil}
\newfontfamily\tamilfontsf[Script=Tamil]{Noto Sans Tamil}
\newcommand{\tam}[1]{\texttamil{#1}}
%%% From Tim
\newcommand{\WMngram}[1][]{$n$-gram#1\xspace}
\newcommand{\infers}{$\rightarrow$\xspace}


\usepackage{rtrees,qtree}
\renewcommand{\lf}[1]{\br{#1}{}}
\usepackage{avm}
%\avmoptions{topleft,center}
\newcommand{\ft}[1]{\textsc{#1}}
\renewcommand{\val}[1]{\textit{#1}}
\newcommand{\typ}[1]{\textit{#1}}
\avmfont{\sc}
\avmvalfont{\sc}
\renewcommand{\avmtreefont}{\sc}
\avmsortfont{\it}


%%% From CSLI book
\newcommand{\mc}{\multicolumn}
\newcommand{\HD}{\textbf{H}\xspace}
\newcommand{\el}{\< \>}
\makeatother
\long\def\smalltree#1{\leavevmode{\def\\{\cr\noalign{\vskip12pt}}%
\def\mc##1##2{\multispan{##1}{\hfil##2\hfil}}%
\tabskip=1em%
\hbox{\vtop{\halign{&\hfil##\hfil\cr
#1\crcr}}}}}
\makeatletter

%\usepackage{tipa}
\usepackage{multicol}


\newcommand{\task}{\marginpar{\large ~~~\textbf{?}}}
\newcommand{\sh}[1]{\href{https://www.arthur-conan-doyle.com/index.php?title=#1}{#1}}

\usepackage{tikz}
\usepackage{tikz-qtree}
\usepackage{forest}

\usepackage{colortbl}
\definecolor{Gray}{gray}{0.9}

\begin{document}
%\begin{CJK}{UTF8}{min}
\header{Lecture 4}{Sentence Relations, Truth and Models}{}\maketitle

%\include{schedule}

\myslide{Overview}

\begin{itemize}\addtolength{\itemsep}{-1ex}
\item Revision: Word Meaning
  \begin{itemize}
  \item Defining \lex{word}
  \item Lexical  and Derivational Relations
  \item Lexical Universals
 \end{itemize}
\item Logic and Truth
\item Necessary Truth, A Priori Truth and Analyticity
\item Entailment
\item Logical Metalanguage (10.2--3)
\item Semantics and Models (10.4--5)
\item Presupposition
\item Next week: Chapter 5: Situations
\end{itemize}

%%%
%%% this changes each year, so keep separate
%%%
\include{schedule}
\section{Revision: \\ Word Meaning}

\myslide{Word meaning}
\begin{itemize}
\item What is a word? How easy is it to define ‘word’?
\item Lexical and grammatical words
\item Lexical Relations
\item Derivational Relations
  \begin{itemize}
  \item Inchoative, causative, conative, \ldots (\txx{alternations})
  \item Agentive nouns
  \end{itemize}
\item Meaning: Relative or universal?
\end{itemize}

\myslide{Words} 
\begin{description}
\item \txx{word} slippery to define: orthographic, phonological, conceptual definitions mainly overlap
\item \txx{lexeme} base (uninflected) form of a word (or multi word expression)
\item \txx{vagueness} having an underspecified meaning
\item \txx{ambiguous} having more than one possible meaning
\item \txx{content word} with a denotation (typically open class : \textbf{lexical word})
\item \txx{function word} no denotation (typically closed class:
  \textbf{grammatical word}, \textbf{structural word})
\end{description}

\myslide{Senses and Relations}

\begin{description}
\item \txx{polysemous} having multiple meanings
\item \txx{monosemous} having just one meaning
\item \txx{homonyms} words unrelated meaning; grammatically equivalent;
  with identical forms
\end{description} 


\myslide{Lexical Relations}
\begin{description}
\item \txx{synonymy}  all meanings identical; in all contexts; descriptive and non-
\item \txx{hyponymy} is-a, kind-of: supertype \textbf{hypernym}; subtype \textbf{hyponym}
\item \txx{meronymy} part-whole: part \textbf{meronym}; whole \textbf{holonym}
\item \txx{antonymy} (complementary, gradable, reverse, converse, taxonomic sisters)
\item \txx{member-collection} member of a group (\eng{tree-forest})
\item \txx{portion-mass} element of stuff (\eng{grain-rice})
\item \txx{domain}  used in a domain (\eng{[software] driver -golf})
\end{description}

\section{Sentence Relations and Truth}
\newpage

\myslide{Meanings can be related}
\begin{exe}
\ex  A and B are \txx{synonymous}: A means the same as B
\begin{xlist}
  \ex \eng{My brother is a bachelor }
  \ex \eng{My brother has never married.}
\end{xlist}
\ex A \txx{entails} B: if we know A then we know B
\begin{xlist}
  \ex \eng{The child killed the cat.  }
  \ex \eng{The cat is dead.}
\end{xlist}
\ex A \txx{contradicts} B: A is inconsistent with B 
\begin{xlist}	
  \ex \eng{Fred has long hair.}
  \ex \eng{Fred is bald.}
\end{xlist}
\newpage
 \ex A \txx{presupposes} B: B is part of the assumed background of A 
\begin{xlist}	
  \ex \eng{ The King of Pop is dead.}
  \ex \eng{There was a King of Pop}
  \ix \eng{I regret eating your lunch.}
  \ex \eng{I ate your lunch.}
\end{xlist}
\ex A is necessarily true --- \txx{tautology}: A is true but not informative  
\begin{xlist}	
  \ex \eng{Smart people are smart.}
\end{xlist}
\ex A is necessarily false --- \txx{contradiction}: A is inconsistent with itself
\begin{xlist}	
  \ex \eng{ ?It is entirely made of copper and it is not made of metal.}
  \ex \eng{A is not A. }
\end{xlist}
\end{exe}

% \myslide{Logic}




% \begin{itemize}
% \item Empirical truth, a priori truth, analytic and synthetic truths, necessary truth
% \item Remember the truth tables of common connectives:
%   \\ if (entailment), and, or, XOR, iff 
%   \item Entailment \hfill [A \textbf{entails} B: B follows from A; if A then always B]
%   \begin{itemize}
%   \item Hyponyms
%   \item Paraphrases
%   \end{itemize}
% \item Presupposition
%   \begin{itemize}
%   \item Presuppositional triggers, lexical triggers
%   \end{itemize}
% \item Common ground
% \end{itemize}



\section{Logic, Truth and Argument}

\myslide{Logic}

\begin{itemize}
\item Classical logic is an attempt to find valid principles of argument and inference.
\\[2ex]
\begin{tabular}{llr}
  $a$ & Humans are mortal & \txx{premise} \\
  $b$ & Socrates is human & \txx{premise}\\ \hline
  $c$ & Socrates is mortal & \txx{conclusion}
\end{tabular}
\item Can we go from $a$ and $b$ to $c$? \hfill {\large Yes}
\item Truth is \txx{empirical}: The premises need to correspond with
  the facts of the world
  \begin{itemize}
  \item Sentences have \txx{truth values} (true, false or unknown)
  \item The state of the world that makes a sentence true or false are its \txx{truth conditions}
  \end{itemize}
\end{itemize}


\myslide{Logical Connectives}
\MyLogo{It isn't just saying \textit{no it isn't}}
\begin{itemize}
\item \txx{and} ($p \wedge q$)
\item \txx{or}  ($p \vee q$: disjunction, inclusive or)
\item \txx{xor} ($p \oplus q$: exclusive or, either or)
\item \txx{if} ($p \rightarrow q$: if then, material implication)
% If it doesn’t rain, p = F, the conditional claim cannot be
% invalidated by whatever is done. (q= T or q =F). p is a
% sufficient but not necessary condition for q. Some other
% factor might want to cause me to go to the movies!
% But not all if.. then…constructions work like that. What are
% some counterexamples that you can think of? If she is smart,
% then I would win the Nobel prize (counterfactuals)
\item \txx{iff} ($p \equiv q$: if and only if) 
  ($(p \rightarrow q) \wedge (q \rightarrow p)$)
\item \txx{not} ($\neg p$: contradiction)
\end{itemize}

\begin{quote}
  An \txx{argument} is a connected series of statements attempting to
  establish a proposition.
\end{quote}

\myslide{Truth Tables}
\MyLogo{\textit{Yes it is}}
\begin{center}
  \begin{tabular}{|c|c|c|c|c|c|c|c|}
    \hline
    $p$ & $q$ & $p \rightarrow q$ & $p \wedge q$ & $p \vee q$ 
    & $p \oplus q$ & $p \equiv q$ & $\neg p$\\
    \hline
    &   & if & and & or &  XOR & iff & not  \\
    \hline
    T & T & T & T & T & F & T & F \\ 
    T & F & F & F & T & T & F & F \\  
    F & T & T & F & T & T & F & T\\ 
    F & F & T & F & F & F & T & T\\ \hline
%    \hline
  \end{tabular}
  \begin{itemize}
  \item Words themselves often carry more implications
    \\ \eng{I did A and B} often implies \eng{I did A first}
  \item There are many ways of saying the operations
  \end{itemize}
\end{center}



\myslide{Modus ponens}
\MyLogo{\textit{No it isn't}}
\begin{center}
  \begin{tabular}{llcl}
    $a$ & All humans are mortal & $p  \rightarrow q$ & 
    \small if someone is human they are mortal\\
    $b$ & Socrates is human & $p$ \\ \hline
    $c$ & Therefore, Socrates is mortal & $q$
  \end{tabular}


  \begin{tabular}{|c|c|c|c|c|c|c|}
    \hline
    $p$ & $q$ & $p \rightarrow q$  \\
    \hline
    \rowcolor{Gray}
    \textbf{T} & T & \textbf{T}  \\ 
    \textbf{T} & F & F  \\ 
    F & T & \textbf{T}  \\ 
    F & F & \textbf{T}  \\ 
    \hline
  \end{tabular}
\end{center}
\begin{itemize}
\item  The way that affirms by affirming (Latin)
\item $p \rightarrow q, p \models q$
\item \txx{material implication}  (Not quite the same as English \lex{if})
\end{itemize}

\myslide{Modus tollens}
\MyLogo{\textit{Yes it is}}
\begin{center}
  \begin{tabular}{llc}
    $a$ & If something is human then it is mortal & $p  \rightarrow q$\\
    $b$ & Zeus is not mortal  & $\neg q$ \\ \hline
    $c$ & Zeus is not human   & $\neg p$
  \end{tabular}

  \begin{tabular}{|c|c|c|c|c|c|c|}
    \hline
    $p$ & $q$ & $p \rightarrow q$  \\
    \hline
    T & T & \textbf{T}  \\ 
    T & \textbf{F} & F  \\ 
    F & T & \textbf{T}  \\ 
    \rowcolor{Gray}
    F & \textbf{F} & \textbf{T}  \\ 
    \hline
  \end{tabular}
\end{center}
\begin{itemize}
\item  The way that negates by negating (Latin)
\item $p \rightarrow q, \neg q \models \neg p$
\end{itemize}

\myslide{Other types of syllogisms}
\MyLogo{\textit{No it isn't}}
(deductive reasoning)
\begin{itemize}
\item \txx{Hypothetical syllogism}
\\[2ex]
 \begin{tabular}{lll}
    $a$ & If something is human then it is mortal &  $p \rightarrow q$ \\
    $b$ & If something is mortal then it dies & $q \rightarrow r$ \\ \hline
    $c$ & If something is human then it dies &  $p \rightarrow r$
  \end{tabular}
\\ $p \rightarrow q, q \rightarrow r \models p \rightarrow r$

\item \txx{Disjunctive syllogism}
\\ (modus tollendo ponens: affirm by denying)
\\[2ex]
 \begin{tabular}{lll}
    $a$ & Either a human is immortal or a human is mortal & $p \oplus q$ \\
    $b$ & A human is not immortal  & $ \neg p $ \\ \hline
    $c$ & A human is mortal & $q$
  \end{tabular}
\\ $p \oplus q, \neg p \models q$
\end{itemize}

\myslide{Bad Arguments}
\MyLogo{And many, many more}
\begin{itemize}
\item Formal
  \begin{itemize}
  \item \txx{Affirming the consequent}: $p \rightarrow q, q \models p$
    \\ \eng{professors talk too much}, \eng{you talk too much}
    \\ $\vdash$ \eng{you are a professor}
  \end{itemize}
\item Informal
  \begin{itemize}
  \item \txx{Equivocation}:  The sign said "fine for parking here", and since it was fine, I parked there. 
  \item \txx{No True Scotsman}: X doesn't do Y; $a$ is an X and does Y;
    \\ $a$ is not a true X
  \item \txx{Slippery Slope}: We mustn't allow text abbreviations or students will not be able to write normal text.
  \item \txx{False Dilemma}: You are with us or against us
  \item \txx{Guilt by Association}: Hitler was a vegetarian \\
    $\vdash$ vegetarianism is bad
  \end{itemize}
\end{itemize}

\myslide{Bad Arguments, Fake News, Conspiracy Theories}

These are important topics, but somewhat out of the scope of this
course.  If you are interested, here are some good links.

\begin{itemize}
\item
  \href{https://crankyuncle.com/a-history-of-flicc-the-5-techniques-of-science-denial/}{A
    history of FLICC}: Fake experts, Logical fallacies, Impossible expectations, Cherry picking, and Conspiracy theories.
  
\item \href{https://academic.oup.com/eurpub/article/19/1/2/463780}{Denialism: what is it and how should scientists respond?}
\item \href{https://www.climatechangecommunication.org/conspiracy-theory-handbook/}{The Conspiracy Theory Handbook}
\end{itemize}


\section{Necessary Truth, A Priori Truth and Analyticity}

\myslide{Other sorts of truth}
\MyLogo{\eng{Is \ul{Elizabeth} the \ul{queen of England}?  Is the \ul{queen} a \ul{woman}?}}
\begin{exe}
  \ex \eng{My sister is my sister.}
  \ex ?\eng{She was murdered but she is still alive.}
\end{exe}

\noindent Can a statement be known to be true without checking the facts of the
world?

\begin{itemize}
\item Arguments from the speaker's knowledge
\begin{itemize}
\item \txx{A priori} truth is truth that is known without experience.
\item \txx{A posteri} truth is truth known from empirical testing.
\end{itemize}
\item Arguments from the facts of the world
  \begin{itemize}
  \item \txx{Necessary truth} is truth that cannot be denied without forcing a
    contradiction.
  \item \txx{Contingent truth} can be contradicted depending on the facts.
  \end{itemize}
\item Arguments from our model of the world
\begin{itemize}
\item \txx{Analytic truth} Truth follows from meaning relations  within the sentence.
\\ need to know word meaning
\item \txx{Synthetic truth} Agrees with facts of the world.
\end{itemize}
\item Normally these give the same results, but not always.  Why?
% \\ \textit{Consuming gum is allowed in Singapore}
% \\ Contingent truth? Why?
\end{itemize}

If we include our model of word meaning in our reasoning, then \eng{an
  apple is a fruit} is \textbf{analytic}.  So it is important to have
an explicit model: these models are typically called \txx{ontologies}.

\begin{itemize}
\item What about \eng{the apple of my eye}?
\end{itemize}

Building an \txx{inference engine} is actually very, very hard, \ldots
\\ But very useful for question answering

\section{Entailment}

\myslide{A truth based approach to entailment}
\begin{itemize}
\item \txx{Entailment} \\[2ex]
  \begin{tabular}{ll}
    $a$ & The evil overlord  assassinated the man in the red shirt. \\ \hline
    $b$ &  The man  in the red shirt died.
  \end{tabular}
  \\[2ex]
  A sentence $a$ entails a sentence $b$ when the truth of the first ($a$)
  guarantees the truth of the second ($b$), and the falsity of the
  second ($a$) guarantees the falsity of the first ($b$).
\begin{center}
  \begin{tabular}{|c|c|c|l}
    \cline{1-3}
    $a$ &  & $b$   \\
    \cline{1-3}
    T & $\rightarrow$  & T  \\ 
    F & $\rightarrow$  & F,T & don't care\\ 
    F & $\leftarrow$  & F  \\ 
    T, F & $\leftarrow$  & T  & don't care\\ 
    \cline{1-3}
  \end{tabular}
\end{center}
\end{itemize}

\myslide{Sources of Entailment}
\begin{itemize}
\item Hyponyms
  \begin{exe}
    \ex \eng{I rescued a dog today.}
    \ex \eng{I rescued an animal today.}
  \end{exe}
\end{itemize}
\myslide{Paraphrases: Mutual entailment}
\begin{itemize}
\item 
  \begin{exe}
    \ex \eng{My mom baked a cake.}
    \ex \eng{A cake was baked by my mom.}
  \end{exe}
\begin{center}
  \begin{tabular}{|c|c|c|}
    \cline{1-3}
    $p$ &  & $q$   \\
    \cline{1-3}
    T & $\rightarrow$  & T  \\ 
    F & $\rightarrow$  & F \\ 
    F & $\leftarrow$  & F  \\ 
    T & $\leftarrow$  & T  \\ 
    \cline{1-3}
  \end{tabular}
\end{center}
\item This is synonymy
\item What about contradiction?
\end{itemize}


\myslide{The Argument Clinic} 

\begin{itemize}
\item A sketch from episode 29 of
  Monty Python's Flying Circus
\item \textit{An argument is a connected series of statements intended to establish a proposition}
\end{itemize}



\section{Formal Semantics}
\MyLogo{A very brief overview --- doing it properly requires a whole course}

\myslide{Language meets Logic (again)}


\begin{itemize}
\item \txx{formal semantics} is also known as
  \begin{itemize}
  \item \txx{truth-conditional semantics}
  \item \txx{model-theoretic semantics}
  \item \txx{Montague Grammar}
  \item \txx{logical semantics}
  \end{itemize}
\item A general attempt to link the meaning of sentences to the
  circumstances of the world: \txx{correspondence theory}
  \begin{itemize}
  \item If the meaning of the sentence and the state of the world
    \emp{correspond} then the sentence is \textbf{true}
  \end{itemize}
\end{itemize}

\myslide{Model-Theoretical Semantics}

\begin{enumerate}
\item Translate from a natural language into a logical language
  with explicitly defined syntax and semantics
  \\[1ex] \eng{Fran is alive} \into 
  \\ \iz{alive(Francis)} or \iz{A(f)} or \iz{alive($e,x$), Francis($x$)}
\item Establish a mathematical model of the situations that the
  language describes
  \\[1ex] Hard to do in general
\item Establish procedures for checking the mapping between the
  expressions in the logical language and the modeled situations.
  \\[1ex] Works for small closed worlds
\end{enumerate}


\section{1: Translating English \\ into a Logical Metalanguage}

\myslide{Empirical truths and connectives}
\MyLogo{Recall lecture 4}
\begin{itemize}
\item \txx{not} ($\neg p$: contradiction: \eng{it is not the case that $p$})
\item \txx{and} ($p \wedge q$)
\item \txx{or}  ($p \vee q$: disjunction, inclusive or)
\item \txx{xor} ($p \oplus q$: exclusive or, either or 
  $\vee_\textnormal{\scriptsize e}$)
\item \txx{if} ($p \rightarrow q$: if then, material implication)
% If it doesn’t rain, p = F, the conditional claim cannot be
% invalidated by whatever is done. (q= T or q =F). p is a
% sufficient but not necessary condition for q. Some other
% factor might want to cause me to go to the movies!
% But not all if.. then…constructions work like that. What are
% some counterexamples that you can think of? If she is smart,
% then I would win the Nobel prize (counterfactuals)
\item \txx{iff} ($p \equiv q$: if and only if) 
  ($(p \rightarrow q) \wedge (q \rightarrow p)$)
\end{itemize}

\myslide{Simple Statements in Predicate Logic}
\MyLogo{Ignore tense for the moment}

\begin{itemize}
\item Consider simple sentences
  \begin{itemize}
  \item Represent the \txx{predicates} by a capital letter
    \\ these can be $n$-ary
  \item Represent the \txx{individual constants} by lower case letters
  \item Represent \txx{variables} by lower case letters (x,y,z)
  \end{itemize}
  \begin{exe}
    \ex \eng{Bobbie is asleep}: A(b)
    \ex \eng{Freddie drinks}: D(f)
    \ex \eng{Freddie drinks beer}: D(f,b)
    \ex \eng{Freddie prefers beer to whiskey}: P(f,b,w)
    \ex \eng{Someone is asleep}: A(x) \hfill (A(x) $\wedge$ P(x)) 
  \end{exe}
\end{itemize}

\myslide{Complex Statements in Predicate Logic}
\MyLogo{Ignore tense for the moment}

\begin{itemize}
\item Join simple sentences with logical connectives
\\ treat relative clauses as \txx{and}
  \begin{exe}
    \ex \eng{Bobbie who is asleep writhes}: A(b) $\wedge$ W(b)
    \ex \eng{Bobbie is asleep and Freddie drinks}: A(b) $\wedge$ D(f)
    \ex \eng{Freddie drinks and sleeps}: D(f) $\wedge$ S(f)
    \ex \eng{Freddie doesn't drink beer}: $\neg$ D(f,b)
    \ex \eng{If Freddie drinks whiskey Bobbie sleeps}: D(f,w) \into S(b)
%    \ex \eng{x is asleep}: A(x)
  \end{exe}
\item If you run out of letters, use two, keep them unique in the
  world you are modeling
 \begin{exe}
    \ex \eng{Bobbie who is asleep snores}: A(b) $\wedge$ Sn(b)
  \end{exe}
\end{itemize}

\myslide{Quantifiers in Predicate Logic}
\MyLogo{Keep ignoring tense, we are also ignoring number}

\begin{itemize}\addtolength{\itemsep}{-1ex}
  \item Quantifiers bind variables and scope over predications
    \begin{itemize}
    \item \txx{Universal Quantifier} ($\forall$: \eng{each, every, all})
    \item \txx{Existential Quantifier} ($\exists$: \eng{some, a})
    \begin{exe}
      \ex \eng{All students learn logic}: $\forall$x (S(x)  $\into$ L(x,l))
      \ex \eng{A student learns logic}: $\exists$x (S(x)  $\wedge$ L(x,l))
      \ex \eng{Some students learn logic}: $\exists$x (S(x)  $\wedge$ L(x,l))
      \ex\label{me} \eng{No students learn logic}: $\neg\exists$x (S(x)  $\wedge$ L(x,l))
      \ex \eng{All students don't learn logic}: $\forall$x (S(x)  $\into$ $\neg$L(x,l))
      \\ logically equivalent to (\ref{me})
    \end{exe}
  \item $\forall$ must check each one (so $\into$)
  \item $\exists$ is falsified by one counter example  (so $\wedge$)
    \end{itemize}

  \item All variables must be bound \\
    If there is an x, y, z it must have a $\forall$ or $\exists$
\end{itemize}

\myslide{Some Advantages in Translating to Predicate Logic}
\MyLogo{Often people omit the P(x), P(y)}

\begin{itemize}
\item Explicit representation of scope ambiguity
  \begin{exe}
    \ex \eng{Everyone loves someone}
    \begin{xlist}
          \ex \eng{Everyone has someone they love}:  
          \\ $\forall$x (P(x) $\into$ $\exists$y (P(y) $\wedge$
          L(x,y))
          \hfill $\approx$ \hfill  $\forall$x$\exists$y (L(x,y))
          \ex \eng{There is some person who is loved by everyone}: 
          \\   $\exists$y (P(y) $\wedge$ $\forall$x (P(x) $\into$ L(x,y))
           \hfill $\approx$ \hfill    $\exists$y$\forall$x (L(x,y))
    \end{xlist}
    \ex \eng{Everyone didn't enjoy the exam}
    \begin{xlist}
          \ex \eng{All the people hated ``didn't enjoy'' the exam}:  
           \\ $\forall$x (P(x)  $\into$  $\neg$E(x,e))
           \hfill $\approx$ \hfill  $\forall$x$\neg$ E(x,e)
          \ex \eng{Not all people enjoyed the exam}:  
          \\ $\neg\forall$x (P(x)  $\into$  E(x,e))
          \hfill $\approx$ \hfill  $\neg\forall$x E(x,e)
    \end{xlist}
  \end{exe}
\item But the big advantage is in reasoning with the real world
   \\ \txx{denotational semantic analysis}
\end{itemize}

\section{2: The Semantics of \\ the Logical Metalanguage\\
(the model of the situations)}

\myslide{Creating a \txx{Model}}

\begin{enumerate}
\item a \txx{semantic interpretation} of the symbols of the predicate logic
\item a \txx{domain}: the model of a situation which identifies the
  linguistically relevant entities, properties and relations
\item a \txx{denotation assignment function}: this is a procedure
  which matches the linguistic elements with the items in the model
  that they denote (a \txx{naming function})
\end{enumerate}

\myslide{Semantic Interpretation of Symbols}

\begin{itemize}
\item Is the denotation correct (does it match the real world)?
  \begin{itemize}
  \item \textbf{Sentence} 
    $p$ is true in situation $v$ if it corresponds with the real world: 
    \\ {\den[v]{p} = T}: the denotatum of $p$ in $v$ is \textbf{true}
    \\ {\den[v]{p} = F}: the sentence $p$ is \textbf{false} in situation $v$
  \item \textbf{Constant} denotation of a constant is the individual entity in question
  \item \textbf{Predicate constants} are sets of individuals for which the predicate holds
\\ {\{$<x, y, z>$: $x$ hands $y$ to $z$\}}
\\ the set of all individuals x, y, z such that x hands y to z
  \end{itemize}

\end{itemize}




\myslide{The Domain}

\begin{itemize}
\item The domain represents the individuals and representations in a situation $v$
\item Consider Joy Division in Manchester, April 1980
  \begin{itemize}
  \item Band Members: Ian Curtis, Bernard Sumner, Peter Hook and Stephen Morris 
  \item Manager: Tony Wilson
  \item Producer: Martin Hannet
  \end{itemize}
\item U = \{Ian, Bernard, Peter, Stephen, Tony, Martin\}
\item Combine this with an assignment function $F$ to form a model
\\ $M_1 = <U_1, F_1>$ (or set of models: $M_2 = <U_1, F_2>$, \ldots)
\end{itemize}

\myslide{Extension}
\MyLogo{Adapted from Wikipedia}
\begin{itemize}
\item The \txx{extension} of a concept or expression is the set of things it denotes.
  \begin{itemize}
  \item The extension of the word \lex{cat} (written \den{cat}) is the
    set of all (past, present and future) cats in the world: the set
    includes Tom, Grumpy Cat, Tama,  and so on.
  \item \den{Wikipedia reader} is every person who has ever read Wikipedia.
  \item The extension of a predicate is all the things for which that predicate holds:
    \den{sing} is everyone who has ever or will ever sing.
  \item The extension of a whole statement, as opposed to a word or
    phrase, is defined as its truth value. So the extension of
    \eng{Wikipedia is useful} is the logical value 'true'.
    \\ \den{Wikipedia is useful} = T
  \end{itemize}
\end{itemize}





\myslide{The Denotation Assignment Function}
\MyLogo{This describes completely a very small world}
\begin{itemize}
\item Match individual constants and predicate constants with the domain
\\ F($x$) returns the \txx{extension} of $x$
    \\ F(i) = Ian; F(b) = Bernard; F(p) = Peter; F(s) = Stephen; 
     \\ ~~~ F(t) = Tony;  F(m) = Martin
 \\ \begin{tabular}{lllll}
      F(J) & = & in Joy Division & = & \{Ian, Bernard, Peter, Stephen\} \\
      F(S) & = & sings & = & \{Ian, Peter\} \\      
      F(G) & = & plays guitar & = & \{Bernard, Ian\} \\
      F(B) & = & plays bass & = & \{Peter\} \\
      F(D) & = & plays drums & = & \{Stephen\} \\
      F(M) & = & is a manager & = & \{Tony\} \\
      F(P) & = & is a producer & = & \{Martin\} \\
      F(F) & = & fires at & = & \{$<$Martin, Tony$>$\} \\
      F(O) & = & (over) produces & = & \{$<$Martin, Ian$>$, $<$Martin, Bernard$>$, \\
      &&&&  $<$Martin, Peter$>$, $<$Martin, Stephan$>$\} \\
    \end{tabular}
\end{itemize}




\section{3: Checking \\ the Truth-Value of Sentences}

\myslide{Evaluating a simple statement}
\begin{itemize}
\item How can we check if \eng{Ian sings},   S(i), is true?
  \begin{itemize}
  \item \den[M_1]{S(i)} = T iff \den[M_1]{i} $\in$ \den[M_1]{S} \\ The
    sentence is true if and only if the extension of \eng{Ian} is part of
    the set defined by \eng{sings} in the model $M_1$
  \item $F_1$(i) = Ian
  \item $F_1$(S) =   \{Ian, Peter\}
  \item Ian $\in$  \{Ian, Peter\}
  \item[$\Rightarrow$] \den[M_1]{S(i)} = T
  \end{itemize}
\item What about  \eng{Martin sings}: S(m)
  \begin{itemize}
%  \item S(m)
  \item $F_1$(m) = Martin
  \item $F_1$(S) =   \{Ian, Peter\}
  \item Martin $\not\in$  \{Ian, Peter\}
  \item[$\Rightarrow$] \den[M_1]{S(m)} = F
  \end{itemize}
\end{itemize}

\myslide{Evaluating a complex statement}
 \begin{itemize}
\item Is \eng{Ian or Peter plays bass} true?
  \begin{itemize}
  \item B(i) $\vee$ B(p)
  \item $F_1$(i) = Ian
  \item $F_1$(p) = Peter
  \item $F_1$(B) =   \{Peter\}
  \item Ian $\in$  \{Peter\} = F
  \item Peter $\in$  \{Peter\} = T
  \item F $\vee$ T = T
  \item[$\Rightarrow$] \den[M_1]{B(i) $\vee$ B(p)} = T \hfill Yes
  \end{itemize}
  The sentence \eng{Ian or Peter plays bass} is true if and only if
  either the extension of \eng{plays Bass} contains \eng{Peter} or the extension
  of \eng{plays Bass} contains \eng{Ian}
\end{itemize}

\myslide{Quantifiers}

\begin{itemize}
\item \eng{Did Martin produce everyone in Joy Division?}
  \begin{itemize}
  \item $\forall$x (J(x) $\into$ O(m,x))
    \begin{itemize}
    \item i $\into$ O(m,i) = ?
    \item b $\into$ O(m,b) = ?
    \item p $\into$ O(m,p) = ?
    \item s $\into$ O(m,s) = ?
    \end{itemize}
  \item T,T,T,T  = ?
  \item[$\Rightarrow$] \den{ $\forall$x(J(x) $\into$ O(m,x))} = T  \hfill Yes
  \end{itemize}
\end{itemize}

\myslide{What are the advantages?}


\begin{itemize}
\item If we can make a translation and define our model
  \begin{itemize}
  \item we can evaluate truth explicitly
  \item we can relate utterances to situations
  \item we can deal with quantification and compositionality
  \item we can automate the reasoning
  \end{itemize}
\item More in week 10
\end{itemize}



\section{Presupposition}

\myslide{Presuppositions}

\begin{itemize}
\item Many statements assume the truth of something else
  \begin{exe}
    \ex   \begin{xlist}
    \ex \eng{Mary's sister bakes the best pies.} %(presupposing sentence p)
    \ex \eng{Mary has a sister.} %(presupposition q)
    \end{xlist}
  \end{exe}
\item Negating the presupposing sentence $a$ doesn't affect the presupposition $b$
\item Names presuppose that their referents exist
\item Triggers 
  \begin{itemize}
  \item Clefts (\eng{it was X that Y}); Time adverbial; Comparative
  \item Factive verbs: \eng{realize}; 
    some judgement verbs: \eng{blame}; 
    some change of state: \eng{stop}
    \end{itemize}
\end{itemize}

\myslide{Presupposition triggers}
\begin{itemize}
\item  \txx{Cleft construction}
\begin{exe}
\ex  \eng{It was his nonsense that irritated me.}
\ex  \eng{What irritated me was his nonsense.} (pseudo)
\ex  \eng{Something irritated me.} \hfill presupposition
\end{exe}
\item  \txx{Time adverbial}
\begin{exe}
\ex  \eng{I was working five jobs before you went to school}
\ex  \eng{You went to school.} \hfill presupposition
\end{exe}
\item  \txx{Comparative}
\begin{exe}
\ex  \eng{You are even more silly than he is.}
\ex  \eng{He is silly.} \hfill presupposition
\end{exe}
\end{itemize}

\myslide{Presupposition triggers: Lexical triggers}
\begin{itemize}
\item  \txx{Factive verbs} presuppose the truth of their complement clauses.
  \begin{exe}
    \ex \begin{xlist}
      \ex  \eng{The students realized that Alex was hungry.}
      \ex  \eng{The students thought that Alex was hungry.} \hfill no presupposition
    \end{xlist}
    \ex \begin{xlist}
      \ex  \eng{Alex regretted not eating lunch.}
      \ex  \eng{Alex considered not eating lunch.} \hfill no presupposition
    \end{xlist}
  \end{exe}
\item  \txx{Verbs of judgement}
  \begin{exe}
    \ex \eng{Kim blamed me for making a mistake}
  \end{exe}
\item \txx{Change of state} (sometimes)
    \begin{exe}
      \ex \eng{Alex stopped talking to their imaginary friend.}
    \end{exe}
\end{itemize}


% \myslide{Presuppositions}
%   To presuppose something means to assume it

% He’s stopped handing in his homework on time.
% b)  He used to hand in his homework on time.
% a) 

% Semantics or pragmatics?
% Truth relations
% 2)  Communications act
% 1) 

\myslide{Semantic approach}

\begin{itemize}
\item
  \begin{tabular}[t]{cll}
    $p$ & \eng{Mary's sister bakes the best pies}  & presupposing sentence\\
    $q$ & \eng{Mary has a sister}  & presupposition
  \end{tabular}
\begin{center}
    \begin{tabular}{|c|c|c|}
    \cline{1-3}
    $p$ &  & $q$   \\
    \cline{1-3}
    T & $\rightarrow$  & T  \\ 
    F & $\rightarrow$  & T \\ 
    F, T & $\leftarrow$  & T  \\ 
%    T & $\leftarrow$  & T  \\ 
    \cline{1-3}
  \end{tabular}
\end{center}
\item Also true of:
  \begin{tabular}[t]{cll}
   $\neg p$ & \eng{Mary's sister doesn't bakes the best pies}
 \end{tabular}

\item Is that different from this? \\[2ex]
  \begin{tabular}{ll}
$a$ & \eng{I gave my dog a bath today.} \\
$b$ & \eng{I gave an animal a bath today.}
  \end{tabular}
\end{itemize}

\myslide{Presupposition versus entailment}
\begin{itemize}
\item  Negating the presupposing sentence does not affect the
presupposition whereas negating an entailing sentence
destroys the entailment.
\item  Can you think of other examples that show this difference?
\end{itemize}

\myslide{Interactional approach}
\begin{itemize}
\item  Presupposition is one aspect of a speaker’s strategy of
organizing information for maximum clarity for the listener.
\begin{exe}
  \ex  \eng{Mary’s sister bakes the best pies.} 
  \textnormal{
  \begin{xlist}
    \ex  Assertion 1: \eng{Mary has a sister X.}
    \ex  Assertion 2: \eng{X bakes the best pies.}
  \end{xlist}}
\end{exe}
\item  Assertion 1 is in the \txx{background} (old information)
\item  Assertion 2 is in the \txx{foreground} (new information)
\end{itemize}
   
\myslide{Presupposition failure}
\begin{itemize}
\item 
  \begin{exe}
  \ex  \eng{The King of France is bald.}
  \ex  \eng{There is a King of France.} \hfill presupposition
\end{exe}
\item  The problem with names and definite description is that they
presuppose the existence of the named or described entities.
\item  Solution: A speaker’s use of a name or
definite description to refer usually carries a guarantee that the
listener can identify the referent.
\end{itemize}


\myslide{Presupposition and context}
\begin{itemize}
\item  Presuppositions are context dependent. 
\begin{exe}
    \ex \begin{xlist}
      \ex \eng{John ate before going to the movies.}
      \ex  \eng{John went to the movies.} \hfill presupposition
    \end{xlist}
 \ex \begin{xlist}
   \ex \eng{John died before going to the movies}
   \ex  \eng{John went to the movies.} \hfill presupposition
 \end{xlist}
\end{exe}
\item Presuppositions are \txx{defeasible}: they can be canceled
  given the right context.
% due to the context of knowledge
% is known as defeasibility.

\end{itemize}

\myslide{Can we really talk about semantics without context?}
\MyLogo{\eng{ceteris paribus} = ``all other things being equal or held constant''}
\begin{itemize}
\item  Some people argue that presupposition is a pragmatic
phenomenon. It is supposedly part of the set of assumptions
made by participants in a conversation: \txx{common ground}.
\item  What happens if I said \eng{Their child is a teacher.}, and you don’t
already know that they have children?
\item  Lewis (1979) proposes a principle of \txx{accommodation} where 
  \begin{quote}
    if at time $t$ something is said that requires presupposition $p$ to
    be acceptable, and if $p$ is not presupposed just before $t$ then
    --- \eng{ceteris paribus} --- presupposition $p$ comes into existence.
\end{quote}
\item Presuppositions are introduced as new information
\end{itemize}

\myslide{Summary}
\MyLogo{}
\begin{itemize}\addtolength{\itemsep}{-1ex}
% \item Revision: Word Meaning
%   \begin{itemize}
%   \item Defining \lex{word}
%   \item Lexical  and Derivational Relations
%   \item Lexical Universals
%  \end{itemize}
\item Logic and Truth
\item Necessary Truth, A Priori Truth and Analyticity
\item Logical Metalanguage (10.2--3)
\item Semantics and Models (10.4--5)
\item Entailment
\item Presupposition
\item Next week: \emp{Chapter 5: Situations}

% \item Definitions from WordNet: \url{http://wordnet.princeton.edu/}
% \item Images from
%   \begin{itemize}
%   \item the Open Clip Art Library: \url{http://openclipart.org/}
%   \item Steven Bird, Ewan Klein, and Edward Loper (2009) 
%      \textit{Natural Language Processing with Python}, O'Reilly Media
%     \\ \url{www.nltk.org/book}
% \end{itemize}
% \item Problems  partially based on exercises from Saeed (2003)
\end{itemize}

%\myslide{Bibliography}
% Reading: Jurafsky and Martin (2008) Chapter 20
\small
\bibliographystyle{aclnat}
\bibliography{abb,mtg,nlp,ling}

\clearpage
%\end{CJK}
\end{document}



%%% Local Variables: 
%%% coding: utf-8
%%% mode: latex
%%% TeX-PDF-mode: t
%%% TeX-engine: xetex
%%% End: 

