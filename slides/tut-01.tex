\documentclass{article}

\title{HG2002: Tutorial One}
\author{Francis Bond}
\date{}%2011-08-15}
%\Restriction{}
%\rightfooter{}
%\leftheader{}
%\rightheader{}
\usepackage{mygb4e}
\newcommand{\lex}[1]{\textbf{\textit{#1}}}

\begin{document}
\maketitle

\begin{enumerate}
\item Try and define the following words without a dictionary:
  \begin{exe}
  \ex \lex{high school}
  \ex \lex{rat}
  \ex \lex{mouse}
  \ex \lex{copper}
  \ex \lex{Singapore}
  \ex \lex{noodle}
  \ex \lex{justice}
  \end{exe}
\item Try and define the translation equivalents of the above words in a language of your choice.
  \begin{quote}
    Have you encountered any difficulties?  If so what?
  \end{quote}
\item How compositional are the following compound nouns:
  \begin{exe}
    \ex \lex{redhead}
    \ex \lex{blackmail}
    \ex \lex{blackbird}
    \ex \lex{blackboard}
    \ex \lex{boyfriend}
    \ex \lex{daydream}
    \ex \lex{horseshoe}
    \ex \lex{sky-scraper}
    \ex \lex{spin-doctor}
    \ex \lex{software}
  \end{exe}
  \begin{quote}
    If it is compositional, what is the relation between the first and
    second component?
  \end{quote}
\end{enumerate}



\end{document}
