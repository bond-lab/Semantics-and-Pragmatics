\documentclass[a4paper]{article}

\title{\vspace*{-30mm}HG2002: Tutorial Three}
\author{Francis Bond \url{<bond@ieee.org>}}
\date{}%2011-08-15}

\newcommand{\ans}[1]{\hfill{#1}}
%\newcommand{\ans}[1]{}

\usepackage{multicol}
%\Restriction{}
%\rightfooter{}
%\leftheader{}
%\rightheader{}
\usepackage{mygb4e}
\newcommand{\lex}[1]{\textbf{\textit{#1}}}
\newcommand{\lx}[1]{\textbf{\textit{#1}}}
\newcommand{\ix}{\ex\it}
\newcommand{\con}[1]{\textsc{#1}}
\usepackage{url}
\usepackage[normalem]{ulem}
\newcommand{\ul}[1]{\uline{#1}}
\newcommand{\txx}[1]{\textbf{#1}}

\begin{document}
\maketitle

\begin{enumerate}
\item Synonymy/Antonymy
  \begin{enumerate}
  \item Find a pair of absolute synonyms in a language that you speak.
  \item Decide if the words in the following sets are absolute or 
near synonyms. How do you decide? What
type of criteria have you used?
\begin{exe}
  \ex \lex{tell, say, talk}
  \ex \lex{sad, unhappy}
\end{exe}
\item  Classify the following pairs of opposites
\begin{multicols}{2}
\begin{exe}
  \ex \lex{temporary/permanent} % simple
  \ex \lex{strong/weak} % gradable
  \ex \lex{open/shut} % simple, reverse
  \ex \lex{monarch/subject} % converse 
  \ex \lex{advance/retreat} % reverse
  \ex \lex{buyer/seller} % converse
  \ex \lex{clean/dirty} % gradable
  \ex \lex{present/absent} % simple
  \ex \lex{red/green} % taxonomic sister
  \ex \lex{yesterday/today} %  taxonomic sister
\end{exe}
\end{multicols}
\end{enumerate}

\item  Below are some nouns ending in \lex{-er} and \lex{-or}. Using your intuition
about their meanings, discuss their status as agentive nouns. In
particular, are they derivable by regular rule or would they need to
be listed in the lexicon?
\begin{quote}
\lex{author, blazer, blinker, choker, crofter, debtor, loner, mentor,
reactor, roller, lecturer}
\end{quote}
Check your decisions against a dictionary's entries.
\item For the following sets of sentences, discuss the meaning relations
  between the underlined words.
  \begin{exe}
    \ex 
    \begin{xlist}
      \ix The wind \ul{shut} the door.
      \ix The door \ul{shut} with a bang.
    \end{xlist}
    \ex 
    \begin{xlist}
      \ix The student \ul{slept} all through the class.
      \ix The student \ul{snored} all through the class.
    \end{xlist}
   \ex 
    \begin{xlist}
      \ix Kim \ul{bought} a text book from Sandy.
      \ix Sandy \ul{sold} a textbook to Kim.
      \ix Kim \ul{obtained} a text book from Sandy.
    \end{xlist}
   \ex 
    \begin{xlist}
      \ix Bobby \ul{showed} her answers to  Hiromi.
      \ix Hiromi \ul{saw} Bobby's answers.
      \ix Hiromi \ul{looked at} Bobby's answers.
    \end{xlist}
  \end{exe}

\end{enumerate}


\vfill
\paragraph{Acknowledgments} Some of these questions are partially
based on exercises from Saeed (2003, ch3)
\end{document}
