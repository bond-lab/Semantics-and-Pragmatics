\documentclass[a4paper]{article}

\title{HG2002: Tutorial 8\\  Speech Acts}
\author{Francis Bond \url{<bond@ieee.org>}}
\date{}%2011-08-15}
\usepackage[margin=25mm]{geometry}
\newcommand{\ans}[1]{\hfill{#1}}
%\newcommand{\ans}[1]{}

\usepackage{multicol}
%\Restriction{}
%\rightfooter{}
%\leftheader{}
%\rightheader{}
\usepackage{mygb4e}
\newcommand{\lex}[1]{\textbf{\textit{#1}}}
\newcommand{\lx}[1]{\textbf{\textit{#1}}}
\newcommand{\ix}{\ex\it}
\newcommand{\con}[1]{\textsc{#1}}
\usepackage{url}
\usepackage[normalem]{ulem}
\newcommand{\ul}[1]{\uline{#1}}
\newcommand{\txx}[1]{\textbf{#1}}
\newcommand{\com}[1]{\hfill #1}
\begin{document}
\maketitle

\begin{enumerate}

\item The ability to insert \textit{hereby} appropriately is a test for performative utterances. For each of the following sentences below, use this as test to decide which of the following sentences would count as a performative utterance when uttered. 
  \begin{exe}
  \ex \textit{I acknowledge you as my legal heir.}
  \ex \textit{I give notice that I will stand down as chairman.}
  \ex \textit{I'm warning you that it won't end here.}
  \ex \textit{I think you're taking this press attention too seriously.}
  \ex \textit{I deny all knowledge of this scandal.}
  \ex \textit{I promised them there'd be no fuss.}
\end{exe}
\item Identify some direct illocutionary acts for the following utterances.
 \begin{exe}
  \ex \textit{Do you know what time it is?}
  \ex \textit{Hello! }
  \ex \textit{What a beautiful day!}
  \ex \textit{Accept the award!}
  \ex \textit{Stop!}
\end{exe}

\item Below are some examples of indirect speech acts. For each one try to identify both the direct and the indirect act.
 \begin{exe}
\ex {[Travel agent to customer]}
\\ \textit{Why not think about Spain for this summer?}
\ex {[Mother to child coming home from school]}
\\ \textit{I bet you're hungry.}
\ex {[Bank manager to applicant for an overdraft]}
\\ \textit{We regret we are unable to accede to your request.}
\ex {[Someone responding to a friend's staying late]}
\\ \textit{Why don't you leave?}
\ex {[Doorman at a club to aspiring entrant]}
\\ \textit{Don't make me laugh.}
\end{exe}

\item What are the felicity conditions that Searle has identified for requesting? Form different indirect requests with the following strategies below.
  \begin{enumerate}
  \item By querying the preparatory condition of the direct request
  \item By stating the preparatory condition of the direct request
  \item By querying the propositional condition of the direct request
  \item By stating the sincerity condition of the direct request.
  \end{enumerate}

\item Cross-cultural differences in the use of direct versus indirect
  speech acts can lead speakers of one language to stereotype speakers
  of another language as impolite. Discuss any experience you may have
  had of such misunderstandings. Also reflect on how requests and
  other speech acts might differ in their directness in the languages
  that you speak. Try to come up with specific examples of
  differences.

\end{enumerate}
\vfill
\paragraph{Acknowledgments} These questions are partially
based on exercises from Saeed (2003).
\end{document}
