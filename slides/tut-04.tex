\documentclass[a4paper]{article}

\title{\vspace*{-30mm}HG2002: Tutorial \\  Sentence Relations and Truth}
\author{Francis Bond \url{<bond@ieee.org>}}
\date{}%2011-08-15}

\newcommand{\ans}[1]{\hfill{#1}}
%\newcommand{\ans}[1]{}

\usepackage{multicol}
%\Restriction{}
%\rightfooter{}
%\leftheader{}
%\rightheader{}
\usepackage{mygb4e}
\newcommand{\lex}[1]{\textbf{\textit{#1}}}
\newcommand{\lx}[1]{\textbf{\textit{#1}}}
\newcommand{\ix}{\ex\it}
\newcommand{\con}[1]{\textsc{#1}}
\usepackage{url}
\usepackage[normalem]{ulem}
\newcommand{\ul}[1]{\uline{#1}}
\newcommand{\txx}[1]{\textbf{#1}}

\begin{document}
\maketitle

\begin{enumerate}
\item Take three sentences, $p$, $q$, and $r$ as follows. 

  \begin{quotation}
    \noindent $p$: The sun is shining. \\
    $q$: The day is warm. \\
    $r$: The sun is shining and the day is warm. 
  \end{quotation}

  Let’s make the working assumption that we can represent sentence $r$
  by the logical formula $p \wedge q$. Use the truth table for $\wedge$ ``logical and'' to
  show the truth-value of $r$ in the situations below:
  \begin{quotation}
    \noindent S1. $p$ is true; $q$ is false.\\
    S2. $p$ is true; $q$ is true.\\
    S3. $p$ is false; $q$ is true. \\
    S4. $p$ is false; $q$ is false.
  \end{quotation}
\item  In propositional logic, some may want to assume that $p \wedge q$ and 
$q \wedge p$ are logically
equivalent i.e. that the order of the elements is irrelevant. Discuss how the following
examples show that this is not true for the way that speakers use English and. 
\begin{exe}
  \ix He woke up and saw on TV that he had won the lottery. 
  \ix Combine the egg yolks with water in a bowl and whisk the mixture until 
  foamy.
  \ix  He made two false starts and was disqualified from the race.
  \ix Move and I'll shoot! 
\end{exe}

\item Prove the Contrapositive ($p \rightarrow q \equiv \neg q \rightarrow \neg p$) using truth tables:

\item  Decide if the following pairs of sentences are pairs of entailment or 
presupposition. How did you make your decisions? 

\begin{exe}
\ex
  \begin{xlist}
 \ix Sandy knows that Joe crashed the car. 
 \ix Joe crashed the car. 
\end{xlist}
\ex 
  \begin{xlist}
    \ix Australia is bigger than Singapore. 
    \ix Singapore is smaller than Australia. 
  \end{xlist}
\ex 
  \begin{xlist}
\ix The minister blames her secretary for leaking the memo to the press. 
\ix The memo was leaked to the press. 
\end{xlist}
\ex 
  \begin{xlist}
\ix Everyone passed the examination. 
\ix No one failed the examination. 
\end{xlist}
\ex 
\begin{xlist}
  \ix Fran has resumed their habit of editing Wikipedia. 
  \ix Fran has a habit of editing Wikipedia. 
\end{xlist}
\end{exe}

\newpage
\item  What are analytically true statements? Decide which of the 
following sentences are analytically true. Discuss the reasons for your decision. 
\begin{exe}
  \ix If it rains, we'll get wet. 
  \ix The train will either arrive or it won't arrive. 
  \ix Every doctor is a doctor. 
  \ix If Bobby killed a deer, Bobby killed an animal. 
  \ix Madrid is the capital of Spain. 
  \ix Every city has pollution problems. 
\end{exe}

\item Translate the following into predicate logic, using restricted
  quantifiers $\forall$ and $\exists$.  If a sentence is ambiguous,
  give both readings.
  \begin{exe}
    \ix Lancelot hated all dragons
    \ix Every dragon feared Lancelot
    \ix One dragon feared every knight.
    \ix Somebody searched for the Holy Grail
    \ix Every dragon did not like spinach
    \ix Every dragon who did not like spinach feared Lancelot
% -> ->?
    \ix Not every one searched for the Holy Grail
    \ix No dragon searched for Lancelot
  \end{exe}
\end{enumerate}

\vfill
\paragraph{Acknowledgments} These questions are partially
based on exercises from Saeed (2003).
\end{document}
