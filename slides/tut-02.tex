\documentclass[a4paper]{article}

\title{HG2002: Tutorial Two}
\author{Francis Bond \url{<bond@ieee.org>}}
\date{}%2011-08-15}

%\Restriction{}
%\rightfooter{}
%\leftheader{}
%\rightheader{}
\usepackage{mygb4e}
\newcommand{\lex}[1]{\textbf{\textit{#1}}}
\newcommand{\con}[1]{\textsc{#1}}
\usepackage{url}
\usepackage[normalem]{ulem}
\newcommand{\ul}[1]{\uline{#1}}
\newcommand{\txx}[1]{\textbf{#1}}

\begin{document}
\maketitle

\begin{enumerate}
\item Imagine the following sentences being spoken.  Decide if the
  underlined nominal expressions are being used to refer.
  \begin{exe}
  \ex \textit{I will meet you at \ul{Canteen 9}}
  \ex \textit{They had \ul{no beer}}
  \ex \textit{Jo is going to give \ul{a lecture}}
  \ex \textit{Cool water on the back of the neck is like \ul{rain on a wilted lettuce}}
  \ex \textit{What we need is \ul{an army of volunteers}}
  \ex \textit{Kim wants to marry \ul{a German}}
  \ex \textit{Sandy is married to \ul{a German}}
  \ex \textit{Every evening, \ul{a kangaroo} hops through my backyard.}
  \end{exe}
\item The \txx{description theory of names} sees names as being used
  based on your knowledge of the referent.  Test this theory by
  listing two facts for each name you recognize below:
  \begin{enumerate}
  \item Confucius
  \item Pikachu
  \item Marie Curie
  \item Noam Chomsky
  \item J.K. Rowling
  \end{enumerate}
  Discuss how a \txx{causal theory} might explain your knowledge of
  these names.  Do you think it is possible to combine these theories
  in some way?

\newpage

\item A traditional proposal is that a concept can be defined by a set
  of \txx{necessary and sufficient conditions}, where the right set of
  attributes might define a concept exactly. If words are labels for
  concepts these attributes might also define word meaning. Lehrer
  (1974) discusses the definitions of words associated with
  cooking. Some of these examples are in the two groups below. For
  each word, try to establish sets of attributes that would
  distinguish it from its companion in the group.
  \begin{enumerate}
  \item \lex{cake, biscuit/cookie, bread,  roll, bun, cracker}
  \item \lex{boil, fry, broil, saut\'e, simmer, grill, roast, toast}
  \end{enumerate}
  Evaluate the usefulness and practicality of this approach.

\item Assuming the \txx{prototype} theory of meaning suggest a list of
  characteristic features for the following concepts.  Give some
  examples of each concept, and grade them according to how typical a
  member they are, as Saeed (2003, p37) did for \lex{sparrow} and
  \lex{penguin} as examples of \con{bird}.
  \begin{enumerate}
  \item \con{clothes}
  \item \con{noodle}
  \item \con{work}
  \item \con{mother}
  \item \con{science}
  \item \con{sport}
  \item \con{game}
  \end{enumerate}

\end{enumerate}

\vfill
\paragraph{Acknowledgments} These questions are based on exercises from 
 Saeed (2003, pp 47--49)
\end{document}
