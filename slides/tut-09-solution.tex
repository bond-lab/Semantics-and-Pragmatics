\documentclass[a4paper]{article}

\title{HG2002: Solution to Tutorial 9\\  Componential Analysis}
\author{Francis Bond \url{<bond@ieee.org>}}
\date{}%2011-08-15}
\usepackage[margin=25mm]{geometry}
\newcommand{\ans}[1]{\hfill{#1}}
%\newcommand{\ans}[1]{}
\usepackage{polyglossia}
\setmainlanguage{english}
\setmainfont[Ligatures=TeX]{TeX Gyre Pagella}
\setsansfont[Ligatures=TeX]{TeX Gyre Heros}
\usepackage{xeCJK}
\setCJKmainfont{Noto Sans CJK JP}
\usepackage{multicol}
%\Restriction{}
%\rightfooter{}
%\leftheader{}
%\rightheader{}
\usepackage{mygb4e}
\newcommand{\lex}[1]{\textbf{\textit{#1}}}
\newcommand{\lx}[1]{\textbf{\textit{#1}}}
\newcommand{\ix}{\ex\it}
\newcommand{\con}[1]{\textsc{#1}}
\newcommand{\eng}[1]{\textit{#1}}
\usepackage{url}
\newcommand{\txx}[1]{\textbf{#1}}
\newcommand{\cmp}[1]{{[\textsc{#1}]}}
\newcommand{\sr}[1]{\ensuremath{\langle}#1\ensuremath{\rangle}}
\usepackage[normalem]{ulem}
\newcommand{\ul}{\uline}
\newcommand{\ull}{\uuline}
\newcommand{\wl}{\uwave}
\newcommand{\vs}{\ensuremath{\Leftrightarrow}~}


\begin{document}
\maketitle

\begin{enumerate}
\item Using semantic components, analyze the following words:
  \begin{quote}
  \lex{son, daughter, child, mother, father, parent, grandfather,
    grandmother, grandparent}
  \end{quote}
  Discuss whether a binary format would be an advantage here.
  \\ You may use two place relations in your descriptions (e.g. \cmp{sibling-of[x,y]}.
  \\ If you speak a language that makes additional distinctions in this area, also describe them (e.g. maternal grandmother, \ldots).

  \begin{itemize}
  \item  child(\ul{x},y): +CHILD-OF(x,y)
  \item  son(\ul{x},y): +MALE(x) +CHILD-OF(x,y)
  \item  daughter(\ul{x},y): -MALE(x) +CHILD-OF(x,y)
  \item  parent(\ul{x},y): +CHILD-OF(y,x)
  \item  father(\ul{x},y): +MALE(x) +CHILD-OF(y,x)
  \item  mother(\ul{x},y): -MALE(x) +CHILD-OF(y,x)
  \item  grandparent(\ul{x},z): +CHILD-OF(y,x) +CHILD-OF(y,z) 
  \item  grandfather(\ul{x},z): +MALE(x) +CHILD-OF(y,x)  +CHILD-OF(y,z)
  \item  grandmother(\ul{x},z): -MALE(x) +CHILD-OF(y,x)  +CHILD-OF(y,z)
    \item 兄(\ul{x},y) \textit{ani} ``older brother [of y]'':   +MALE(x)
      +CHILD-OF(x,z)  +CHILD-OF(y,z) +OLDER(x,y)
    \item 弟(\ul{x},y) \textit{otouto} ``younger brother [of y]'':   +MALE(x)
      +CHILD-OF(\ul{x},z)  +CHILD-OF(y,z) +OLDER(y,x)
  \end{itemize}
  We could replace +CHILD-OF(x,y) with +PARENT-OF(y,x);
  \\ we could replace +MALE with {-FEMALE} and -MALE with +FEMALE
  

%%% Note that in English, fact that there are two arguments shows up in det. choice
%%% "My son is sleepy" (not "a/the son is sleepy")
%%% can also do with parent-of and reversed elements, likewise +/-FEMALE
%%% really need two place predicates to get a good def of grandparents
%%% if there is time at the end, maybe consider /ancestor/ :-)
%%% maybe discuss markedness/defaults

\item  \label{cia} Which of the following participate in the
  \textbf{causative/inchoative alternation}.
  \\ \textbf{Note:} your judgements may be different from mine
  \begin{exe}
    \ex \textit{The goalkeeper bounced the ball.}
    \trans Y: \textit{The ball bounced}
    \ex \textit{The assassin murdered the general.}
    \trans N: *\textit{The general murdered} ``died''
    \ex \textit{The waiter melted the chocolate.}
    \trans Y: \textit{The chocolate melted}
    \ex \textit{Charlie built the new swimming pool.}
    \trans N: \textit{The new swimming pool built}
    \ex \textit{The people lowered the boat.}
    \trans N:  *\textit{The boat lowered.}
    \ex \textit{Kim worried Sandy.}
    \trans Y: \textit{Sandy worried.}
    \ex \textit{The censors destroyed the film.}
    \trans N:  *\textit{The film destroyed.}
    \ex \textit{Jo dried the clothes.}
    \trans Y:  \textit{The clothes dried.}
  \end{exe}
  For those verbs that do undergo the alternation, translate them
  into a language of your choice and report on whether the
  translations undergo a similar alternation.
\item Levin and Rapaport Hovav (1995: 102--5) argue that transitive
  verbs which do not undergo the \textbf{causative/inchoative
    alternation} need an intentional and volitional Agent.  In contrast,
  verbs that undergo this alternation should also allow a non-Agent subject:
  \begin{enumerate}
  \item \textit{John broke the window with a rock} \hfill Agent Subject
  \item \textit{The rock broke the window} \hfill Non-Agent (Instrument) Subject 
  \item \textit{The window broke} \hfill Inchoative Alternation
  \end{enumerate}
  Test this hypothesis on the sentences from Question~\ref{cia}.

  \begin{exe}
    \ex ?? \textit{The wall bounced the ball.}
    \ex \textit{The heat melted the chocolate.}
    \ex \textit{The news worried Sandy.}
    \ex \textit{The heat dried the clothes.}
  \end{exe}
It generally seems to be true, but not always
  
\item Consider the following semantic and syntactic tests for countability:
  \begin{itemize}
  \item Semantic: Can it be divided and still use the same name 
    (\textbf{divisibility}):
    \begin{itemize}
    \item Mass:     \eng{half some gold} is \eng{gold}
    \item Count: \eng{half a dog} is not \eng{a dog} 
    \end{itemize}
  \item Syntactic: does it co-occur with \eng{much} or \eng{many}
    (\textbf{enumerability}):
    \begin{itemize}
    \item Mass:  \eng{I don't have \ul{much gold}}
    \item Count: \eng{I don't have \ul{many dogs}} 
    \end{itemize}
  \end{itemize}
  Classify the following nouns using these tests:
  \begin{quote}
    \lex{monkey, program, software, chair, furniture, 
      beer, icecream, curry, chocolate,
      chicken, salmon, potato, rice, oats, mink}
  \end{quote}
  Do the tests always give unique results?  If not, why not?

  \begin{tabular}{lcccl}
    Word & Divisible  & Enumerable  & Countable & Comment \\ \hline
    monkey	&  $-$  &  $+$   &  $+$ \\
    program	&  $-$  &  $+$   &  $+$ \\
    software	&  $+$  &  $-$   &  $-$ \\
    chair	&  $-$  &  $+$   &  $+$ \\
    furniture	&  $+$  &  $-$   &  $+$ \\
    beer	&  $+$  &  $+$   &  $-$ & enumeration gives a kind reading \\
    icecream	&  $+$  &  $+$   &  $-$ & enumeration gives a kind reading  \\
    curry	&  $+$  &  $+$   &  $-$ & enumeration gives a kind reading  \\
    chocolate	&  $+$  &  $+$   &  $-$ & enumeration gives a kind reading  \\
    chicken	&  $+$  &  $-$   &  $+$ & meat \\
    chicken	&  $-$  &  $+$   &  $+$ & bird \\
    salmon	&  $+$  &  $-$   &  $+$ & meat \\
    salmon	&  $-$  &  $+$   &  $+$ & fish  \\
    potato	&  $-$  &  $+$   &  $-$ & the vegetable\\
    potato	&  $+$  &  $-$   &  $-$ & mashed potato\\
    rice	&  $+$  &  $-$   &  $-$ \\
    oats	&  $+$  &  $-$   &  $-$ \\
    mink	&  $+$  &  $-$   &  $-$ & fur \\
    mink	&  $-$  &  $+$   &  $+$ & animal\\
  \end{tabular}
  
\end{enumerate}
\vfill
\paragraph{Acknowledgments} These questions are partially
based on exercises from Saeed (2003).
\end{document}

%%% Local Variables: 
%%% coding: utf-8
%%% mode: latex
%%% TeX-PDF-mode: t
%%% TeX-engine: xetex
%%% End: 
